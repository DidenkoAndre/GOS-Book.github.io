\chapter{TODOLIST}

----------------------------------------------

ТЕКУЩЕЕ СОСТОЯНИЕ

формулы. 5ий параграф Кудрявцева

Билеты. Начало третьей официальной волны переработки. 

----------------------------------------------

1) дописать все билеты полностью. особенно третья часть про кратные интегралы плохо написаны.

4)картиночки везде добавить и поработать над обтеканием их.
 
8)рисунок райгородского в формуле полной вероятности? 
 
9) надо оформить все функции вставляющие картинки как \usepict с автоматическим label и pictures/
 
10) надо сделать рисунок катета и гипотенузы в УКР билета 33
(Как Карлов пояснил вообще написать)
 
12) картинка ,33.4 там Г хотя я в книге использую обозначение партиал Ж для края

13) относительно 12 подумай какое лучше обозначение для края будет с плюсиком или без плюсика. 

14) (\textbf{пока непонятно что делать, но это яковлев 90})

16) написать о геометрической вероятности в 30 билете.

17) в определение зависимости событий

Здесь было бы полезно сказать, что события не могут быть независимыми или зависимыми сами по себе. Свойство независимости событий зависит от введенной вероятностной меры. Если на сигма-алгебре ввести одну вероятностную меру, то события могут оказаться независимыми, а если другую вероятностную меру, то могут оказаться зависимыми.

И уже тем более, никакого отношения к причинно-следственным связям стохастическая независимость не имеет, об этом тоже можно сказать.

18) Обязательно нужна теорема ЗБЧ по Хинчину (которая доказывается с помощью хар. функций) и УЗБЧ Колмогорова (если эти теоремы были в курсе, естественно). И в этом случае надо быть готовым сказать, что такое хар. функция и ее простейшие свойства.

А добавил бы я эти теоремы хотя бы потому, что на практике ими чаще  пользуются. В матстатистике так это вообще рабочие инструменты.

%!!!!!!!!!!!!!!!!!!!!!!!!!!!!!!!!!!!!!!!!!!!!!!!!!!!!!!!!!!!!!!!!!!!!!!!!!!!!!!!!!!!!!!!!!!!!!!!!!!!!!
20) нормальные ссылки \label \ref расставить. (а не label{ksdfnksfjsdf} всякие там, о чем я думал)

35) определение 4 стр 15 половинкин.и подобная ересь.
опред 5 стр 16. Вводить ли снова такие основы основ. 

Или сослаться на материал предыдущих билетов? (очень плохо.)

36) вопрос Лехи Малышева про R и [a,b] и Павла Останина че-то там про дифференцируемость.

42) исправить максимальное количество всевозможных warning'и. и вообще overful и underfull некрасиво смотрятся.

44) (ведь тогда $\exists\, \delta$ такая, что $\dneio{\delta}{x_0}\cap D_f=\emptyset$). это из первого билета.

подумать, как это можно более верно сказать. Противоречие с кольцом. что если всегда для любого эпсилон можно найти кольцо вокруг точки х_0 что свойство предела будет выполнено.

45) переход к пределу в неравенствах используешь в билете 2, но даже не написал формулировки.

46) замечание про точки разрыва с 93 страницы. В связи с чем перегруппировать содержание 2 и 3 билета.

47) рисунок в теорему о промежуточных значениях. собственно ручный.

50) сделать в самом конце различные мини-версии для мини-печати и сделать ссылки на них.

51) формулы Производная сложной функции, параметрически заданной, параграф 13 кудрявцева 1

кривые кудрявцев 511 страница. 255 в пдфке

58) 33 билет разобраться со сноской

105) переписать все формулы в стиле с памяткой. иногда инконсинсетнинти. пробелы у кванторов. 

107) Посмотреть в маленькие методички с сайта МФТИ и материалы с групп вконтакте, студенты РТ и тупой скат и прочее поискать

116) подумать над знаком "выполняется", может быть стоит ввести собственную ЗНАМЕНИТУЮ СТРЕЛОЧКУ.

118) что-то не так с "--~ в rindex и самом предметном указателе

120) Определение функции во втором параграфе \todo{игрек из МНОЖЕСТВА ИГРЕК}

Второе определение про \todo{про состоит из одного элемента}
\begin{defnn}
\end{defnn}

122) убрать замечания, которые вовсе не замечания (давай считать замечанием то что относится к предыдущему билету или совсем не относится к теме)

123) придерживаться единой системы речи, формулировок теорем, оформлять их единообразно. Как с точки зрения построения предложений, так и с точки зрения обозначений типо функция $f$ или $f(x)$ но это сложнло очень

124) подумать заменить везде 1,2,3,...n на underline 1,n ради однообразия

126) давай везде использовать ldots между бинарными знаками, напиши об этом в формулы туду. То есть все-таки надо привыкать везде использовать ldots

127) Тире в индексе что-то портит(

128) разобраться со стрелкой следует (например она есть в аксиомах, а далее ее нигде нет)

129) стрелка Longleftrightarrow везде лучше использовать

134) исправить двухстрочные нахвания в хедерах

135) что делать, когда два высказвания в одной строчке инлайн   \textbf{это можно записать следующим образом: $X \subset Y \Longleftrightarrow \forall y\in Y\quad  y\in X$.}

137) символы докво вправо влево (необходиоме достаточкное) надо что-то придумать

138) Для сокращения записи используются следующие обозначения:
\begin{itemize}
\item[]
\begin{itemize}[noitemsep, label = ---]
\item \makebox[0pt][r]{$\forall$\hspace{0.75cm}}
квантор всеобщности <<для любого>>, <<для каждого>>, <<для всех>>;
\item \makebox[0pt][r]{$\exists$\hspace{0.75cm}}
квантор существования  <<существует>>, <<найдется>>;
\item \makebox[0pt][r]{$:$\hspace{0.75cm}}
логическая связка <<такой, что>>, <<такие, что>>;
\item \makebox[0pt][r]{$\triangleq$\hspace{0.75cm}} 
<<по обозначению равно>>;
\item \makebox[0pt][r]{$\to$\hspace{0.75cm}}
<<соответствует>>, <<поставлено в соответствие>> или <<стремится>>, <<при стремлении>>;
\item \makebox[0pt][r]{$\Rightarrow$\hspace{0.75cm}}
логическая связка <<следует>>;
\item \makebox[0pt][r]{$\Longleftrightarrow$\hspace{0.75cm}}
логическая связка <<равносильно>>, <<тогда и только тогда>>.
\end{itemize}

переделать. Точно удалить последние два пункта? ведь в книге куча других обозначений, может вообще удалить это. Последние два вообще лучше из книги вычленить. или нет. хрен его знает.

140) питоновскими скриптами проверить всякие конструкции?)

ПЕРВАЯ ГЛАВА
141) определение 4 первой главы про порядки. Переписать из-за стрелочек

142) ТЕОРЕМА 4 Критерий Коши переписать из-за стрелочек и + завести что-то специальное для теорем где есть док-во необходимости и достаточности

Остальное все нормально в первой главе. И вообще, вроде памятка нормальная теперь. Может немного перебор.

143) оформить таблицами в формулах все то, что будет круто оформить таблицей.


-------------------------------------

144) NEW

------------------------------------

20. ***Прямые и плоскости в пространстве.*** Формулы расстояния от точки до прямой и плоскости, между прямыми в пространстве. Углы между прямыми и плоскостями.

22. Линейное отображение конечномерных линейных пространств, его матрица. *Сюръективное и инъективное отображения. Ядро и образ линейного отображения.*

NEW 23. Собственные значения и собственные векторы линейных преобразований. *Диагонализируемость линейных преобразований.*

!!!30.->31. Полная система событий. Формула полной вероятности. Формула Байеса. *Независимость событий и классов событий.*

!!!31->32. Математическое ожидание и дисперсия случайной величины, их свойства. *Вычисление для нормального распределения*

32.->33. Неравенство Чебышева и закон больших чисел.

34. NEW one

Центральная предельная теорема для независимых одинаково распределенных случайных величин с конечной дисперсией. 


-------------------------------------

NEW

------------------------------------


145) выписать распределения вероятностные в формулы.


146) с.90 - здесь в первый раз напимсано, чему равно M_f, хотя это не первое употребление этой записи. Я бы перенёс это определение M_f на с.89 перед первым употреблением. 

у меня вообще много щаметок в разных книжках и лекциях про то, как лучше оформить те билеты. надо бы переписать.

148 Дописать 12 билет (в Иванове хорошо написано) и 34 (ЦПТ)

152 Неплохо бы в конце добавить сводку типовых контрпримеров, многие препы любят их спрашивать.

153 Дописать ЦПТ, Равномерную сходимость Фурье, Равномерную сходимость

154 Крутая вещь, не помешали бы разные версии от разных лекторов (в частности лемма гурса для потока половинкина), ну и теорвер не особо полный

155 
ССО только в действительных числах, что плохо и не всем преподавателям нравится.

156 Доказательство интегральной теоремы коши через формулу Грина ошибочна(проблемы на границе). Через лемму Гурса збс.

158 решить что делать со знаками препинания в конце формул не инлайн.

159 выстроить философию построению билетов. 

160 БИЛЕТ 35 ПОСЛЕДНЯЯ ТЕОРЕМА \footnote{Хочу обратить ваше внимание, что мы везде писали $\Gamma$ (следуя обозначениям Е.С.~Половинкина), однако подразумевали, что берем положительно ориентированную границу $G$ относительно области $G$. Т.е. нагляднее было бы писать $(\partial G)^{+}$. Просто не забывайте, что в Теоремах 4 и 5  нужна положительно ориентированная граница. }
