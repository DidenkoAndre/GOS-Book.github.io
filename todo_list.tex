\chapter{TODOLIST}

0) посмотри 157 пункт!!!

1) дописать все билеты полностью. особенно третья часть про кратные интегралы плохо написаны.

4)картиночки везде добавить и поработать над обтеканием их.
 
8)рисунок райгородского в формуле полной вероятности?
 
9) надо оформить все функции вставляющие картинки как \usepict с автоматическим label и pictures/
 
10) надо сделать рисунок катета и гипотенузы в УКР билета 33
(Как Карлов пояснил вообще написать)
 
12) картинка ,33.4 там Г хотя я в книге использую обозначение партиал Ж для края

13) относительно 12 подумай какое лучше обозначение для края будет с плюсиком или без плюсика. 

14) (\textbf{пока непонятно что делать, но это яковлев 90})

16) написать о геометрической вероятности в 30 билете.

17) в определение зависимости событий

Здесь было бы полезно сказать, что события не могут быть независимыми или зависимыми сами по себе. Свойство независимости событий зависит от введенной вероятностной меры. Если на сигма-алгебре ввести одну вероятностную меру, то события могут оказаться независимыми, а если другую вероятностную меру, то могут оказаться зависимыми.

И уже тем более, никакого отношения к причинно-следственным связям стохастическая независимость не имеет, об этом тоже можно сказать.

18) Обязательно нужна теорема ЗБЧ по Хинчину (которая доказывается с помощью хар. функций) и УЗБЧ Колмогорова (если эти теоремы были в курсе, естественно). И в этом случае надо быть готовым сказать, что такое хар. функция и ее простейшие свойства.

А добавил бы я эти теоремы хотя бы потому, что на практике ими чаще  пользуются. В матстатистике так это вообще рабочие инструменты.

%!!!!!!!!!!!!!!!!!!!!!!!!!!!!!!!!!!!!!!!!!!!!!!!!!!!!!!!!!!!!!!!!!!!!!!!!!!!!!!!!!!!!!!!!!!!!!!!!!!!!!
20) нормальные ссылки \label \ref расставить. (а не label{ksdfnksfjsdf} всякие там, о чем я думал)

35) определение 4 стр 15 половинкин.и подобная ересь.
опред 5 стр 16. Вводить ли снова такие основы основ. 

Или сослаться на материал предыдущих билетов? (очень плохо.)

36) вопрос Лехи Малышева про R и [a,b] и Павла Останина че-то там про дифференцируемость.

42) исправить максимальное количество всевозможных warning'и. и вообще overful и underfull некрасиво смотрятся.

44) (ведь тогда $\exists\, \delta$ такая, что $\dneio{\delta}{x_0}\cap D_f=\emptyset$). это из первого билета.

подумать, как это можно более верно сказать. Противоречие с кольцом. что если всегда для любого эпсилон можно найти кольцо вокруг точки х_0 что свойство предела будет выполнено.

45) переход к пределу в неравенствах используешь в билете 2, но даже не написал формулировки.

46) замечание про точки разрыва с 93 страницы. В связи с чем перегруппировать содержание 2 и 3 билета.

47) рисунок в теорему о промежуточных значениях. собственно ручный.

50) сделать в самом конце различные мини-версии для мини-печати и сделать ссылки на них.

51) формулы Производная сложной функции, параметрически заданной, параграф 13 кудрявцева 1

кривые кудрявцев 511 страница. 255 в пдфке

58) 33 билет разобраться со сноской

64) сделать разделение на материал полезный и именно к билету относящийся

105) переписать все формулы в стиле с памяткой. иногда инконсинсетнинти. пробелы у кванторов. 

106) Обратиться за помощью к сообществу Botay? 

107) Посмотреть в маленькие методички с сайта МФТИ и материалы с групп вконтакте, студенты РТ и тупой скат и прочее поискать

109) Обозначение для индикатора лучше использовать $I()$ а не $\xi$

112) посмотреть как лучше всего селать всякие дроби типо 1/a \frac{1}{a} в одной строчке. Ну маленькая же, нет? напиример 3-ья страница.

ГОЛОСОВАНИЕ: 
Яковлев за $\dfrac{1}{a}$
Иванов за $\frac{1}{a}$

115) сделать %%%%%%%%%%%%%%%%%%%%%%%%%%%%%%%%%%%%%%%%%
% The Legrand Orange Book
% Structural Definitions File
% Version 2.0 (9/2/15)
%
% Original author:
% Mathias Legrand (legrand.mathias@gmail.com) with modifications by:
% Vel (vel@latextemplates.com)
% 
% This file has been downloaded from:
% http://www.LaTeXTemplates.com
%
% License:
% CC BY-NC-SA 3.0 (http://creativecommons.org/licenses/by-nc-sa/3.0/)
%
%%%%%%%%%%%%%%%%%%%%%%%%%%%%%%%%%%%%%%%%%

%----------------------------------------------------------------------------------------
%	VARIOUS REQUIRED PACKAGES AND CONFIGURATIONS
%----------------------------------------------------------------------------------------

\usepackage[margin=1.6cm, top=1.2cm,right=1.9cm,left=1.9cm, footskip = 1 cm, headheight=20pt,headsep=0.2cm]{geometry}

\usepackage{graphicx} % Required for including pictures
%\graphicspath{{Pictures/}} % Specifies the directory where pictures are stored
\usepackage[english, russian]{babel}
\usepackage{lipsum} % Inserts dummy text

\usepackage{tikz} % Required for drawing custom shapes

\usepackage{enumitem} % Customize lists
%\setlist{nolistsep} % Reduce spacing between bullet points and numbered lists

\usepackage{booktabs} % Required for nicer horizontal rules in tables

\usepackage{xcolor} % Required for specifying colors by name
\definecolor{ocre}{RGB}{150, 120, 182} % Define the orange color used for highlighting throughout the book

%----------------------------------------------------------------------------------------
%	FONTS
%----------------------------------------------------------------------------------------

%\usepackage{avant} % Use the Avantgarde font for headings
%\usepackage{times} % Use the Times font for headings
%\usepackage{mathptmx} % Use the Adobe Times Roman as the default text font together with math symbols from the Sym­bol, Chancery and Com­puter Modern fonts

\usepackage{microtype} % Slightly tweak font spacing for aesthetics
\usepackage{cmap}
\usepackage[utf8]{inputenc} % Required for including letters with accents
\usepackage[T2A]{fontenc} % Use 8-bit encoding that has 256 glyphs

\newcommand{\cyrfamily}[5]{%
  \DeclareFontShape{#1}{#2}{#3}{#4}{
    <-6>    #50500
    <6-7>   #50600
    <7-8>   #50700
    <8-9>   #50800
    <9-10>  #50900
    <10-12> #51000
    <12-17> #51200
    <17-20> #51728
    <20-23> #52074
    <23-28> #52488
    <28-34> #52986
    <35->   #53583
  }{}%
}
\DeclareFontFamily{T2A}{cmss}{}
\cyrfamily{T2A}{cmss}{m}{n}{lass}
\cyrfamily{T2A}{cmss}{m}{sl}{lasi}
\cyrfamily{T2A}{cmss}{m}{it}{lasi}
\cyrfamily{T2A}{cmss}{bx}{n}{lasx}
\cyrfamily{T2A}{cmss}{bx}{it}{laso}
\cyrfamily{T2A}{cmss}{bx}{sl}{laso}
\DeclareFontShape{T2A}{cmss}{m}{sc}{<->sub*cmr/m/sc}{}
\DeclareFontShape{T2A}{cmss}{sbc}{n}{<->lassdc10}{}

\iffalse
\DeclareFontFamily{T2A}{pag}{}
\cyrfamily{T2A}{pag}{m}{n}{lass}
\cyrfamily{T2A}{pag}{m}{sl}{lasi}
\cyrfamily{T2A}{pag}{m}{it}{lasi}
\cyrfamily{T2A}{pag}{bx}{n}{lasx}
\cyrfamily{T2A}{pag}{bx}{it}{laso}
\cyrfamily{T2A}{pag}{bx}{sl}{laso}
\DeclareFontShape{T2A}{pag}{m}{sc}{<->sub*cmr/m/sc}{}
\DeclareFontShape{T2A}{pag}{sbc}{n}{<->lassdc10}{}
\fi

%----------------------------------------------------------------------------------------
%	BIBLIOGRAPHY AND INDEX
%----------------------------------------------------------------------------------------

\usepackage[style=alphabetic,citestyle=numeric,sorting=nyt,sortcites=true,autopunct=true,babel=hyphen,hyperref=true,abbreviate=false,backref=true,backend=biber]{biblatex}
\addbibresource{bibliography.bib} % BibTeX bibliography file
\defbibheading{bibempty}{}

\usepackage{calc} % For simpler calculation - used for spacing the index letter headings correctly
\usepackage{calc} 
\usepackage[xindy]{imakeidx}
\indexsetup{level=\chapter}
\makeindex[program=texindy, options=-M mystyle.xdy -L russian -C utf8]

\makeatletter
\newcommand{\rindex}[2][\imki@jobname]{%
	\index[#1]{\detokenize{#2}}%
}
\makeatother

\usepackage{filecontents}
\begin{filecontents*}{mystyle.xdy}
	;;; xindy style file
	(markup-locclass-list :open "\dotfill" :sep "")	
	
	(define-letter-groups
	("a" "b" "c" "d" "e" "f" "g" "h" "i" "j" "k" "l" "m"
	"n" "o" "p" "q" "r" "s" "t" "u" "v" "w" "x" "y" "z"))
	
	(require
	"rules/latin-tolower.xdy")
	
	(use-rule-set
	:run 0
	:rule-set ("latin-tolower"))
	
	(markup-letter-group
	:open-head "\nopagebreak\tikz\node at (0pt,0pt) [draw=none,fill=ocre!50,line width=1pt,inner sep=5pt]{\parbox{\linewidth-2\fboxsep-2\fboxrule-2pt}{\centering\large\sffamily\bfseries\textcolor{white}{ "
				:close-head "}}};\vspace*{0.2cm}\nopagebreak"
	:capitalize)
\end{filecontents*}

%----------------------------------------------------------------------------------------
%	MAIN TABLE OF CONTENTS
%----------------------------------------------------------------------------------------

\usepackage{titletoc} % Required for manipulating the table of contents

\contentsmargin{0cm} % Removes the default margin

%Part text styling
\titlecontents{part}[0cm]
{\addvspace{20pt}\centering\large\bfseries}
{}
{}
{}

% Chapter text styling
\titlecontents{chapter}[1.25cm] % Indentation
{\addvspace{12pt}\large\sffamily\bfseries} % Spacing and font options for chapters
{\color{ocre!60}\contentslabel[\Large\thecontentslabel]{1.25cm}\color{ocre}} % Chapter number
{\color{ocre}}  
{\color{ocre!60}\normalsize\;\titlerule*[.5pc]{.}\;\thecontentspage} % Page number

% Section text styling
\titlecontents{section}[1.25cm] % Indentation
{\addvspace{3pt}\sffamily\bfseries} % Spacing and font options for sections
{\contentslabel[\thecontentslabel]{1.25cm}} % Section number
{}
{\hfill\color{black}\thecontentspage} % Page number
[]

% Subsection text styling
\titlecontents{subsection}[1.25cm] % Indentation
{\addvspace{1pt}\sffamily\small} % Spacing and font options for subsections
{\contentslabel[\thecontentslabel]{1.25cm}} % Subsection number
{}
{\ \titlerule*[.5pc]{.}\;\thecontentspage} % Page number
[]

% List of figures
\titlecontents{figure}[0em]
{\addvspace{-5pt}\sffamily}
{\thecontentslabel\hspace*{1em}}
{}
{\ \titlerule*[.5pc]{.}\;\thecontentspage}
[]

% List of tables
\titlecontents{table}[0em]
{\addvspace{-5pt}\sffamily}
{\thecontentslabel\hspace*{1em}}
{}
{\ \titlerule*[.5pc]{.}\;\thecontentspage}
[]

%----------------------------------------------------------------------------------------
%	MINI TABLE OF CONTENTS IN PART HEADS
%----------------------------------------------------------------------------------------

% Chapter text styling
\titlecontents{lchapter}[0em] % Indenting
{\addvspace{15pt}\large\sffamily\bfseries} % Spacing and font options for chapters
{\color{ocre}\contentslabel[\Large\thecontentslabel]{1.25cm}\color{ocre}} % Chapter number
{}  
{\color{ocre}\normalsize\sffamily\bfseries\;\titlerule*[.5pc]{.}\;\thecontentspage} % Page number

% Section text styling
\titlecontents{lsection}[0em] % Indenting
{\sffamily\small} % Spacing and font options for sections
{\contentslabel[\thecontentslabel]{1.25cm}} % Section number
{}
{}

% Subsection text styling
\titlecontents{lsubsection}[.5em] % Indentation
{\normalfont\footnotesize\sffamily} % Font settings
{}
{}
{}

%----------------------------------------------------------------------------------------
%	PAGE HEADERS
%----------------------------------------------------------------------------------------

\usepackage{fancyhdr} % Required for header and footer configuration

\pagestyle{fancy}
\renewcommand{\chaptermark}[1]{\markboth{\ifnum\value{secnumdepth}>-1 \sffamily\small #1 \fi}{}} % Chapter text font settings
\renewcommand{\sectionmark}[1]{\markright{\sffamily\small \thesection\hspace{5pt}#1}{}} % Section text font settings
\fancyhf{} \fancyhead[LE,RO]{\sffamily\normalsize\thepage} % Font setting for the page number in the header
\fancyhead[LO]{\nouppercase{\small \rightmark}} % Print the nearest section name on the left side of odd pages
\fancyhead[RE]{\nouppercase{\small \leftmark}} % Print the current chapter name on the right side of even pages
\renewcommand{\headrulewidth}{0.5pt} % Width of the rule under the header
\addtolength{\headheight}{2.5pt} % Increase the spacing around the header slightly
\renewcommand{\footrulewidth}{0pt} % Removes the rule in the footer
\fancypagestyle{plain}{\fancyhead{}\renewcommand{\headrulewidth}{0pt}} % Style for when a plain pagestyle is specified

% Removes the header from odd empty pages at the end of chapters
\makeatletter
\renewcommand{\cleardoublepage}{
\clearpage\ifodd\c@page\else
\hbox{}
\vspace*{\fill}
\thispagestyle{empty}
\newpage
\fi}

%----------------------------------------------------------------------------------------
%	THEOREM STYLES
%----------------------------------------------------------------------------------------

\usepackage{amsmath,amsfonts,amssymb,amsthm} % For math equations, theorems, symbols, etc

\newcommand{\intoo}[2]{\mathopen{]}#1\,;#2\mathclose{[}}
\newcommand{\ud}{\mathop{\mathrm{{}d}}\mathopen{}}
\newcommand{\intff}[2]{\mathopen{[}#1\,;#2\mathclose{]}}
\newtheorem{notation}{Notation}[chapter]

% Boxed/framed environments
\newtheoremstyle{ocrenumbox}% % Theorem style name
{0pt}% Space above
{0pt}% Space below
{\normalfont}% % Body font
{}% Indent amount
{\small\bf\sffamily\color{ocre}}% % Theorem head font
{.}% Punctuation after theorem head
{0.25em}% Space after theorem head
{\small\sffamily\color{ocre}\thmname{#1}\nobreakspace\thmnumber{\@ifnotempty{#1}{}\@upn{#2}}% Theorem text (e.g. Theorem 2.1)
\thmnote{\nobreakspace\the\thm@notefont\sffamily\bfseries\color{black}\nobreakspace(#3)}} % Optional theorem note
\renewcommand{\qedsymbol}{$\blacksquare$}% Optional qed square

\newtheoremstyle{blacknumex}% Theorem style name
{5pt}% Space above
{5pt}% Space below
{\normalfont}% Body font
{} % Indent amount
{\small\bf\sffamily}% Theorem head font
{.}% Punctuation after theorem head
{0.25em}% Space after theorem head
{\small\sffamily{\tiny\ensuremath{\blacksquare}}\nobreakspace\thmname{#1}\nobreakspace\thmnumber{\@ifnotempty{#1}{}\@upn{#2}}% Theorem text (e.g. Theorem 2.1)
\thmnote{\nobreakspace\the\thm@notefont\sffamily\bfseries(#3)}}% Optional theorem note

\newtheoremstyle{blacknumbox} % Theorem style name
{0pt}% Space above
{0pt}% Space below
{\normalfont}% Body font
{}% Indent amount
{\small\bf\sffamily}% Theorem head font
{.}% Punctuation after theorem head
{0.25em}% Space after theorem head
{\small\sffamily\thmname{#1}\nobreakspace\thmnumber{\@ifnotempty{#1}{}\@upn{#2}}% Theorem text (e.g. Theorem 2.1)
\thmnote{\nobreakspace\the\thm@notefont\sffamily\bfseries(#3)}}% Optional theorem note

% Non-boxed/non-framed environments
\newtheoremstyle{ocrenum}% % Theorem style name
{5pt}% Space above
{5pt}% Space below
{\normalfont}% % Body font
{}% Indent amount
{\small\bf\sffamily\color{ocre}}% % Theorem head font
{.}% Punctuation after theorem head
{0.25em}% Space after theorem head
{\small\sffamily\color{ocre}\thmname{#1}\nobreakspace\thmnumber{\@ifnotempty{#1}{}\@upn{#2}}% Theorem text (e.g. Theorem 2.1)
\thmnote{\nobreakspace\the\thm@notefont\sffamily\bfseries(#3)}} % Optional theorem note

\newtheoremstyle{note}%
{3pt}% Space above1
{3pt}% Space below1
{}% Body font
{}% Indent amount2
{\bfseries}% Theorem head font
{.}% Punctuation after theorem head
{.5em}% Space after theorem head3
{}%
\renewcommand{\qedsymbol}{$\blacksquare$}% Optional qed square
\makeatother

%----------------------------------------------------------------------------------------
%	DEFINITION OF COLORED BOXES
%----------------------------------------------------------------------------------------

\RequirePackage[framemethod=default]{mdframed} % Required for creating the theorem, definition, exercise and corollary boxes

% Theorem box
\newmdenv[skipabove=7pt,
skipbelow=7pt,
backgroundcolor=black!5,
linecolor=ocre,
innerleftmargin=5pt,
innerrightmargin=5pt,
innertopmargin=5pt,
leftmargin=0cm,
rightmargin=0cm,
innerbottommargin=5pt]{tBox}

% Exercise box	  
\newmdenv[skipabove=7pt,
skipbelow=7pt,
rightline=false,
leftline=true,
topline=false,
bottomline=false,
backgroundcolor=ocre!10,
linecolor=ocre,
innerleftmargin=5pt,
innerrightmargin=5pt,
innertopmargin=5pt,
innerbottommargin=5pt,
leftmargin=0cm,
rightmargin=0cm,
linewidth=4pt]{eBox}	

% Definition box
\newmdenv[skipabove=7pt,
skipbelow=7pt,
rightline=false,
leftline=true,
topline=false,
bottomline=false,
linecolor=ocre,
innerleftmargin=5pt,
innerrightmargin=5pt,
innertopmargin=0pt,
leftmargin=0cm,
rightmargin=0cm,
linewidth=4pt,
innerbottommargin=0pt]{dBox}	

% Corollary box
\newmdenv[skipabove=7pt,
skipbelow=7pt,
rightline=false,
leftline=true,
topline=false,
bottomline=false,
linecolor=gray,
backgroundcolor=black!5,
innerleftmargin=5pt,
innerrightmargin=5pt,
innertopmargin=5pt,
leftmargin=0cm,
rightmargin=0cm,
linewidth=4pt,
innerbottommargin=5pt]{cBox}

% Creates an environment for each type of theorem and assigns it a theorem text style from the "Theorem Styles" section above and a colored box from above

% Defines the theorem text style for each type of theorem to one of the three styles above

\theoremstyle{blacknumbox}
\newtheorem{defnT}{Определение}[chapter]
\renewcommand{\thedefnT}{\arabic{defnT}}
\newcommand{\theHdefnT}{\thechapter.\arabic{defnT}}
\newenvironment{defn}{\begin{dBox}\begin{defnT}}{\end{defnT}\end{dBox}}	
\newcounter{defn}

\newtheorem{defnnT}{Определение} %штрихованное.
\renewcommand{\thedefnnT}{\arabic{defnT}'}
\newcommand{\theHdefnnT}{\thechapter.\arabic{defnnT}'}
\newenvironment{defnn}{\begin{dBox}\begin{defnnT}}{\end{defnnT}\end{dBox}}	
\newcounter{defnn}

\theoremstyle{ocrenumbox}
\newtheorem{thmT}{Теорема}[chapter]
\renewcommand{\thethmT}{\arabic{thmT}}
\newcommand{\theHthmT}{\thechapter.\arabic{thmT}}
\newenvironment{thm}{\begin{tBox}\begin{thmT}}{\end{thmT}\end{tBox}}
\newcounter{thm}

\newtheorem{thmnT}{Теорема}[chapter] %штрихованное.
\renewcommand{\thethmnT}{\arabic{thmT}'}
\newcommand{\theHthmnT}{\thechapter.arabic{thmnT}'}
\newenvironment{thmn}{\begin{tBox}\begin{thmnT}}{\end{thmnT}\end{tBox}}
\newcounter{thmn}

\newtheorem{lemmT}{Лемма}[chapter]
\renewcommand{\thelemmT}{\arabic{lemmT}}
\newcommand{\theHlemmT}{\thechapter.\arabic{lemmT}}
\newenvironment{lemm}{\begin{tBox}\begin{lemmT}}{\end{lemmT}\end{tBox}}
\newcounter{lemm}

\newtheorem{lemmnT}{Лемма}[chapter] %штрихованное.
\renewcommand{\thelemmnT}{\arabic{lemmT}'}
\newcommand{\theHlemmnT}{\thechapter.\arabic{lemmnT}'}
\newenvironment{lemmn}{\begin{tBox}\begin{lemmnT}}{\end{lemmnT}\end{tBox}}
\newcounter{lemmn}

\newtheorem{sttT}{Утверждение}[chapter]
\renewcommand{\thesttT}{\arabic{sttT}}
\newcommand{\theHstt}{\thechapter.\arabic{sttT}}
\newenvironment{stt}{\begin{tBox}\begin{sttT}}{\end{sttT}\end{tBox}}
\newcounter{stt}

\newtheorem{axiomeT}{Аксиома}[chapter]
\renewcommand{\theaxiomeT}{\arabic{axiomeT}}
\newcommand{\theHaxiomeT}{\thechapter.\arabic{axiomeT}}
\newenvironment{axiome}{\begin{tBox}\begin{axiomeT}}{\end{axiomeT}\end{tBox}}
\newcounter{axiome}

\newtheorem{axiomenT}{Аксиома}[chapter] %штрихованное.
\renewcommand{\theaxiomenT}{\arabic{axiomeT}'}
\newcommand{\theHaxiomenT}{\thechapter.\arabic{axiomenT}'}
\newenvironment{axiomen}{\begin{tBox}\begin{axiomenT}}{\end{axiomenT}\end{tBox}}
\newcounter{axiomen}

\theoremstyle{blacknumbox}
\newtheorem{consT}{Следствие}[chapter]
\renewcommand{\theconsT}{\arabic{consT}}
\newcommand{\theHconsT}{\thechapter.\arabic{consT}}
\newenvironment{cons}{\begin{cBox}\begin{consT}}{\end{consT}\end{cBox}}	
\newcounter{cons}

\newtheorem{consnT}{Cледствие}[chapter] %штрихованное.
\renewcommand{\theconsnT}{\arabic{consT}'}
\newcommand{\theHconsnT}{\thechapter.\arabic{consnT}'}
\newenvironment{consn}{\begin{cBox}\begin{consnT}}{\end{consnT}\end{cBox}}	
\newcounter{consn}

\theoremstyle{blacknumex}  
\newtheorem{exmplT}{Пример}[chapter]
\renewcommand{\theexmplT}{\arabic{exmplT}}
\newcommand{\theHexmplT}{\thechapter.\arabic{exmplT}}
\newenvironment{exmpl}{\begin{exmplT}}{\hfill{\tiny\ensuremath{\blacksquare}}\end{exmplT}}
\newcounter{exmpl}

\theoremstyle{ocrenumbox}
\newtheorem{exercT}{Упражнение}[chapter]
\renewcommand{\theexercT}{\arabic{exercT}}
\newcommand{\theHexercT}{\thechapter.\arabic{exercT}}
\newenvironment{exerc}{\begin{eBox}\begin{exercT}}{\hfill{\color{ocre}\tiny\ensuremath{\blacksquare}}\end{exercT}\end{eBox}}	
\newcounter{exerc}


\newtheorem*{notionT}{Замечание}
\newenvironment{notion}{\begin{eBox}\begin{notionT}}{\hfill{\color{ocre}\tiny\ensuremath{\blacksquare}}\end{notionT}\end{eBox}}	

\numberwithin{equation}{chapter}
\renewcommand{\theequation}{\arabic{equation}}
\newcommand{\theHequation}{\thechapter.\arabic{equation}}

\newenvironment{solution}
  {\begin{proof}[Решение.]}
  {\end{proof}}

%\renewcommand\qedsymbol{$\triangle$}

%----------------------------------------------------------------------------------------
%	REMARK ENVIRONMENT
%----------------------------------------------------------------------------------------

\newenvironment{remark}{\par\vspace{10pt}\small % Vertical white space above the remark and smaller font size
\begin{list}{}{
\leftmargin=35pt % Indentation on the left
\rightmargin=25pt}\item\ignorespaces % Indentation on the right
\makebox[-2.5pt]{\begin{tikzpicture}[overlay]
\node[draw=ocre!60,line width=1pt,circle,fill=ocre!25,font=\sffamily\bfseries,inner sep=2pt,outer sep=0pt] at (-15pt,0pt){\textcolor{ocre}{R}};\end{tikzpicture}} % Orange R in a circle
\advance\baselineskip -1pt}{\end{list}\vskip5pt} % Tighter line spacing and white space after remark

%----------------------------------------------------------------------------------------
%	SECTION NUMBERING IN THE MARGIN
%----------------------------------------------------------------------------------------

\makeatletter
\renewcommand{\@seccntformat}[1]{\llap{\textcolor{ocre}{\csname the#1\endcsname}\hspace{1em}}}                    
\renewcommand{\section}{\@startsection{section}{1}{\z@}
{-4ex \@plus -1ex \@minus -.4ex}
{1ex \@plus.2ex }
{\normalfont\large\sffamily\bfseries}}
\renewcommand{\subsection}{\@startsection {subsection}{2}{\z@}
{-3ex \@plus -0.1ex \@minus -.4ex}
{0.5ex \@plus.2ex }
{\normalfont\sffamily\bfseries}}
\renewcommand{\subsubsection}{\@startsection {subsubsection}{3}{\z@}
{-2ex \@plus -0.1ex \@minus -.2ex}
{.2ex \@plus.2ex }
{\normalfont\small\sffamily\bfseries}}                        
\renewcommand\paragraph{\@startsection{paragraph}{4}{\z@}
{-2ex \@plus-.2ex \@minus .2ex}
{.1ex}
{\normalfont\small\sffamily\bfseries}}

%----------------------------------------------------------------------------------------
%	PART HEADINGS
%----------------------------------------------------------------------------------------

% numbered part in the table of contents
\newcommand{\@mypartnumtocformat}[2]{%
\setlength\fboxsep{0pt}%
\noindent\colorbox{ocre!20}{\strut\parbox[c][.7cm]{\ecart}{\color{ocre!70}\Large\sffamily\bfseries\centering#1}}\hskip\esp\colorbox{ocre!40}{\strut\parbox[c][.7cm]{\linewidth-\ecart-\esp}{\Large\sffamily\centering#2}}}%
%%%%%%%%%%%%%%%%%%%%%%%%%%%%%%%%%%
% unnumbered part in the table of contents
\newcommand{\@myparttocformat}[1]{%
\setlength\fboxsep{0pt}%
\noindent\colorbox{ocre!40}{\strut\parbox[c][.7cm]{\linewidth}{\Large\sffamily\centering#1}}}%
%%%%%%%%%%%%%%%%%%%%%%%%%%%%%%%%%%
\newlength\esp
\setlength\esp{4pt}
\newlength\ecart
\setlength\ecart{1.2cm-\esp}
\newcommand{\thepartimage}{}%
\newcommand{\partimage}[1]{\renewcommand{\thepartimage}{#1}}%
\def\@part[#1]#2{%
\ifnum \c@secnumdepth >-2\relax%
\refstepcounter{part}%
\addcontentsline{toc}{part}{\texorpdfstring{\protect\@mypartnumtocformat{\thepart}{#1}}{\partname~\thepart\ ---\ #1}}
\else%
\addcontentsline{toc}{part}{\texorpdfstring{\protect\@myparttocformat{#1}}{#1}}%
\fi%
\startcontents%
\markboth{}{}%
{\thispagestyle{empty}%
\begin{tikzpicture}[remember picture,overlay]%
\node at (current page.north west){\begin{tikzpicture}[remember picture,overlay]%	
\fill[ocre!20](0cm,0cm) rectangle (\paperwidth,-\paperheight);
\node[anchor=north, text width=14.3cm] at (7.5cm,-3.25cm){\raggedleft\color{ocre!40}\fontsize{220}{100}\raggedleft\sffamily\bfseries\raggedleft\@Roman\c@part}; 
\node[anchor=south east] at (\paperwidth-1cm,-\paperheight+1cm){\parbox[t][][t]{8.5cm}{
\printcontents{l}{0}{\setcounter{tocdepth}{1}}%
}};
\node[anchor=north east] at (\paperwidth-0.5cm,-1cm){\parbox[t][][t]{14cm}{\strut\raggedleft\color{white}\fontsize{30}{30}\sffamily\bfseries#2}};
\end{tikzpicture}};
\end{tikzpicture}}%
\@endpart}
\def\@spart#1{%
\startcontents%
\phantomsection
{\thispagestyle{empty}%
\begin{tikzpicture}[remember picture,overlay]%
\node at (current page.north west){\begin{tikzpicture}[remember picture,overlay]%	
\fill[ocre!20](0cm,0cm) rectangle (\paperwidth,-\paperheight);
\node[anchor=north east] at (\paperwidth-1.5cm,-3.25cm){\parbox[t][][t]{15cm}{\strut\raggedleft\color{white}\fontsize{30}{30}\sffamily\bfseries#1}};
\end{tikzpicture}};
\end{tikzpicture}}
\addcontentsline{toc}{part}{\texorpdfstring{%
\setlength\fboxsep{0pt}%
\noindent\protect\colorbox{ocre!40}{\strut\protect\parbox[c][.7cm]{\linewidth}{\Large\sffamily\protect\centering #1\quad\mbox{}}}}{#1}}%
\@endpart}
\def\@endpart{\vfil\newpage
\if@twoside
\if@openright
\null
\thispagestyle{empty}%
\newpage
\fi
\fi
\if@tempswa
\twocolumn
\fi}

%----------------------------------------------------------------------------------------
%	CHAPTER HEADINGS
%----------------------------------------------------------------------------------------

% A switch to conditionally include a picture, implemented by  Christian Hupfer
\newif\ifusechapterimage
\usechapterimagetrue
\newcommand{\thechapterimage}{}%
\newcommand{\chapterimage}[1]{\ifusechapterimage\renewcommand{\thechapterimage}{#1}\fi}%
\def\@makechapterhead#1{%
{\parindent \z@ \raggedright \normalfont
\ifnum \c@secnumdepth >\m@ne
\if@mainmatter
\begin{tikzpicture}[remember picture,overlay]
\node at (current page.north west)
{\begin{tikzpicture}[remember picture,overlay]
\node[anchor=north west,inner sep=0pt] at (0,0) {\ifusechapterimage\includegraphics[width=\paperwidth]{\thechapterimage}\fi};
\draw[anchor=west] (\Gm@lmargin,-6cm) node [line width=2pt,rounded corners=15pt,draw=ocre,fill=white,fill opacity=0.5,inner sep=35pt]{\strut\makebox[22cm]{}};
\draw[anchor=west] (\Gm@lmargin+.3cm,-6cm) node[text width=12.5cm] {\huge\sffamily\bfseries\color{black}\thechapter. #1\strut};
\end{tikzpicture}};
\end{tikzpicture}
\else
\begin{tikzpicture}[remember picture,overlay]
\node at (current page.north west)
{\begin{tikzpicture}[remember picture,overlay]
\node[anchor=north west,inner sep=0pt] at (0,0) {\ifusechapterimage\includegraphics[width=\paperwidth]{\thechapterimage}\fi};
\draw[anchor=west] (\Gm@lmargin,-6cm) node [line width=2pt,rounded corners=15pt,draw=ocre,fill=white,fill opacity=0.5,inner sep=35pt]{\strut\makebox[22cm]{}};
\draw[anchor=west] (\Gm@lmargin+.3cm,-6cm) node[text width=12.5cm] {\huge\sffamily\bfseries\color{black}#1\strut};
\end{tikzpicture}};
\end{tikzpicture}
\fi\fi\par\vspace*{200\p@}}}

%-------------------------------------------

\def\@makeschapterhead#1{%
\begin{tikzpicture}[remember picture,overlay]
\node at (current page.north west)
{\begin{tikzpicture}[remember picture,overlay]
\node[anchor=north west,inner sep=0pt] at (0,0) {\ifusechapterimage\includegraphics[width=\paperwidth]{\thechapterimage}\fi};
\draw[anchor=west] (\Gm@lmargin,-6cm) node [line width=2pt,rounded corners=15pt,draw=ocre,fill=white,fill opacity=0.5,inner sep=35pt]{\strut\makebox[22cm]{}};
\draw[anchor=west] (\Gm@lmargin+.3cm,-6cm) node[text width=12.5cm] {\huge\sffamily\bfseries\color{black}#1\strut};
\end{tikzpicture}};
\end{tikzpicture}
\par\vspace*{200\p@}}
\makeatother

\iffalse
\usepackage{hyperref}
\hypersetup{hidelinks,backref=true,pagebackref=true,hyperindex=true,colorlinks=false,breaklinks=true,urlcolor= ocre,bookmarks=true,bookmarksopen=false,pdftitle={Title},pdfauthor={Author}}
\usepackage{bookmark}
\bookmarksetup{
open,
numbered,
addtohook={%
\ifnum\bookmarkget{level}=0 % chapter
\bookmarksetup{bold}%
\fi
\ifnum\bookmarkget{level}=-1 % part
\bookmarksetup{color=ocre,bold}%
\fi
}
}
\fi
. Проблема в картинке главы конкретно перед содержанием

116) подумать над знаком "выполняется", может быть стоит ввести собственную ЗНАМЕНИТУЮ СТРЕЛОЧКУ.

118) что-то не так с "--~ в rindex и самом предметном указателе

120) Определение функции во втором параграфе \todo{игрек из МНОЖЕСТВА ИГРЕК}

Второе определение про \todo{про состоит из одного элемента}
\begin{defnn}
\end{defnn}

120) где-то но не в книге написать про консультацию Половинкина

121) заменить ли R с чертой везде на что-то более заметное, например R объединение +-бесконечность.

122) убрать замечания, которые вовсе не замечания (давай считать замечанием то что относится к предыдущему билету или совсем не относится к теме)

123) придерживаться единой системы речи, формулировок теорем, оформлять их единообразно. Как с точки зрения построения предложений, так и с точки зрения обозначений типо функция $f$ или $f(x)$ но это сложнло очень

124) подумать заменить везде 1,2,3,...n на underline 1,n ради однообразия

125) крайне избегать знаменталеей в тексте, и вообще придумать что с ними делать.

126) давай везде использовать ldots между бинарными знаками, напиши об этом в формулы туду. То есть все-таки надо привыкать везде использовать ldots

127) Тире в индексе что-то портит(

128) разобраться со стрелкой следует (например она есть в аксиомах, а далее ее нигде нет)

129) стрелка Longleftrightarrow везде лучше использовать

134) исправить двухстрочные нахвания в хедерах

135) что делать, когда два высказвания в одной строчке инлайн   \textbf{это можно записать следующим образом: $X \subset Y \Longleftrightarrow \forall y\in Y\quad  y\in X$.}

137) символы докво вправо влево (необходиоме достаточкное) надо что-то придумать

138) Для сокращения записи используются следующие обозначения:
\begin{itemize}
\item[]
\begin{itemize}[noitemsep, label = ---]
\item \makebox[0pt][r]{$\forall$\hspace{0.75cm}}
квантор всеобщности <<для любого>>, <<для каждого>>, <<для всех>>;
\item \makebox[0pt][r]{$\exists$\hspace{0.75cm}}
квантор существования  <<существует>>, <<найдется>>;
\item \makebox[0pt][r]{$:$\hspace{0.75cm}}
логическая связка <<такой, что>>, <<такие, что>>;
\item \makebox[0pt][r]{$\triangleq$\hspace{0.75cm}} 
<<по обозначению равно>>;
\item \makebox[0pt][r]{$\to$\hspace{0.75cm}}
<<соответствует>>, <<поставлено в соответствие>> или <<стремится>>, <<при стремлении>>;
\item \makebox[0pt][r]{$\Rightarrow$\hspace{0.75cm}}
логическая связка <<следует>>;
\item \makebox[0pt][r]{$\Longleftrightarrow$\hspace{0.75cm}}
логическая связка <<равносильно>>, <<тогда и только тогда>>.
\end{itemize}

переделать. Точно удалить последние два пункта? ведь в книге куча других обозначений, может вообще удалить это. Последние два вообще лучше из книги вычленить. или нет. хрен его знает.

140) питоновскими скриптами проверить всякие конструкции?)

ПЕРВАЯ ГЛАВА
141) определение 4 первой главы про порядки. Переписать из-за стрелочек

142) ТЕОРЕМА 4 Критерий Коши переписать из-за стрелочек и + завести что-то специальное для теорем где есть док-во необходимости и достаточности

Остальное все нормально в первой главе. И вообще, вроде памятка нормальная теперь. Может немного перебор.

143) оформить таблицами в формулах все то, что будет круто оформить таблицей.


-------------------------------------

144) NEW

------------------------------------

20. ***Прямые и плоскости в пространстве.*** Формулы расстояния от точки до прямой и плоскости, между прямыми в пространстве. Углы между прямыми и плоскостями.

22. Линейное отображение конечномерных линейных пространств, его матрица. *Сюръективное и инъективное отображения. Ядро и образ линейного отображения.*

NEW 23. Собственные значения и собственные векторы линейных преобразований. *Диагонализируемость линейных преобразований.*

!!!30.->31. Полная система событий. Формула полной вероятности. Формула Байеса. *Независимость событий и классов событий.*

!!!31->32. Математическое ожидание и дисперсия случайной величины, их свойства. *Вычисление для нормального распределения*

32.->33. Неравенство Чебышева и закон больших чисел.

34. NEW one

Центральная предельная теорема для независимых одинаково распределенных случайных величин с конечной дисперсией. 


-------------------------------------

NEW

------------------------------------


145) выписать распределения вероятностные в формулы.


146) с.90 - здесь в первый раз напимсано, чему равно M_f, хотя это не первое употребление этой записи. Я бы перенёс это определение M_f на с.89 перед первым употреблением. 

у меня вообще много щаметок в разных книжках и лекциях про то, как лучше оформить те билеты. надо бы переписать.

148 Дописать 12 билет (в Иванове хорошо написано) и 34 (ЦПТ)

150 На втором билете зафачили соседа, так как в нем неверно сформулировано. Фурье анализ плохо заходил, в Тере лучше

151 Выпилить к херам определения из вики, исправить ряд косяков в билетах, позже отпишу замеченные, отрефакторить структуру билетов по Д/У, ЛА, перевести госбук в удобоваримый для шпаргалки вид - задуматься о перекомпоновке содержимого страницы

152 Неплохо бы в конце добавить сводку типовых контрпримеров, многие препы любят их спрашивать.

153 Дописать ЦПТ, Равномерную сходимость Фурье, Равномерную сходимость

154 Крутая вещь, не помешали бы разные версии от разных лекторов (в частности лемма гурса для потока половинкина), ну и теорвер не особо полный
плюс очень бы помог "дебильник"

155 
ССО только в действительных числах, что плохо и не всем преподавателям нравится.

156 Доказательство интегральной теоремы коши через формулу Грина ошибочна(проблемы на границе). Через лемму Гурса збс.

157 посмотреть книги из последних задавальников. Например, Чехлов -- хорошая книга. Надо скачать.

----------------------------------------------

ТЕКУЩЕЕ СОСТОЯНИЕ

формулы. 5ий параграф Кудрявцева

Билеты. Начало второй волны. 

158 решить что делать со знаками препинания в конце формул не инлайн.
