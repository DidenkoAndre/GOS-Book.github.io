\chapter[О ГОСе]{О ГОСе\footnotemark}
\footnotetext{Заметим, что этот рассказ основан на реальных событиях, произошедших в 2016 году. Возможно, уже что-то поменялось, и информация здесь уже не актуальна (а может, и сама книга уже не актуальна).}
В первую очередь, нужно рассказать о самой процедуре проведения выпускного квалификационного государственного экзамена по математике в МФТИ глазами студента третьего курса. 

Прежде всего, процедура проведения этого экзамена почти ничем не отличается от обычных экзаменов кафедры высшей математики. 
И самое главное, успех на этом экзамене зависит, прежде всего, от вашей удачи, затем от уровня подготовки и, напоследок, от того, насколько вы умеете <<вертеться>>, впрочем, как и на протяжении почти всей жизни на Физтехе.

Сначала вас ждет письменная работа, в которой будет много задач, порядка двадцати. На нее отводится довольно малое количество времени (в моем году давали три астрономических часа, т.е. 180 минут). На ней будет присутствовать 2--3 преподавателя и смотреть за вами, и никто не знает, будут ли они смотреть, чтобы никто не списывал, или следить за глобальной тишиной. У них, к тому же, будет ручной металлоискатель, которым они по желанию могут воспользоваться (он, кстати, неприятно пищит). Конечно же, рекомендую тщательно подготовиться к письменному экзамену, прорешать максимальное количество вариантов, посмотреть консультации преподавателей по решению задач и отнестись к ним лишь рекомендательно, потому что каждый год кафедра высшей математики преподносит некоторые сюрпризы, например, появлялись задачи, о которых на консультациях некоторые преподаватели говорили: <<Не будет>>, <<Маловероятно>>, <<Это слишком сложно для ГОСа>>. По желанию сходите на очные консультации преподавателей. Получить достойные баллы при должной подготовке вполне реально. Разбалловку рассчитывают, основываясь на результатах всех студентов (подгоняя под распределение Гаусса, как рассказывал Карлов М.И.), в наш год было так: что-то около 40 баллов из 68 для получения оценки <<Отлично, 10>> за письменный экзамен.

Влияние этой письменной работы, как и на всех экзаменах вышмата, зависит от преподавателей, к которым вы попадете на устном экзамене. Некоторые считают эту оценку барьером, выше которого нельзя ставить оценку в зачетку, некоторые считают среднее арифметическое по всем оценкам от кафедры высшей математики и как-то к ним прибавляют оценку за письменный экзамен, некоторые не обращают внимания вовсе, некоторые просто для себя оценивают студентов и принимают оценку за примерный уровень подготовки и спрашивают, основываясь на этом. Все, как всегда, не определено заранее.

Плавно перейдем к устному экзамену, который проходит через несколько дней после письменного. Многие после ГОСа по физике удивляются, что устный экзамен по математике проходит в больших аудиториях, таких как Актовый Зал, Большая Физическая, 117 ГК и прочие, но на самом деле, это правда. Это самый обычный экзамен от кафедры вышмата с некоторыми особенностями, о которых ниже. Снова, как и все три года, если у вас есть хоть какие-то трудности с математикой, вам надо надеяться на удачу, на то, что придут хорошие преподаватели, которые мягко принимают экзамен. И конечно же, этот экзамен "--- это один из самых больших по объему материалов для подготовки, поэтому усердно работайте, постарайтесь хорошо подготовиться, надеюсь, моя книжка вам поможет.

Преподаватели собираются в 9 часов в аудитории и мило общаются между собой. На разных факультетах, как я понимаю, процедура ГОСа немножко разная, например, может деканат прийти и смотреть, как бы кого не отправили с пересдачей (да, пересдачи на ГОСе "--- большая, нет, огромная редкость, и даже в этом случае за вас деканат, остальные преподаватели заступятся), может прийти секретарь из деканата и заниматься бумажной работой, освободив одного преподавателя от этих дел для какого-никакого ускорения процесса. В 9 часов запускают первые группы студентов, раздают билеты и отправляют на задние парты. Там вы пишете билет отведенный час, причем первые полчаса  преподаватели делятся на комиссии (которые состоят либо из одного, либо из двух человек), получают бумажки, решают всякие бюрократические проблемы и прочее, поэтому все самую малость заняты и обращают меньше внимания на студентов в эти первые полчаса после первого захода, а для остальных заходов "--- так вообще уже будут иметь студентов для допроса на математические темы. Когда преподаватели разобрались между собой, раздают зачетки и вашу фамилию называет один из двух преподавателей, который и будет принимать у вас экзамен, и вы идете навстречу своей судьбе. Продолжаете писать билет неподалеку от  непосредственно вашей комиссии.

В очередной раз подчеркну, что процесс вашего экзамена во многом определяется преподавателями, которые вас слушают, а судьба вам случайно подкидывает их. На ГОСе нельзя проситься к преподавателям, а у нас вроде никто и не пытался. Преподаватели бывают разные. У всех разное отношение к студентам, к самому ГОСу. У некоторых преподавателей есть свои любимые темы, а у некоторых, наоборот, темы, которые он совсем не помнит. Так, десятки людей жаловались, что некоторые преподаватели плохо помнят материал из теории вероятностей, однако в билете попался вопрос оттуда, и получались весьма нелепые ситуации. Вас слушают два преподавателя (в тотальном большинстве случаев), и оба ведут себя так, как будто они просто принимают у вас самый обычный экзамен\footnote{Рекомендую ознакомиться с этим гугл-документом \href{https://docs.google.com/spreadsheets/d/10jIg9Nr5oM1-Zjo_iIlKs8uxrBKrzfuJNas_YJIIxPs/edit\#gid=0}{$https://docs.google.com/...$} и тоже заполнить его по окончанию ГОСа}. Отличие ГОСа от обычных экзаменов в том, что для каждого студента заводится так называемое личное дело, которое представляет из себя обычный листик с анкетой. Его один из этих преподавателей заполняет касательно вас. Фамилия, имя, отчество, номер билета, вопросы в билете, какие были дополнительные вопросы, как вы ответили на все, какое общее впечатление о студенте и самое главное, \textit{рекомендуемая} оценка за ГОС (которую они выбирают лично, основываясь на чем-угодно) "--- все это есть в этой анкете. 

Итак, вы садитесь к преподавателю, он вам выдает вашу контрольную работу, вы ее разбираете, разочаровываетесь или радуетесь. Далее, эти два преподавателя спрашивают ваш билет, как они умеют это делать. После они задают дополнительные вопросы, которые по формату ГОСа должны быть либо очень простыми задачами, типо посчитать собственные числа у матрицы 2x2, либо прямо формулировка какой-нибудь хорошей теоремы, например, Стокса (формально, у них даже есть список рекомендуемых вопросов). Но преподаватели бывают разные. Кстати, есть еще одна особенность ГОСа "--- ты сидишь между этими преподавателями, и тебя слушают эти 20-30 минут непрерывно. Формат не позволяет давать задачи на подумать, не позволяет спрашивать сразу несколько человек. Вы находитесь один против двоих преподавателей, и вам деваться некуда. Конечно, за редким исключением, когда комиссия из одного человека или кто-то отлучится кофе попить, или телефонный звонок прервет ваш экзамен на несколько минут. Все равно здесь действует система очереди "--- пока один не ответил до конца, другого не пускают. Хотя, может, некоторые преподаватели нарушают этот порядок. После дополнительных вопросов вас отпускают домой, перед этим вы можете спросить у преподавателя рекомендуемую оценку, или подглядеть ее в той анкете.

Как только комиссии приняли всех студентов, начинается заседание комиссий, к которому у студентов нет доступа. Там оглашаются рекомендуемые оценки, и поскольку на ГОС преимущественно приглашаются люди, которые уже работали на этом факультете, т.е. лекторы и семинаристы, то, скорей всего, там будет несколько человек, которые вас знают и помнят. На самом деле, Я не знаю, чем они там занимаются. Но в итоге, там ставят в зачетку рекомендуемую оценку, или изменённую оценку, если средний балл сильно отличается от этой оценки и (или) найдутся люди, которые будут защищать (или губить) вас.

Итак, вы возвращаетесь уже вечером в институт, к 16--17 часам. Ждете приглашения в ту же аудиторию. Далее, председатель комиссии зачитывает оценки. Все аплодируют, смеются, радуются за сдачу экзамена, кто-то огорчен своей оценкой. Но все рады окончанию сессии, учебного года. И все дружно уходят из аудитории, забирая зачетки с собой с росписями всех преподавателей кафедры высшей математики, которые были у вас на экзамене.

Так и заканчивается учебный год. Целый период жизни на физтехе. Все прощаются с кафедрой высшей математики. Все уходят на летний отдых или по делам. Но никого уже не трогает сессия. И все у всех становится хорошо. 

\mbox{}

\noindent\textit{Диденко А.А.}

\mbox{}

PS. Возможно, вас интересует вопрос, что же получил основной автор данного пособия на ГОСе? 

Буду честным, я получил оценку <<\textit{отлично (8)}\,>>. Если кратко, то меня принимал Шаньков В.В. О нем много как положительных, так и отрицательных отзывов. Однако, я ему сразу не понравился. Я ещё не успел подойти к нему, как он сразу об этом заявил. Не могу дать ответ, почему, точнее, я не знаю ответа. При моем рассказе билета он давил, говорил различные колкости в мой адрес.  В общем, если кто-то знает Шанькова, то, наверное, понимает, что переубедить его в том, что я знаю математику, было очень сложно. Я даже растерялся чутка от его своеобразных манер поведения. Он поставил мне в графу рекомендуемая оценка <<\textit{хорошо (6)}\,>>. 

\glqqКто же будет читать пособие для подготовки к ГОСу от человека, который сам получил <<\textit{хорошо (6)}\,>>?\grqq "--- подумал я и слегка расстроился тому обстоятельству, что больше читателей мне не видать. Возможно, надо было больше готовиться самому, а не писать данное пособие. Однако, о моей сдаче экзамена от моих друзей узнала моя любимая семинаристка, Яковлева Тамара Харитоновна. Два года она мучилась, чтобы научить нашу группу всей прелести матанализа! На заседании комиссий она и заступилась за меня, рассказала ему и комиссиям, что я очень старательный мальчик и что средний балл по матдисциплинам у меня около \textit{9.5}. В итоге, мне в зачетку поставили <<\textit{отлично (8)}\,>>. 

Святая женщина!

\newpage
Если вам интересны некоторые официальные правила, то в наше время они были такими

\textbf{Условия проведения государственного письменного экзамена по математике.}

\begin{enumerate}
\item Задания на государственном письменном экзамене по математике выполняются только в
стандартных по размерам (170х205 мм) и количеству листов (12, 18 листов) ученических тетрадях.
Допускается наличие двух рабочих тетрадей для черновика и чистовика, которые сдаются на
проверку по окончании письменной работы. Не допускается вырывание страниц из рабочих
тетрадей.
\item Во время выполнения письменных заданий не разрешается иметь при себе электронные
средства любого вида, а также посторонние материалы, которые могут быть использованы как
средства недобросовестного выполнения письменной работы.
Нарушение данных условий влечет недопуск или удаление студента с экзамена.
\item На титульном листе рабочей тетради студент должен сделать запись:
«С условиями проведения экзамена ознакомлен» и поставить свою подпись.
\end{enumerate}