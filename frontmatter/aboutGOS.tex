\chapter[О ГОСе]{О ГОСе\footnotemark}
\footnotetext{Заметим, что этот рассказ описывает реальные события, произошедшие в 2016 году. Возможно, уже что-то поменялось, и информация здесь уже не актуальна (а может, и сама книга уже не актуальна).}
В первую очередь, нужно рассказать о самой процедуре проведения выпускного квалификационного государственного экзамена по математике в МФТИ глазами студента третьего курса. 

Процедура проведения этого экзамена почти ничем не отличается от обычных экзаменов кафедры высшей математики. 
Лично я искренне считаю, что успех на этом экзамене зависит, прежде всего, от вашей удачи, затем от уровня подготовки и, напоследок, от того, насколько вы умеете <<вертеться>>, впрочем, как и на протяжении всей жизни на Физтехе...

Сначала нас ждала письменная работа, в которой было много задач, порядка двадцати. На нее отводилось довольно малое количество времени (три астрономических часа, т.е. 180 минут). На ней присутствовало 2--3 преподавателя, которые смотрели за нами в силу своих возможностей и желаний. К тому же, у них был ручной металлоискатель, которым в моей аудитории преподаватель неловко потыкала в студентов. К слову, пищит он неприятно! 

Конечно же, рекомендую тщательно подготовиться к письменному экзамену, прорешать множество вариантов, посмотреть консультации преподавателей по решению задач и отнестись к ним лишь рекомендательно, потому что каждый год кафедра высшей математики преподносит некоторые сюрпризы, например, появлялись задачи, о которых на консультациях некоторые преподаватели говорили: <<Не будет>>, <<Маловероятно>>, <<Это слишком сложно для ГОСа>>. Получить достойные баллы при должной подготовке вполне реально. Разбалловку рассчитывают, основываясь на результатах всех студентов (подгоняя под распределение Гаусса, как рассказывал Карлов М.И.), в наш год было так: что-то около 40 баллов из 68 для получения оценки <<\textit{отлично}~(10)>> за письменный экзамен.

Влияние этой письменной работы, как и на всех экзаменах вышмата, зависит от преподавателей, к которым вы попадете на устном экзамене. Некоторые считают эту оценку барьером, выше которого нельзя ставить оценку в зачетку, некоторые считают среднее арифметическое по всем оценкам от кафедры высшей математики и как-то к ним прибавляют оценку за письменный экзамен, некоторые не обращают внимания вовсе, некоторые принимают оценку за примерный уровень подготовки и спрашивают, основываясь на этом. Все, как всегда, не определено заранее.

Перейдем к обсуждению устного экзамена, который проходит через несколько дней после письменного. Многие после ГОСа по физике удивляются, что устный экзамен по математике проходит в больших аудиториях, таких как Актовый Зал, Большая Физическая, 117 ГК и прочие, но на самом деле, ничего удивительного. Как оказалось, это самый обычный экзамен от кафедры вышмата с некоторыми особенностями, о которых ниже. Конечно же, этот экзамен "--- это один из самых больших по объему материалов для подготовки, поэтому усердно работайте, постарайтесь хорошо подготовиться, надеюсь, моя книжка вам поможет.

Преподаватели собираются к 9 часам в аудитории и мило общаются между собой. На разных факультетах, как я понимаю, процедура ГОСа немножко разная, например, может деканат прийти и смотреть, как бы кого не отправили с пересдачей (да, пересдачи на ГОСе "--- большая, нет, огромная редкость, и даже в этом случае за вас деканат, остальные преподаватели заступятся), может прийти секретарь из деканата и заниматься бумажной работой, освободив одного преподавателя от этих дел. В 9 часов запускают первые группы студентов, раздают билеты и отправляют на задние парты. Там вы пишете билет отведенный час, причем первые полчаса  преподаватели делятся на комиссии (которые состоят либо из одного, либо из двух человек), получают бумажки, решают всякие бюрократические проблемы и прочее, поэтому все слегка заняты и обращают меньше внимания на студентов в эти первые полчаса после первого захода, а для остальных заходов "--- так вообще уже будут иметь студентов для допроса на математические темы. В какой-то момент вашу фамилию называет один из двух преподавателей, который и будет принимать у вас экзамен, и вы идете навстречу своей судьбе. Продолжаете писать билет неподалеку от непосредственно вашей комиссии.

В очередной раз подчеркну, что процесс вашего экзамена во многом определяется преподавателями, которые вас слушают, а судьба вам случайно подкидывает их. На ГОСе нельзя проситься к преподавателям, а у нас вроде никто и не пытался. Преподаватели бывают разные: у всех разное отношение к студентам, к самому ГОСу. У некоторых преподавателей есть свои любимые темы, а у некоторых, наоборот, темы, которые они совсем не помнят. Так, десятки студентов жаловались, что некоторые преподаватели плохо помнят материал из теории вероятностей, и получались весьма нелепые ситуации. Вас слушают два преподавателя (в большинстве случаев), и оба ведут себя так, как будто они просто принимают у вас самый обычный экзамен\footnote{Рекомендую ознакомиться с  этой анкетой о ГОСе на гугл-диске \href{https://docs.google.com/spreadsheets/d/10jIg9Nr5oM1-Zjo_iIlKs8uxrBKrzfuJNas_YJIIxPs/edit\#gid=0}{$docs.google.com/...$} и тоже заполнить ее по окончанию ГОСа}. Отличие ГОСа от других экзаменов состоит в том, что для каждого студента заводится так называемое личное дело, которое представляет из себя листик с анкетой. Один из этих преподавателей заполняет его касательно вас. Фамилия, имя, отчество, номер билета, вопросы в билете, какие были дополнительные вопросы, как вы ответили на все, какое общее впечатление о студенте и самое главное, \textit{рекомендуемая оценка} за ГОС (которую они выбирают лично, основываясь на чем-угодно) "--- все это есть в этой анкете. 

Итак, вы садитесь к преподавателю, он вам выдаёт вашу контрольную работу. Далее, эти два преподавателя спрашивают ваш билет, как они умеют это делать. После они задают дополнительные вопросы, которые по формату ГОСа должны быть либо очень простыми задачами, наподобие посчитать собственные числа у матрицы $2\times 2$, либо прямо формулировка какой-нибудь хорошей теоремы, например, Стокса (формально, у них даже есть список рекомендуемых вопросов). Но преподаватели бывают разные. Кстати, ещё одна особенность ГОСа "--- вы сидите между этими преподавателями, и они вас слушают непрерывно 20--30 минут. Формат не позволяет давать задачи на подумать, не позволяет спрашивать сразу несколько человек. Вы находитесь один против двоих преподавателей, и вам деваться некуда. Конечно, за редким исключением, когда комиссия состоит из одного человека, или кто-то отлучится кофе попить, или телефонный звонок преподавателю прервёт ваш экзамен на несколько минут. Все равно здесь действует система очереди "--- пока один не ответил до конца, другого не пускают. Хотя, может, некоторые преподаватели нарушают этот порядок. После дополнительных вопросов вас отпускают домой, перед этим вы можете попробовать спросить у преподавателя рекомендуемую оценку, или подглядеть ее в той анкете.

Как только комиссии послушали всех студентов, начинается заседание комиссий. Здесь оглашаются рекомендуемые оценки, и поскольку на ГОС преимущественно приглашаются люди, которые уже работали на вашем факультете, т.е. лекторы и семинаристы, то, скорей всего, там будет несколько человек, которые вас знают и помнят, они могут немного подискутировать о данной оценке. На самом деле, я не знаю, чем они там занимаются. В итоге, там ставят в зачётку рекомендуемую оценку, или изменённую оценку, если средний балл сильно отличается от этой оценки и найдутся преподаватели, которые будут защищать (или губить) вас на этом заседании.

Итак, вы возвращаетесь обратно в институт к 16--17 часам. Ждёте приглашения в ту же аудиторию. Далее, председатель комиссии зачитывает оценки. Все аплодируют, смеются, радуются за сдачу экзамена, кто-то огорчён своей оценкой. Но все рады окончанию сессии, учебного года. И все дружно уходят из аудитории, забирая зачётки с собой с росписями всех преподавателей кафедры высшей математики, которые были у вас на экзамене.

Так и заканчивается учебный год. Целый период жизни на Физтехе. Все прощаются с кафедрой высшей математики. Все уходят на летний отдых или по делам. Но никого уже не трогает сессия. И все у всех становится хорошо. 

\mbox{}

\noindent\textit{Диденко А.А.}

\mbox{}

PS. Возможно, вас интересует вопрос, что же получил основной автор данного пособия на ГОСе? 

Буду честным, я получил оценку <<\textit{отлично}~(8)>>. 

За письменную работу я получил <<\textit{отлично}~(9)>> "--- наошибался в арифметике (подчеркну, исключительно в арифметике) аж на 20 с лишним баллов из 68. Вообще, это отдельная тема для дискуссий: почему арифметические ошибки так много весят (аж треть баллов)? Такое чувство, что больше половины Физтеха имели бы полный балл за выполненные ими задачи, если бы преподаватели мягко относились к арифметике. Ожидается, что будут проверять наши знания по высшей математике, а в итоге, снимают баллы за опечатки в такой сложной и напряженной работе. Хотя с другой стороны, какой разговор о матанализе может идти с человеком, который на письменном экзамене написал что-нибудь вроде $2 + 2 = 5$... В общем, читатель, будь внимателен и максимально собран на письменной работе и не повторяй ошибок автора данной книги. Но на самом деле, девятка меня вполне устраивала, для меня нет принципиальной разницы между градациями оценки <<отлично>>.

Устный экзамен у меня принимал Шаньков В.В.. О нем я слышал много как положительных, так и отрицательных отзывов. Однако, не успел подойти к нему, как он заявил на всю аудиторию, обращаясь ко мне: <<Бездарность!>> Не могу дать ответ, почему, точнее, я и не знаю ответа (возможно, моя письменная работа его чем-то расстроила), но, безусловно, я и даже некоторые преподаватели были шокированы данной его выходкой. Свой билет я знал отлично. Однако, если читатель знает Шанькова, то, наверное, понимает, что уже переубедить его в том, что я знаю математику, было очень сложно. При рассказе билета он немного давил, говорил различные колкости в мой адрес. Я даже растерялся чутка от его весьма своеобразных манер поведения. По окончании, он поставил мне в графу рекомендуемая оценка <<\textit{хорошо}~(6)>>. 

``Кто же будет читать пособие для подготовки к ГОСу от человека, который сам получил <<\textit{хорошо}~(6)>>?'' "--- подумал я и слегка расстроился тому обстоятельству, что больше читателей мне не видать. Возможно, надо было больше готовиться самому, а не писать данное пособие, тем более местами оно не ахти. Однако, о моей сдаче экзамена от моих друзей узнала моя любимая семинаристка, Яковлева Тамара Харитоновна. Два года она мучилась, чтобы научить нашу группу всей прелести матанализа! На заседании комиссий она и заступилась за меня, рассказала ему и комиссиям, что я очень старательный мальчик и что средний балл по математическим дисциплинам у меня около~\textit{9.5}. В итоге, мне в зачётку поставили <<\textit{отлично}~(8)>>. 

Святая женщина!