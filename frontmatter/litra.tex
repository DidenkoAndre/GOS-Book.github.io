\chapter{Список литературы и используемые материалы.}

Предоставляю список литературы, которыми мы пользовались в основном для написания билетов. 

Кстати говоря, почти всю перечисленную ниже литературу, Вы сможете получить по следующей ссылке: \href{https://drive.google.com/drive/u/0/folders/0BzuzEyNkpwYDcENXcV9jNWdwVlU}{$drive.google.com/...$}.

\begin{itemize}
\item[\textit{1-6}]
\; --- \: \textit{Введение в математический анализ.} 
\begin{itemize}
\item[\textbullet]
Лекции Сакбаева В.Ж.
\item[\textbullet]
Учебное пособие Яковлева Г.Н. "Лекции по математическому анализу" (часть 1)
\item[\textbullet]
Учебное пособие Бесова О.В. "Лекции по математическому анализу".
\end{itemize}

\item[\textit{7-8}] 
\; --- \: \textit{Многомерный анализ, интегралы и ряды.}
\begin{itemize}
\item[\textbullet]
Лекции Сакбаева В.Ж.
\end{itemize}

\item[\textit{9-10}] 
\; --- \: \textit{Кратные интегралы и теория поля.}
\begin{itemize}
\item[\textbullet]
Лекции Сакбаева В.Ж.
\end{itemize}

\item[\textit{11-13}] 
\; --- \: \textit{Многомерный анализ, интегралы и ряды.}
\begin{itemize}
\item[\textbullet]
Лекции Сакбаева В.Ж.
\end{itemize}

\item[\textit{14-16}] 
\; --- \: \textit{Кратные интегралы и теория поля.}
\begin{itemize}
\item[\textbullet]
Лекции Сакбаева В.Ж.
\end{itemize}

\item[\textit{17-19}] 
\; --- \: \textit{Гармонический анализ.}
\begin{itemize}
\item[\textbullet]
Лекции Сакбаева В.Ж.
\item[\textbullet]
Учебное пособие Яковлева Г.Н. "Лекции по математическому анализу"  (часть 3).
\end{itemize}

\item[\textit{20}] 
\; --- \: \textit{Аналитическая геометрия.}
\begin{itemize}
\item[\textbullet]
Лекции Чубарова И.А. 
\item[\textbullet]
Учебное Пособие Беклемишева Д.В. "Курс аналитической геометрии и линейной алгебры".
\end{itemize}

\item[\textit{21-25}] 
\; --- \: \textit{Линейная алгебра.}
\begin{itemize}
\item[\textbullet]
Лекции Чубарова И.А. 
\item[\textbullet]
Учебное Пособие Беклемишева Д.В. "Курс аналитической геометрии и линейной алгебры".
\end{itemize}

\item[\textit{26-29}] 
\; --- \: \textit{Дифференциальные уравнения.}
\begin{itemize}
\item[\textbullet] 
Учебное пособие Романко В.К. "Курс дифференциальных уравнений и вариационного исчисления".
\end{itemize}

\item[\textit{30-32}]
\; --- \: \textit{Теория вероятностей.}
\begin{itemize}
\item[\textbullet]
Лекции Райгородского А.М. (\href{http://lectoriy.mipt.ru/course/Maths-ProbabilityTheoryBasics-L15}{\textit{lectoriy.mipt.ru/...}}) и (\href{https://www.youtube.com/playlist?list=PLJOzdkh8T5kouOIbZDCqzB72hBn9T7gsJ}{\textit{youtube.com/...}\footnote{На всякий случай предупрежу, что второй набор лекций с ШАДа имеет мало общего с ГОСом, как и большинство курсов Райгородского в ШАДе и coursera.org, но полезные вещи, конечно, можно почерпнуть.}})
\item[\textbullet]
Пособие Севастьянова Б.А. "Курс теории вероятностей и математической статистики"
\item[\textbullet]
Учебное пособие Гнеденко Б.В. "Курс теории вероятностей"
\item [\textbullet]
Учебное пособие трех авторов: Захарова В.К., Севастьянов Б.А., Чистякова В.П. "Теория вероятностей"
\item[\textbullet]
Пособие Чистякова В.П. "Курс теории вероятностей"
\item[\textbullet]
Cеминарские записи Карлова М.И.
\end{itemize}

\item[\textit{33-36}]
\; --- \: \textit{Теория функций комплексного переменного.}
\begin{itemize}
\item[\textbullet]
Учебное пособие Половинкина Е.С. "Курс лекций по теории функции комплексного переменного"
\item[\textbullet] 
Лекции Карлова М.И. (\href{http://lectoriy.mipt.ru/course/Maths-ComplexAnalysis-13L}{\textit{lectoriy.mipt.ru/...}})
\item[\textbullet] 
Семинарские заметки Агаханова Н.Х.
\item[\textbullet] 
Шабат Б.В. "Введение в комплексный анализ"
\end{itemize}
\end{itemize}
