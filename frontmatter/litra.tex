\chapter{Список литературы и используемые материалы}

Предоставляю список литературы, которыми мы в основном пользовались для написания данной книги. 

Кстати говоря, почти всю перечисленную ниже литературу, вы сможете получить в электронном виде по следующей ссылке: \href{https://drive.google.com/drive/u/0/folders/0BzuzEyNkpwYDcENXcV9jNWdwVlU}{drive.google.com/...} (отсканированные pdf рукописных лекций лежат в папке ``Рукописные лекции'' и книги "--- в папке ``Книги'').

\boxtitle{I}{Введение в математический анализ}

\begin{itemize}[wide,  labelwidth=!, noitemsep, label=$\blacktriangleright$, labelindent = 0pt]
\item[$\bullet$]
Лекции Сакбаева В.Ж.
\item
Яковлев Г.Н. ``Лекции по математическому анализу'' 
\item
Бесов О.В. ``Лекции по математическому анализу''
\item
Иванов Г.Е. ``Лекции по математическому анализу'' 
\item 
Кудрявцев Л.Д. ``Курс математического анализа''
\item[$\blacksquare$]
Семинарские заметки Яковлевой Т.Х.
\end{itemize}

\boxtitle{II}{Многомерный анализ, интегралы и ряды.}
\begin{itemize}[wide,  labelwidth=!, noitemsep, label=$\blacktriangleright$, labelindent = 0pt]
\item[$\bullet$]
Аудиторные лекции Сакбаева В.Ж.
\item
Яковлев Г.Н. ``Лекции по математическому анализу''
\item 
Кудрявцев Л.Д. ``Курс математического анализа''
\item[$\blacksquare$]
Семинарские заметки Яковлевой Т.Х.
\end{itemize}

\boxtitle{III}{Кратные интегралы и теория поля.}
\begin{itemize}[wide,  labelwidth=!, noitemsep, label=$\blacktriangleright$, labelindent = 0pt]
\item[$\bullet$]
Аудиторные лекции Сакбаева В.Ж.
\item
Яковлева Г.Н. ``Лекции по математическому анализу'' 
\item
Петрович А.Ю. ``Лекции по математическому анализу''
\item[$\blacksquare$]
Семинарские заметки Яковлевой Т.Х.
\end{itemize}

\boxtitle{IV}{Гармонический анализ.}
\begin{itemize}[wide,  labelwidth=!, noitemsep, label=$\blacktriangleright$, labelindent = 0pt]
\item[$\bullet$]
Аудиторные лекции Сакбаева В.Ж.
\item
Яковлев Г.Н. ``Лекции по математическому анализу''
\item[$\blacksquare$]
Семинарские заметки Яковлевой Т.Х.
\end{itemize}

%\begin{samepage}
\boxtitle{V}{Аналитическая геометрия.}
\begin{itemize}[wide,  labelwidth=!, noitemsep, label=$\blacktriangleright$, labelindent = 0pt]
\item[$\bullet$]
Аудиторные лекции Чубарова И.А. 
\item
Беклемишев Д.В. ``Курс аналитической геометрии и линейной алгебры''
\item[$\blacksquare$]
Семинарские заметки Яковлевой Т.Х.
\end{itemize}
%\end{samepage}

\boxtitle{VI}{Линейная алгебра.}
\begin{itemize}[wide,  labelwidth=!, noitemsep, label=$\blacktriangleright$, labelindent = 0pt]
\item[$\bullet$]
Аудиторные лекции Чубарова И.А. 
\item
Беклемишев Д.В. ``Курс аналитической геометрии и линейной алгебры''
\item
Чехлов В.И. ``Лекции по аналитической геометрии и линейной алгебре''
\item[$\blacksquare$]
Семинарские заметки Яковлевой Т.Х.
\end{itemize}

\boxtitle{VII}{Дифференциальные уравнения.}
\begin{itemize}[wide,  labelwidth=!, noitemsep, label=$\blacktriangleright$, labelindent = 0pt]
\item
Романко В.К. ``Курс дифференциальных уравнений и вариационного исчисления''
\item[$\blacksquare$]
Семинарские заметки Яковлевой Т.Х.
\end{itemize}

\boxtitle{VIII}{Теория вероятностей.}
\begin{itemize}[wide,  labelwidth=!, noitemsep, label=$\blacktriangleright$, labelindent = 0pt]
\item[$\bullet$]
Онлайн-лекции Райгородского А.М. (\href{http://lectoriy.mipt.ru/course/Maths-ProbabilityTheoryBasics-L15}{lectoriy.mipt.ru/...} и \href{https://www.youtube.com/playlist?list=PLJOzdkh8T5kouOIbZDCqzB72hBn9T7gsJ}{youtube.com/...\footnote{На всякий случай предупрежу, что этот набор лекций с Школы Анализа Данных имеет мало общего с ГОСом, как и большинство курсов Райгородского в ШАДе и на \href{https://www.coursera.org}{coursera.org}, но полезные вещи, конечно, можно почерпнуть.}})
\item
Севастьянов Б.А. ``Курс теории вероятностей и математической статистики''
\item
Гнеденко Б.В. ``Курс теории вероятностей''
\item 
Захаров~В.К., Севастьянов~Б.А., Чистяков~В.П. ``Теория вероятностей''
\item
Чистяков В.П. ``Курс теории вероятностей''
\item[$\blacksquare$]
Семинарские заметки Карлова М.И.
\end{itemize}

\boxtitle{IX}{Теория функций комплексного переменного.}
\begin{itemize}[wide,  labelwidth=!, noitemsep, label=$\blacktriangleright$, labelindent = 0pt]
\item[$\bullet$]
Онлайн-лекции Карлова М.И. (\href{http://lectoriy.mipt.ru/course/Maths-ComplexAnalysis-13L}{lectoriy.mipt.ru/...})
\item
Половинкин Е.С. ``Курс лекций по теории функции комплексного переменного''
\item
Шабат Б.В. ``Введение в комплексный анализ''
\item[$\blacksquare$]
Семинарские заметки Агаханова Н.Х.
\end{itemize}
