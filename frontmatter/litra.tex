\chapter{Список литературы и используемые материалы}

Предоставляю список литературы, которыми мы пользовались в основном для написания билетов. 

Кстати говоря, почти всю перечисленную ниже литературу, вы сможете получить по следующей ссылке: \href{https://drive.google.com/drive/u/0/folders/0BzuzEyNkpwYDcENXcV9jNWdwVlU}{drive.google.com/...}.

\boxtitle{I}{Введение в математический анализ}

\begin{itemize}[wide,  labelwidth=!, noitemsep, label=$\blacktriangleright$, labelindent = 0pt]
\item
Лекции Сакбаева В.Ж.
\item
Учебное пособие Яковлева Г.Н. ``Лекции по математическому анализу'' (часть 1)
\item
Учебное пособие Бесова О.В. ``Лекции по математическому анализу''
\item
Учебное пособие Иванова Г.Е. ``Лекции по математическому анализу'' (часть 1)
\item
Семинарские заметки Яковлевой Т.Х.
\end{itemize}

\boxtitle{II}{Многомерный анализ, интегралы и ряды.}
\begin{itemize}[wide,  labelwidth=!, noitemsep, label=$\blacktriangleright$, labelindent = 0pt]
\item
Лекции Сакбаева В.Ж.
\item
Учебное пособие Яковлева Г.Н. ``Лекции по математическому анализу'' (1 и 2 части)
\end{itemize}

\boxtitle{III}{Кратные интегралы и теория поля.}
\begin{itemize}[wide,  labelwidth=!, noitemsep, label=$\blacktriangleright$, labelindent = 0pt]
\item
Лекции Сакбаева В.Ж.
\item
Учебное пособие Яковлева Г.Н. ``Лекции по математическому анализу'' (2 и 3 части)
\item
Учебное пособие Петровича А.Ю. ``Лекции по математическому анализу'' (3 часть)
\end{itemize}

\boxtitle{IV}{Гармонический анализ.}
\begin{itemize}[wide,  labelwidth=!, noitemsep, label=$\blacktriangleright$, labelindent = 0pt]
\item
Лекции Сакбаева В.Ж.
\item
Учебное пособие Яковлева Г.Н. ``Лекции по математическому анализу''  (часть 3).
\end{itemize}

\boxtitle{V}{Аналитическая геометрия.}
\begin{itemize}[wide,  labelwidth=!, noitemsep, label=$\blacktriangleright$, labelindent = 0pt]
\item
Лекции Чубарова И.А. 
\item
Учебное Пособие Беклемишева Д.В. ``Курс аналитической геометрии и линейной алгебры''.
\end{itemize}

\boxtitle{VI}{Линейная алгебра.}
\begin{itemize}[wide,  labelwidth=!, noitemsep, label=$\blacktriangleright$, labelindent = 0pt]
\item
Лекции Чубарова И.А. 
\item
Учебное Пособие Беклемишева Д.В. ``Курс аналитической геометрии и линейной алгебры''.
\end{itemize}

\boxtitle{VII}{Дифференциальные уравнения.}
\begin{itemize}[wide,  labelwidth=!, noitemsep, label=$\blacktriangleright$, labelindent = 0pt]
\item
Учебное пособие Романко В.К. ``Курс дифференциальных уравнений и вариационного исчисления''.
\end{itemize}

\boxtitle{VIII}{Теория вероятностей.}
\begin{itemize}[wide,  labelwidth=!, noitemsep, label=$\blacktriangleright$, labelindent = 0pt]
\item
Лекции Райгородского А.М. (\href{http://lectoriy.mipt.ru/course/Maths-ProbabilityTheoryBasics-L15}{lectoriy.mipt.ru/...}) и (\href{https://www.youtube.com/playlist?list=PLJOzdkh8T5kouOIbZDCqzB72hBn9T7gsJ}{youtube.com/...\footnote{На всякий случай предупрежу, что этот набор лекций с Школы Анализа Данных имеет мало общего с ГОСом, как и большинство курсов Райгородского в ШАДе и на \href{https://www.coursera.org}{coursera.org}, но полезные вещи, конечно, можно почерпнуть.}})
\item
Пособие Севастьянова Б.А. ``Курс теории вероятностей и математической статистики''
\item
Учебное пособие Гнеденко Б.В. ``Курс теории вероятностей''
\item 
Учебное пособие трех авторов: Захарова В.К., Севастьянов Б.А., Чистякова В.П. ``Теория вероятностей''
\item
Пособие Чистякова В.П. \``Курс теории вероятностей''
\item
Cеминарские заметки Карлова М.И.
\end{itemize}

\boxtitle{IX}{Теория функций комплексного переменного.}
\begin{itemize}[wide,  labelwidth=!, noitemsep, label=$\blacktriangleright$, labelindent = 0pt]
\item
Учебное пособие Половинкина Е.С. ``Курс лекций по теории функции комплексного переменного''
\item
Лекции Карлова М.И. (\href{http://lectoriy.mipt.ru/course/Maths-ComplexAnalysis-13L}{lectoriy.mipt.ru/...})
\item
Семинарские заметки Агаханова Н.Х.
\item
Шабат Б.В. ``Введение в комплексный анализ''
\end{itemize}
