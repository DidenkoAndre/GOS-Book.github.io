\chapter{Предисловие}

\begin{center} 
	Здравствуй, мой дорогой читатель!
\end{center}

Пособие, которое ты сейчас читаешь, "--- результат титанического труда многих людей. И мы все верим, что результат получился более чем достойный. 

Цель данной книги – облегчить подготовку студентов Московского Физико-Технического Института к устному выпускному квалификационному государственному экзамену по математике, попросту к ГОСу. Как вы уже поняли, оглавление представляет собой программу к ГОСу 2016 года. Заметьте, что оно <<кликабельно>>, что упрощает работу с книгой. И <<кликабельно>> не только оглавление данной книги, но и всевозможные числа и названия, указывающие на теоремы, которые уже использовались ранее в книге. Надеюсь, это кому-нибудь поможет. 

Надо бы отметить для любителей учебников и лекций определённых авторов, что мы писали билеты, существенно опираясь на учебные пособия, лекции различных преподавателей. Список соответствия билетов из программы и названий курсов от кафедры высшей математики материалам, которыми мы в основном пользовались (подчеркиваю, в основном) смотрите ранее, в списке литературы.

Мы были предельно внимательны к составлению данной книги, стараясь уменьшить количество опечаток и повысить качество излагаемого материала. Но мы отказываемся от ответственности за всевозможные недочеты в этой книге (пожалуй, главный недочет этой книги "--- чрезмерная избыточность в некоторых местах, а иногда, наоборот, недостаточно материала), ведь мы, на данный момент, всего лишь студенты, а главное "--- люди, которые могут ошибаться. И поэтому, прошу Вас не забывать отправлять нам (ссылки ниже) сообщения о любых неточностях, опечатках, ошибках, недочетах. Также пишите, если хотите дать совет или выразить любые личные пожелания. Вместе с вами можно довести эту книгу до очень хорошего пособия.

Не могу не отметить доброжелательного и внимательного отношения всех студентов МФТИ к этому пособию. Хочу сказать всем, кто присылал сообщения об опечатках и ошибках: \glqq Спасибо\grqq. Также хочется выразить особую благодарность Кудашову Аркадию, Лузянину Артемию, Проскину Роману, Вербе Глебу и Браславскому Илье за непосредственное соучастие в написании этой книги и выразить признательность Брицыну Евгению и Дроботу Олегу за многочисленные комментарии и исправления.

Не обошлось даже без участия преподавательского состава МФТИ. Так, например, Максим Широбоков, преподаватель теории вероятностей, случайных процессов и математической статистики, прочитал билеты по теории вероятностей, оставил важные и ценные комментарии, и в ходе долгой дискуссии после редактирования билетов одобрил последние. За это все я очень благодарен ему.

Также Чубаров Игорь Андреевич на своей очной консультации упомянул, что он читал это пособие, и сказал, что билеты по его предмету (аналитическая геометрия и линейная алгебра) написаны хорошо.
 
Мне лишь остается выразить надежду, что настоящее пособие поможет студентам при изучении математики в целом и подготовке к ГОСу. Но все же я настоятельно рекомендую пользоваться не только данным пособием при подготовке к ГОСу.
\vspace*{\baselineskip}
\hyphenation{ГОСе}

\mbox{}

Желаю всем отличных результатов~на~ГОСе.

\mbox{}

\noindent Диденко А. А.

\noindent\href{https://vk.com/didenko_andre}{$https://vk.com/didenko\_andre$}

\noindent\href{https://telegram.me/didenko_andre}{$https://telegram.me/didenko\_andre$}

\mbox{}

\noindent PS. Пользуйтесь Дидодичкой на здоровье, с удовольствием и умом. :)

\mbox{}

\noindent PPS. Вы можете сами помочь нам своими действиями, сделав \textit{Pull Request} или указав на существующий \textit{Issue} в следующем репозитории:

\noindent\href{https://github.com/DidenkoAndre/GOS_book}{$https://github.com/DidenkoAndre/GOS\_book$}
