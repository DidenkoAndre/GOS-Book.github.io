\chapter{Предисловие}
Здравствуй, мой дорогой читатель. Написать данное учебное пособие было титаническим трудом. Но я думаю, что я и мои коллеги справились достойно.

Данное учебное пособие предназначено для студентов Московского Физико-Технического Института для подготовки конкретно к устному ГОСу по математике. Как вы уже поняли, оглавление представляет собой программу к ГОСу 2016 года. Заметьте, что оно <<кликабельно>>, что упрощает работу с книгой. И <<кликабельно>> не только оглавление данной книги, но и всевозможные числа и названия, указывающие на теоремы, которые уже использовались ранее в книге. Надеюсь, это кому-нибудь поможет. 

Надо бы отметить для любителей учебников и лекций определенных авторов, что мы писали билеты, существенно опираясь на учебные пособия, лекции различных преподавателей. Список соответствия билетов из программы и названий курсов от кафедры высшей математики материалам, которыми мы в основном пользовались (подчеркиваю, в основном) смотрите ранее, в списке литературы.

Мы были предельно внимательны к составлению данной книги, стараясь уменьшить количество опечаток и повысить качество излагаемого материала. Но мы отказываемся от ответственности за всевозможные недочеты в этой книге, ведь мы, на данный момент, всего лишь студенты 3-ого курса, а главное --- люди, которые могут ошибаться. И поэтому, прошу Вас не забывать отправлять мне (ссылка ниже) сообщения о любых неточностях, опечатках, ошибках, недочетах. Также пишите, если хотите дать совет или выразить любые личные пожелания. 

Не могу не отметить доброжелательного и внимательного отношения всех студентов МФТИ к этому пособию. Хочу сказать всем: \glqq Спасибо\grqq, кто присылал сообщения об опечатках и ошибках. Также хочется выразить особую благодарность Кудашову Аркадию, Лузянину Артемию, Проскину Роман, Вербе Глебу и Браславскому Илье за соучастие в написании этой книги и выразить признательность Брицыну Евгению и Дроботу Олегу за многочисленные комментарии и исправления.

Не обошлось даже без участия преподавательского состава МФТИ. Так, например, Максим Широбоков, преподаватель теории вероятностей, случайных процессов и математической статистики, прочитал билеты по теории вероятностей, оставил важные и ценные комментарии, и в ходе долгой дискуссии после редактирования билетов одобрил их. За это я очень благодарен ему.
 
Мне лишь остается выразить надежду, что настоящее пособие поможет студентам при изучении математики в целом и подготовке к ГОСу.
\vspace*{\baselineskip}
\hyphenation{ГОСе}

Желаю всем отличных результатов~на~ГОСе.

\mbox{}

\noindent Диденко А. А.

\noindent\href{https://vk.com/didenko.andre}{$https://vk.com/didenko.andre$}

\noindent\href{https://telegram.me/didenko_andre}{$https://telegram.me/didenko\_andre$}

\mbox{}

\noindent PS. Вы можете сами помочь нам своими действиями, сделав Pull Request или указав на существующий Issue в следующем репозитории:

\noindent\href{https://github.com/DidenkoAndre/GOS_book}{$https://github.com/DidenkoAndre/GOS\_book$}