\chapter{Обозначения и элементы теории множеств}

Для сокращения записи используются следующие обозначения:
\begin{itemize}
\item[]
\begin{itemize}[noitemsep, label = ---]
\item \makebox[0pt][r]{$\forall$\hspace{0.75cm}}
квантор всеобщности <<для любого>>, <<для каждого>>, <<для всех>>;
\item \makebox[0pt][r]{$\exists$\hspace{0.75cm}}
квантор существования  <<существует>>, <<найдется>>;
\item \makebox[0pt][r]{$:$\hspace{0.75cm}}
логическая связка <<такой, что>>, <<такие, что>>;
\item \makebox[0pt][r]{$\triangleq$\hspace{0.75cm}} 
<<по обозначению равно>>;
\item \makebox[0pt][r]{$\to$\hspace{0.75cm}}
<<соответствует>>, <<поставлено в соответствие>> или <<стремится>>, <<при стремлении>>;
\item \makebox[0pt][r]{$\Rightarrow$\hspace{0.75cm}}
логическая связка <<следует>>;
\item \makebox[0pt][r]{$\Longleftrightarrow$\hspace{0.75cm}}
логическая связка <<равносильно>>, <<тогда и только тогда>>.
\end{itemize}
\end{itemize}

\textit{Множество} является одним из исходных понятий в математике, оно не определяется. Множество состоит из объектов, которые принято называть \textit{элементами}. Вместо слова <<множество>> иногда говорят \textit{<<набор>>, <<совокупность>>, <<собрание>>}. Множество состоит из объектов, которые принято называть его \textit{элементами}. Вводится также пустое множество, обозначаемое символом $\emptyset$, как множество, не содержащее ни одного элемента. Множества часто обозначают прописными буквами $X$,~$Y$,~$Z$,~\ldots, а элементы множеств "--- строчными.  Запись $x = y$ означает, что и $x$, и $y$ "--- это один и тот же элемент. Запись $x\neq y$ означает обратное. Запись $x\in X$, $X\ni x$ означает, что \textit{элемент $x$ содержится во множестве $X$, принадлежит $X$, множество $X$ содержит элемент~$x$}. Запись $x \notin X$ означает, что множество $X$ не содержит элемент~$x$. Причем
\begin{axiome}
Для любого $x$ из множества $X$ и любого множества $Y$ выполняется одно и только одно из двух условий: $x\in Y$ или $x\notin Y$.
\end{axiome}

\begin{defn}
Множество $Y$ называется \textit{подмножеством} множества $X$, если любой элемент множества~$Y$ является элементом множества~$X$. Обозначается $Y \subset X$, $X \supset Y$.
\end{defn}

Заметим, что с помощью квантора $\forall$ это можно записать следующим образом. Условие $X \subset Y$ выполняется, если $\forall y\in Y\quad  y\in X$. 

\begin{defn}
Если $X \subset Y$ и $Y \subset X$, то множества $X$ и $Y$ называют \textit{равными} между собой и пишут $X = Y$.
\end{defn}


При определении новых множеств часто используют:
\begin{enumerate}[wide, labelwidth=!, noitemsep]
\item 
Метод перечисления: 
$$
X = \{x_1,\,x_2,\,\dots,\,x_n,\,\dots \}, \text{ например, }
X = \{1,\, 2,\, \dots,\, n,\, \dots \};
$$

\item Метод наложения условия: 
$$
X = \{x\,\big|\, \text{выполняется некоторое условие для $x$}\}, \text{ например, }
X = \{x\,\big|\,x^2 < 1\}.
$$
\end{enumerate}

\begin{defn}
$Z = X \cup Y = \{z\,\big|\, z\in X \text{ и } z\in Y\}$ (\textit{объединение} множеств~$X$ и~$Y$) "--- множество, состоящее из всех элементов, каждый из которых принадлежит хотя бы одному из множеств $X$, $Y$.
\end{defn}
\begin{defn}
$Z = X \cap Y = \{z\,\big|\, z\in X \text{ или } z\in Y\}$ (\textit{пересечение} множеств~$X$ и~$Y$) "--- множество, состоящее из всех элементов, каждый из которых принадлежит как множеству $X$, так и множеству $Y$.
\end{defn}
\begin{defn}
$Z =X \setminus Y = \{z\,\big|\, z\in X \text{ и } z\notin Y\}$ (\textit{дополнение} множества~$Y$ до~$X$) "--- множество, состоящее из всех элементов, каждый из которых принадлежит множеству $x$, но при этом одновременно не принадлежит~$Y$.
\end{defn}

\begin{defn}
Пусть теперь заданы множества $X= \{x\}$, $Y = \{y\}$. Множество, состоящее из двух элементов $x\in X$ и $y\in Y$, называется \textit{парой}~\{$x$,~$y$\} элементов $x$, $y$. 
\end{defn}

\begin{defn}
Пара вида \{$x$,~\{$x$, $y$\}\}, где $x\in X$, $y\in Y$, \{$x$,~$y$\} "--- пара элементов $x$, $y$, называется \textit{упорядоченной парой} элементов $x$ и $y$. Элемент $x$ называется первым элементом упорядоченной пары \{$x$,~\{$x$,~$y$\}\}, а элемент $y$ "--- вторым. Упорядоченная пара \{$x$,~\{$x$,~$y$\}\} обозначается через~($x$,~$y$). В дальнейшем под парой обычно понимается упорядоченная пара.
\end{defn}

\begin{defn}
Множество всех упорядоченных пар ($x$,~$y$), $x\in X$, $y \in Y$, называется \textit{прямым (декартовым) произведением множеств} $X$ и $Y$ и обозначается через $X\times Y$. 
\end{defn}
При этом не предполагается, что обязательно множество $X$ отлично от множества $Y$, т.е.~возможен и случай, когда $X = Y$.