todolist

 2) жирные просьбы - теорема о монотонных односторонних, сократить параграф с обозн., посмотреть используется конечная-произвбеск.
\3)textbf{CМОТРИ В БИЛЕТЕ 6 пока что. + странно что материал страниц 116 нигде не пригодился, хотя это фундаментальные понятия и леммы}
 4)картиночки
 5)добавить везде ВСЕ нужные теоремы и определения
 6)если успеется. то и все-все доказательства
 7)орнамент?)
 8)рисунок райгородского в формуле полной вероятности?
 9) надо оформить все функции вставляющие картинки как \usepict с автоматическим label и pictures/
 
 10) надо сделать рисунок катета и гипотенузы в УКР билета 33
 11) Из теоремы \hyperref[exp14]{о трех функциях} следует: $\exists \lim_{\substack{\Delta x \to 0\\ \Delta y \to 0}}\limits \frac{|\alpha_1(\Delta x,\Delta y)|}{\sqrt{\Delta x ^2 + \Delta y^2}} = 0.$
Аналогично, $\exists \lim_{\substack{\Delta x \to 0\\ \Delta y \to 0}}\limits \frac{|\alpha_2(\Delta x,\Delta y)|}{\sqrt{\Delta x^2 + \Delta y^2}} = 0$. 
Отсюда равенства \eqref{exp15} означают дифференцируемость функций $u(x,y)$ и $v(x,y)$ в точке $(x_0,y_0) \in \bbR^2$, причем 


33 билет - некрасиво же.

12) картинка ,33.4 там Г хотя я в книге использую обозначение партиал Ж для края

13) относительно 12 подумай какое лучше обозначение для края будет с плюсиком или без плюсика. 

в header.tex есть строчка геометрии. перед каждым аутпутом надо ее менять на ровные границы.

14) (\textbf{пока непонятно что делать, но это яковлев 90})

15) избавиться от \fa, \ex 
16) написать о геометрической вероятности в 30 билете.
17) определение зависимости событий

Здесь было бы полезно сказать, что события не могут быть независимыми или зависимыми сами по себе. Свойство независимости событий зависит от введенной вероятностной меры. Если на сигма-алгебре ввести одну вероятностную меру, то события могут оказаться независимыми, а если другую вероятностную меру, то могут оказаться зависимыми.

И уже тем более, никакого отношения к причинно-следственным связям стохастическая независимость не имеет, об этом тоже можно сказать.

18) Обязательно нужна теорема ЗБЧ по Хинчину (которая доказывается с помощью хар. функций) и УЗБЧ Колмогорова (если эти теоремы были в курсе, естественно). И в этом случае надо быть готовым сказать, что такое хар. функция и ее простейшие свойства.

А добавил бы я эти теоремы хотя бы потому, что на практике ими чаще  пользуются. В матстатистике так это вообще рабочие инструменты.

19) проблема с содержпнаием в пдфке самой \textpdforcript который акробат ридрер восьмой билет и не только.

20) нормальные ссылки \label \ref расставить.

21) отметить те книги, которые есть в рекомендуемой к госу литре

22) разбить на тома (части) соответсвующие каждому предмету и использовать орнамент на них :-D K P A C U B O

23)сделать адекватный бэкматтер

24) Добавить синию или черную линию для супер обязательных вещей.

25) сделать супер шпору со всеми необходимыми формулами.

26) добавить рекомендацию ботать по норм книгам в предисловии

27) про непрерывность слева функции распределения

28) отметить в аппендиксе кучи материалов.

29) добавить лису фокси вообще везде.

30) самая дурацкая дуга на свете в тфкп

31) формулы приведения
32) ПриложениеА!!!!!!!!!!!!!!!!!!!!!!!!!!!!!!!фансичтдр

33) найти более оптимальнуювещь для my eq. (укр кур)

34) СИНЯЯ ЛИНИЯ --- ГЛЕБ В ТЕЛЕГРАМЕ ОТПРАВЛЯЛ.

35) определение 4 стр 15 половинкин.и подобная ересь.
опред 5 стр 16

36) вопрос Лехи и Павла