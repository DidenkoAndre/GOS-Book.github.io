todolist

2) жирные просьбы - теорема о монотонных односторонних, сократить параграф с обозн., посмотреть используется конечная-произвбеск.

\3)textbf{CМОТРИ В БИЛЕТЕ 6 пока что. + странно что материал страниц 116 нигде не пригодился, хотя это фундаментальные понятия и леммы}
 
4)картиночки везде добавить и поработать над обтеканием их.

5)добавить везде ВСЕ нужные теоремы и определения
 
6) и все-все доказательства
 
8)рисунок райгородского в формуле полной вероятности?
 
9) надо оформить все функции вставляющие картинки как \usepict с автоматическим label и pictures/
 
10) надо сделать рисунок катета и гипотенузы в УКР билета 33
(Как Карлов пояснил вообще написать)
 
12) картинка ,33.4 там Г хотя я в книге использую обозначение партиал Ж для края

13) относительно 12 подумай какое лучше обозначение для края будет с плюсиком или без плюсика. 

14) (\textbf{пока непонятно что делать, но это яковлев 90})

15) избавиться от \fa, \ex 

16) написать о геометрической вероятности в 30 билете.

17) в определение зависимости событий

Здесь было бы полезно сказать, что события не могут быть независимыми или зависимыми сами по себе. Свойство независимости событий зависит от введенной вероятностной меры. Если на сигма-алгебре ввести одну вероятностную меру, то события могут оказаться независимыми, а если другую вероятностную меру, то могут оказаться зависимыми.

И уже тем более, никакого отношения к причинно-следственным связям стохастическая независимость не имеет, об этом тоже можно сказать.

18) Обязательно нужна теорема ЗБЧ по Хинчину (которая доказывается с помощью хар. функций) и УЗБЧ Колмогорова (если эти теоремы были в курсе, естественно). И в этом случае надо быть готовым сказать, что такое хар. функция и ее простейшие свойства.

А добавил бы я эти теоремы хотя бы потому, что на практике ими чаще  пользуются. В матстатистике так это вообще рабочие инструменты.

20) нормальные ссылки \label \ref расставить. (а не label{fuck} всякие там, позор же :D о чем я думал)

23)сделать адекватный бэкматтер, может вообще как в книгах сделать его.

25) сделать супер шпору со всеми необходимыми формулами в аппендиксе. может сделать ссылку на печать супермини версию.

30) самая дурацкая дуга на свете в тфкп в 33 билете в параграфе про интегралы. смотрится убого.

35) определение 4 стр 15 половинкин.и подобная ересь.
опред 5 стр 16. Вводить ли снова такие основы основ. 

Или сослаться на материал предыдущих билетов? (очень плохо.)

36) вопрос Лехи Малышева про R и [a,b] и Павла Останина че-то там про дифференцируемость.

41) Добавить в аппендикс всевозможные полезные ссылки для бота, для письменных экзаменов, собрать все вообще.

42) исправить максимальное количество всевозможных warning'и. и вообще overful и underfull некрасиво смотрятся.

44) (ведь тогда $\exists\, \delta$ такая, что $\dneio{\delta}{x_0}\cap D_f=\emptyset$). это из первого билета.

подумать, как это можно более верно сказать. Противоречие с кольцом. что если всегда для любого эпсилон можно найти кольцо вокруг точки х_0 что свойство предела будет выполнено.

45) переход к пределу в неравенствах используешь в билете 2, но даже не написал формулировки.

46) замечание про точки разрыва с 93 страницы. В связи с чем перегруппировать содержание 2 и 3 билета.

47) рисунок в теорему о промежуточных значениях. собственно ручный.


49) все скобочки в италикс поменять на текстап потому что убого смотрится.

50) сделать в самом конце различные мини-версии для мини-печати и сделать ссылки на них.

51) формулы
\subsubsection{Производная сложной функции, параметрически заданной, параграф 13 кудрявцева 1}

кривые кудрявцев 511 страница. 255 в пдфке

52) сделать ссылки на википедию в формулах.

56) 34 билет последнее следствие ушло

57) в линале есть замечание под опрделением.

58) 33 билет разобраться со сноской

59) От Евгении Шульгиной посмотреть редактирование предисловия.

60) Написать, что госбук был сделан в 2016 году, а может и вообще историю идеи. 

63) остались какие-то проблемы с \,dx\,dy

64) сделать разделение на материал полезный и именно к билету относящийся

69)