%\titlespacing{\subsubsection}{0pt}{0mm plus 1mm minus 1mm}{0pt plus 1mm minus 1mm}  
%\titlespacing{\section}{0pt}{1mm plus 2mm minus 1mm}{1pt plus 2mm minus 1mm}  
\renewcommand*{\arraystretch}{1.5}
\everymath{\displaystyle}
\addtocontents{toc}{\protect\setcounter{tocdepth}{0}}
\addtocontents{ptc}{\protect\setcounter{tocdepth}{2}}
\setlength{\parindent}{0cm}

\part{Приложения}
\chapter[Формулы к письменному ГОСу]{Формулы к письменному ГОСу}

Прошу прощения, но для компактности в данном разделе не указаны допустимые значения аргументов и параметров в большинстве случаев. Имейте в виду.

\section{Тригонометрические формулы}

\subsubsection{Основные тригонометрические формулы}
\begin{longtable}[l]{l l l}
$\sin^2x+\cos^2x=1;$
&
$\tg^2x+1=\frac{1}{\cos^2x};$
&
$\tg x\cdot\ctg x=1.$
\end{longtable}

\subsubsection{Формулы сложения и вычитания аргументов}
\begin{longtable}[l]{l l l}
$\sin(x\pm y)=\sin x \cos y \pm \cos x \sin y;$
&
$\tg(x\pm y)=\frac{\tg x \pm \tg y}{1 \mp \tg x\tg y};$
\\
$\cos(x\pm y)= \cos x \cos y \mp \sin x \sin y;$
&
$\ctg(x\pm y)=\frac{\ctg x\ctg y \mp 1}{\ctg x \pm \ctg y}.$
\end{longtable}

\subsubsection{Формулы двойного угла}
\begin{longtable}[l]{ l l }
$\sin 2x = 2\sin x \cos x;$
&
$\tg 2x = \frac{2\tg x}{1-\tg^2x};$
\\
$\cos 2x = \cos^2 x - \sin^2 x;$
&
$\ctg 2x = \frac{\ctg^2x-1}{2\ctg x}.$
\end{longtable}

\subsubsection{Формулы понижения степени}
\begin{longtable}[l]{ l l }
$2\sin^2x=1-\cos 2x;$
&
$2\cos^2x=1+\cos 2x;$
\\
$4\sin^3x=3\sin x-\sin 3x;$
&
$4\cos^3x=3\cos x+\cos 3x;$
\end{longtable}

\subsubsection{Формулы преобразования произведения}
\begin{longtable}[l]{ l}
$2\sin x \sin y = \cos (x-y)-\cos(x+y);$
\\
$2\cos x \cos y = \cos (x-y)+\cos(x+y);$
\\
$2\sin x \cos y = \sin (x-y)+\sin(x+y).$
\end{longtable}

\newpage
\subsubsection{Формулы преобразования суммы}
\begin{longtable}[l]{ l}
$\sin x\pm \sin y =2 \sin \frac{x\pm y}{2} \cos \frac{x\mp y}{2};$
\\
$\cos x + \cos y  = 2 \cos \frac{x + y}{2} \cos \frac{x - y}{2};$
\\
$\cos x - \cos y  = -2 \sin \frac{x + y}{2} \sin\frac{x - y}{2};$
\\
$\tg x \pm \tg y = \frac{\sin (x \pm y)}{ \cos x \cos y};$
\end{longtable}
  
\subsubsection{Формула Эйлера и ее следствия}
\begin{longtable}[l]{ l l l}
$e^{ix} = \cos x + i \sin x;$
&
$\sin x = \frac{e^{ix}-e^{-ix}}{2i};$
&
$\cos x = \frac{e^{ix}+e^{-ix}}{2}.$
\end{longtable}

\subsubsection{Формулы приведения}
\tiny
\begin{longtable}[l]{|c|c|c|c|c|c|c|c|c|}
\hline
&
$\pi/2-x$
&
$\pi/2+x$
&
$\pi-x$
&
$\pi+x$
&
$3\pi/2-x$
&
$3\pi/2+x$
&
$2\pi-x$
&
$2\pi+x$
\\\hline
$\sin$
&
$\cos x$
&
$\cos x$
&
$\sin x$
&
$-\sin x$
&
$-\cos x$
&
$-\cos x$
&
$-\sin x$
&
$\sin x$
\\\hline
$\cos$
&
$\sin x$
&
$-\sin x$
&
$-\cos x$
&
$-\cos x$
&
$-\sin x$
&
$\sin x$
&
$\cos x$
&
$\cos x$
\\\hline
$\tg$
&
$\ctg x$
&
$-\ctg x$
&
$-\tg x$
&
$\tg x$
&
$\ctg x$
&
$-\ctg x$
&
$-\tg x$
&
$\tg x$
\\\hline
$\ctg$
&
$\tg x$
&
$-\tg x$
&
$-\ctg x$
&
$\ctg x$
&
$\tg x$
&
$-\tg x$
&
$-\ctg x$
&
$\ctg x$
\\\hline
\end{longtable}

$ $

\normalsize
Алгоритм действий, который здесь работает:
\begin{enumerate}
\item
Определите знак первоначальной функции в соответствующей четверти. Ставим этот знак перед новой функцией. Напомню знаки:
\usepict[0.12]{forpict1}
\item
\begin{itemize}
\item 
При $\pi/2$ и $3\pi/2$ функция меняется на кофункцию.
\item
При $\pi$ и $2\pi$ функция не меняется на кофункцию.
\end{itemize}
\end{enumerate}

\section{Гиперболические функции}

\subsubsection{Определение}
\begin{longtable}[l]{ l l}
$\sh x = \frac{e^{x}-e^{-x}}{2};$
&
$\th x = \frac{\sh x}{\ch x}=\frac{e^{x}-e^{-x}}{e^{x}+e^{-x}}=\frac{e^{2x}-1}{e^{2x}+1};$
\\
$\ch x = \frac{e^{x}+e^{-x}}{2};$
&
$\cth x = \frac{1}{\th x};$
\end{longtable}

\subsubsection{Связь с тригонометрическими функциями}
\begin{longtable}[l]{  l l l}
$\ch (ix) = \cos x;$
&
$\sh (ix)= i \sin x;$
&
$\th (ix) = i \tg x;$
\\
$\cos (ix) = \ch x;$
&
$\sin (ix) = i \sh x;$
&
$\tg (ix) = i \th x.$
\end{longtable}

\subsubsection{Важное соотношение}
\begin{longtable}[l]{  l}
$\ch^2 x - \sh^2 x = 1;$
\end{longtable}

\subsubsection{Формулы сложения и вычитания аргументов}
\begin{longtable}[l]{  l l}
$\sh (x\pm y) =  \sh x \ch y \pm \sh y \ch x;$
&
$\th (x \pm y) = \frac{\th x \pm \th y}{1 \pm \th x \th y};$
\\
$\ch (x\pm y) =  \ch x \ch y \pm \sh y \sh x;$
&
$\cth (x \pm y) = \frac{1 \pm \cth x \cth y}{\cth x \pm \cth y}.$
\end{longtable}

\subsubsection{Формулы двойного угла}
\begin{longtable}[l]{  l l}
$\sh 2x = 2\ch x\sh x$
&
$\th 2x = \frac{2\th x}{1 + \th^2 x}$
\\
$\ch 2x = \ch^2 x + \sh^2 x$
&
$\cth 2x = \frac{1}{2}(\th x + \cth x)$
\end{longtable}

\subsubsection{Формулы понижения степени}
\begin{longtable}[l]{  l l}
$2\ch^2 x = \ch 2x + 1$
&
$2\sh^2 x = \ch 2x - 1$
\\
$4\sh^3 x = \sh 3x - 3 \sh x$
&
$4 \ch^3 x = \ch 3x + 3 \ch x$
\end{longtable}

\subsubsection{Формулы преобразования произведения}
\begin{longtable}[l]{ l}
$\sh x \sh y = \frac{\ch(x + y) - \ch (x - y)}{2}$
\\
$\sh x \ch y = \frac{\sh(x + y) + \sh(x - y)}{2}$
\\
$\ch x \ch y = \frac{\ch(x + y) + \ch(x - y)}{2}$
\end{longtable}

\subsubsection{Формулы преобразования суммы}
\begin{longtable}[l]{  l}
$\sh x \pm \sh y = 2 \sh\frac{x \pm y}{2} \ch\frac{x \mp y}{2}$
\\
$\ch x + \ch y = 2 \ch\frac{x + y}{2}\ch\frac{x - y}{2}$
\\
$\ch x - \ch y = 2 \sh\frac{x + y}{2}\sh\frac{x - y}{2}$
\\
$\th x \pm \th y = \frac{\sh(x \pm y)}{\ch x \ch y}$
\end{longtable}

\section{Обратные гиперболические функции}

\begin{longtable}[l]{l}
$\arsh x = \ln(x + \sqrt{x^2 + 1})$
\\
$\arch x = \ln(x + \sqrt{x^2 - 1}); \: x \ge 1$
\\
$\arth x = \frac{1}{2}\ln \left(\frac{1 + x}{1 - x} \right),\ |x|<1.$
\\
$\arcth x = \frac{1}{2}\ln \left( \frac{x+ 1}{x - 1} \right),\ |x|<1.$
\end{longtable}

\section{Производные}

\subsubsection{Формулы для производных элементарных функций}
\begin{longtable}[l]{  l l }
$(C)'=0$ ($C$ --- постоянная); 
&
$(e^x)'=e^x;$
\\
$(x^n)' = nx^{n-1}, n\in \bbN;$
&
$(x^a)'=ax^{a-1},\ x>0;$
\\
$(a^x)'=\ln a\cdot a^x,\ a>0;$
&
$(\log_a x)'=\frac{\log_a e}{x};$%,\ x>0,\ a>0;$
\\
$(\ln x)' = \frac{1}{x};$%,\ x>0;$
&
$(\ln |x|)' = \frac{1}{x};$%,\ x\ne 0;$
\\
$(\sin x)'=\cos x;$
&
$(\cos x)'=-\sin x;$
\\
$(\tg x)'= \frac{1}{\cos^2 x};$%,\ x \ne \frac{\pi}{2}(2n+1),\, n\in \bbZ;
&
$(\ctg x)'= -\frac{1}{\sin^2 x};$ 
\\
$(\arcsin x)' = \frac{1}{\sqrt{1-x^2}};$
&
$(\arccos x)' = -\frac{1}{\sqrt{1-x^2}};$
\\
$(\arctg x)' = \frac{1}{1+x^2}$
&
$(\arcctg x)' = -\frac{1}{1+x^2}$
\\
$(\sh x)'= \ch x;$
&
$(\ch x)'= \sh x;$
\\
$(\th x)'= \frac{1}{\ch^2 x};$
&
$(\cth x)'= -\frac{1}{\sh^2 x};$
\\
$(\arsh x)'=\frac{1}{\sqrt{x^2+1}};$
&
$(\arch x)'=\frac{1}{\sqrt{x^2-1}};$
\\
$(\arth x)'=\frac{1}{1-x^2};$
&
$(\arcth x)'=-\frac{1}{x^2-1};$
\end{longtable}

\subsubsection{Формула Лейбница}
\begin{longtable}[l]{l}
$(uv)^{(n)}=\sum\limits^n_{k=0} C^k_n u^{n-k} v^{k}$
\end{longtable}

\section{Ряд Тейлора}

\begin{notion}
Здесь указаны ряды Тейлора при $x \to 0$, т.е. ряды Маклорена. За $R$ обозначен радиус сходимости. Для ряда $\sum\limits_{n=-\infty}^{+\infty} c_n (z-z_0)^n $ его можно определить по формуле Коши"--~Адамара:
$$
R = \frac{1}{\overline{\lim\limits_{n \to + \infty}} \sqrt[n]{|c_n|}}
$$
Более удобная, но менее универсальная формула:$$
R = \lim\limits_{n \to + \infty} \left| \frac{c_n}{c_{n + 1}} \right|
$$
\end{notion}

\begin{longtable}{ l l }
$e^x=1+\frac{x}{1!}+\frac{x^2}{2!}+\frac{x^3}{3!}+o(x^3);$
&
$e^x=\sum\limits_{n=0}^{+\infty}\displaystyle\frac{x^n}{n!},\ R=\infty;$
\\
$\sin x = x - \frac{x^3}{3!} + \frac{x^5}{5!}+o(x^5);$
&
$\sin x = \sum\limits_{n=0}^{+\infty} (-1)^{n}\displaystyle\frac{x^{2n+1}}{(2n+1)!},\; R=\infty;$
\\
$\cos x = 1 - \frac{x^2}{2!} + \frac{x^4}{4!}+o(x^4);$
&
$\cos x  = \sum\limits_{n=0}^{+\infty} (-1)^{n}\displaystyle\frac{x^{2n}}{(2n)!},\; R=\infty;$
\\
$\tg x = x + \frac{x^3}{3}+\frac{2x^5}{15}+o(x^5)$
&
$\tg x  = \sum\limits_{n=0}^{+\infty} \frac{B_{2n}\cdot (-4)^n (1-4^n)}{(2n)!} x^{2n-1},\; R=\frac{\pi}{2};$ \footnote{В этой формуле $B_{2n}$ "--- это числа Бернулли. На самом деле, далеко не все формулы надо знать наизусть. Но левый столбец из формул Маклорена по-хорошему надо для ГОСа знать}
\\ 
$\sh x = x +\frac{x^3}{3!}+\frac{x^5}{5!}+o(x^5);$
&
$\sh x =\sum\limits_{n=0}^{+\infty}\displaystyle\frac{x^{2n+1}}{(2n+1)!},\ R=\infty;$
\\
$\ch x = 1 + \frac{x^2}{2!}+\frac{x^4}{4!}+o(x^4);$
&
$\ch x  = \sum\limits_{n=0}^{+\infty}\displaystyle\frac{x^{2n}}{(2n)!},\ R=\infty;$
\\
$\th x = x - \frac{x^3}{3}+\frac{2x^5}{15}+o(x^5);$
&
$\th x  = \sum\limits_{n=0}^{+\infty} \frac{B_{2n} 4^n (4^n-1)}{(2n)!} x^{2n-1}, R=\frac{\pi}{2};$
\\
$\arcsin x = x + \frac{x^3}{6}+\frac{3x^5}{40}+o(x^5);$
&
$\arcsin x=\sum\limits_{n=0}^{+\infty} \frac{(2n)!}{4^n(n!)^2(2n + 1)}x^{2n + 1},  R = 1 ;$
\\
$\arccos x= \frac{\pi}{2} - \arcsin x;$
&
$\arcctg x = \frac{\pi}{2} - \arctg x;$
\\
$\arctg x = x - \frac{x^3}{3}+\frac{x^5}{5}+o(x^5);$
&
$\arctg x = \sum\limits_{n=0}^{+\infty} (-1)^n\displaystyle\frac{x^{2n+1}}{2n+1},\ R=1;$
\\
$\ln(1+x)=x-\frac{x^2}{2}+\frac{x^3}{3}+o(x^3);$
&
$\ln(1+x)= \sum\limits_{n=1}^{+\infty} \frac{(-1)^{n+1}}{n} x^n, R=1;$
\\
$\ln(1-x)=-x-\frac{x^2}{2}-\frac{x^3}{3}+o(x^3);$
&
$\ln(1-x)= -\sum\limits_{n=1}^{+\infty} \frac{1}{n} x^n, R=1;$
\\
$
(1+x)^{a}=1+ax+\displaystyle\frac{a(a-1)}{2!}x^2+
$
&
$(1+x)^{a}= \sum\limits_{n=0}^{+\infty} C^{n}_{a} x^n,\;\text{где} $
\\
$\qquad+\displaystyle\frac{a(a-1)(a-2)}{3!}x^3+o(x^3);$
&
$\quad C^{n}_{a}=\frac{a(a-1)\dots(a-n+1)}{n!},\; R=1;$
\\
$\frac{1}{1-x}=1+x+x^2+x^3+o(x^3);$
&
$\frac{1}{1-x}= \sum\limits_{n=0}^{+\infty} x^{n};$
\\
$\arsh x = x - \frac{x^3}{6}+\frac{3x^5}{40}+o(x^5);$
&
$\arsh x = \sum\limits_{n=0}^{+\infty} \frac{(-1)^n(2n)!}{4^n (n!)^2 (2n + 1)} x^{2n + 1}, R = 1;$
\\
$\arth x = x + \frac{x^3}{3}+\frac{x^5}{5}+o(x^5);$
&
$\arth x = \sum\limits_{n=0}^{+\infty} \frac{1}{2n + 1} x^{2n + 1}, R = 1;$
\end{longtable}

\subsubsection{Несколько рядов Лорана в догонку}
\begin{longtable}{ l l l }
$\cth x = \frac{1}{x} + \frac{x}{3}+\frac{x^3}{45}+o(x^3);$
&
$\cth x  = \sum\limits_{n=0}^{+\infty} \frac{2^{2n}B_{2n}x^{2n-1}}{(2n)!},\quad0<|x|<\pi;$
\\
$\arch x = \ln 2x - \frac{1}{4x^2}-\frac{3}{32x^4}+o\left(\frac{1}{x^{4}}\right);$ 
&
$\arch x = \ln 2x - $
\\
& $-\sum_{n=1}^\infty \left( \frac {(2n)!} {2^{2n}(n!)^2} \right) \frac {x^{-2n}} {(2n)} , \ |x| > 1; $
\\
$\arcth x = =\frac{1}{x}+\frac{1}{3x^3}+\frac{1}{5x^5}+o\left(\frac{1}{x^5}\right); $
&
$\arcth x = \sum_{\infty}^{n=1}\frac{1}{(2n-1)x^{2n-1}}, x > 1.$
\end{longtable}

\section{Основные неопределенные интегралы}
\begin{longtable}[l]{l l}
$\int x^\alpha \,dx = \frac{x^{\alpha+1}}{\alpha+1},\; \alpha\neq -1;$ & $ $ 
\\ $\int\frac{dx}{x+a} = \ln|x+a|+C;$ & $ $ 
\\ $\int a^x dx = \frac{a^x}{\ln a} + C, \; a>0,\; a\neq 1;$ & $ $ 
\\ $\int \sin x\,dx =-\cos x + C; $ 
& $\int \cos x \,dx = \sin x + C;$ 
\\ $\int \frac{dx}{\cos^2 x} = \tg x + C;$ 
& $\int \frac{dx}{\sin^2 x} = -\ctg x + C;$ 
\\ $\int \sh x \,dx = \ch x + C; $ 
& $\int \ch x \,dx = \sh x + C;$ 
\\ $\int \frac{dx}{\ch^2 x} = \th x + C; $ 
& $\int \frac{dx}{\sh^2 x} = -\cth x + C; $ 
\\ \multicolumn{2}{l}{$\int\frac{dx}{x^2+a^2} = \frac{1}{a}\arctg\frac{x}{a} + C = -\frac{1}{a}\arctg\frac{x}{a} + C,\; a\neq 0; $}
\\ \multicolumn{2}{l}{$\int\frac{dx}{x^2-a^2} = \frac{1}{2a}\ln\left|\frac{x-a}{x+a}\right| + C, \; a\neq 0 ;$ }
\\ \multicolumn{2}{l}{$\int\frac{dx}{\sqrt{a^2-x^2}} = \arcsin\frac{x}{a} + C=-\arccos\frac{x}{a} + C,\; |x|<a,\;a\neq 0;$}
\\ \multicolumn{2}{l}{$\int\frac{dx}{\sqrt{x^2+a^2}} = \ln|x+\sqrt{x^2+a^2}|+C,\; a\neq 0; $}
\\\multicolumn{2}{l}{$\int\frac{dx}{\sqrt{x^2-a^2}}= \ln|x+\sqrt{x^2-a^2}|+C,\; a\neq 0 \;(|x|<|a|). $}
\end{longtable}

\section{Вычисление площадей плоских фигур и длин кривых}



\section{Вычисление объемов тел и площадей поверхностей}

\section{Несобственные интегралы}
\begin{itemize}
\item
Интеграл $\int\limits_{1}^{+\infty} \frac{dx}{x^a}$ сходится при $\alpha > 1$ и расходится при $\alpha \le 1$.
\item 
Интеграл $\int\limits_{0}^{1} \frac{dx}{x^a}$ сходится при $\alpha < 1$ и расходится при $\alpha \ge 1$.
\item
Интеграл $\int\limits_{2}^{+\infty} \frac{dx}{x^a |\ln x|^{\beta}}$ сходится при $\alpha > 1$ и любых $\beta$, расходится при $\alpha < 1$ и любых $\beta$, а если $\alpha = 1$, то при $\beta > 1$ интеграл сходится, а при $\beta \le 1$ "--- расходится.
\item 
Интеграл $\int\limits_{0}^{2} \frac{dx}{x^a |\ln x|^{\beta}}$ сходится при $\alpha < 1$ и любых $\beta$, расходится при $\alpha > 1$ и любых $\beta$, а если $\alpha = 1$, то при $\beta > 1$ интеграл сходится, а при $\beta \le 1$ "--- расходится.
\end{itemize}

\section{Криволинейные интегралы}
\subsubsection{Первого рода}
$\int\limits_{\Gamma} F(x,y,z) ds = \int\limits_{\alpha}^{\beta} F(x(t),y(t),z(t))\sqrt{(x'(t))^2+(y'(t))^2+(z'(t))^2}\,dt$

$\int\limits_{\Gamma} F(x,y) ds = \int\limits_{\alpha}^{\beta} F(x,f(x))\sqrt{1+(f'(x))^2}\,dt, \quad y=f(x)$

\subsubsection{Второго рода}
$\int\limits_{\Gamma} P\,dx+Q\,dy+R\,dz=\int\limits_{\alpha}^{\beta} \left(P x'(t)+ Q y'(t)+ R z'(t)\right)\,dt$

\subsubsection{Формула Грина}
$\oint\limits_{\Gamma ^{+}} P \,dx + Q \,dy = \iint\limits_{D} \left( \frac{\partial Q}{\partial x} - \frac{\partial P}{\partial y} \right) \,dx\,dy$

\section{Поверхностные интегралы}
\subsubsection{Первого рода} 
${ I=\iint \limits _{S}{f\left(x, y, z\right)dS }=\iint \limits _{D}{f\left(x\left(u,v\right),y\left(u,v\right),z\left(u,v\right)\right){\sqrt {EG-F^{2}}}\;du\;dv}}$

${ E=\left(x_{u}'\right)^{2}+\left(y_{u}'\right)^{2}+\left(z_{u}'\right)^{2}}, \quad {F=x_{u}'\;x_{v}'+y_{u}'\;y_{v}'+z_{u}'\;z_{v}'}, \quad { G=\left(x_{v}'\right)^{2}+\left(y_{v}'\right)^{2}+\left(z_{v}'\right)^{2}}$

$ I=\iint \limits _{S}{f\left(x, y, z\right)dS } = \iint \limits _{D} f(x,y) \sqrt {z_x^2 + z_y^2 + 1}dxdy, \quad z=z(x, y) $

\subsubsection{Второго рода} 

\qquad  \qquad  \qquad \qquad \qquad \qquad \qquad \qquad \qquad (считать как интеграл 1 рода)

${\iint \limits _{S }{P\;dy\;dz+Q\;dz\;dx+R\;dx\;dy}} = {\iint \limits _{S }{((P,Q,R),n)}dS}) = \iint \limits _{S}{det \begin{vmatrix}P&Q&R\\x_u&y_u&z_u\\x_v&y_v&z_v\end{vmatrix}}dudv$


$n = [r_u, r_v], \quad r = r(u,v) = (x(u,v), y(u,v), z(u,v),) , \quad n = \frac{-z_x i-z_y j+k}{\sqrt{(z_x)^2 + (z_y)^2 + 1}}, \quad z = z(x,y)$


\subsubsection{Формула Гаусса-Остроградского}
${\iint \limits _{S^{+}}{P\;dy\;dz+Q\;dz\;dx+R\;dx\;dy}} = {\iiint \limits _{G} \left(P_x + Q_y + R_z\right)dxdydz}, \quad \iiint\limits_V\mathrm{div}\,\mathbf{F}\,dV=\int\limits_{\;\,S}\!\!\!\!\int\!\!\!\!\!\!\!\!\!\!\!\!\;\!\!\;\subset\!\!\supset\mathbf F\cdot\mathbf{n}\,dS,$

\subsubsection{Формула Стокса}
${\int \limits _{\Gamma ^{+}}{P\;dx+Q\;dy+R\;dz}} = \iint \limits _{S^{+}}{det \begin{vmatrix}\rm{cos}\alpha &\rm{cos}\beta&\rm{cos}\gamma\\ \frac {\partial}{\partial x}&\frac {\partial}{\partial y}&\frac {\partial}{\partial z}\\P&Q&R\end{vmatrix}}dudv, \quad \int \limits _{\Sigma }{\mathrm  {rot}}\,{\mathbf  {F}}\cdot d{\mathbf  {\Sigma }}=\int \limits _{{\partial \Sigma }}{\mathbf  {F}}\cdot d{\mathbf  {r}}$

\section{Преобразование координат}

$J = \det {\begin{pmatrix}{\pd{u_{1}}{x_{1}}}(x)&{\pd{u_{1}}{x_{2}}}(x)\\{\pd{u_{2}}{x_{1}}}(x)&{\pd{u_{2}}{x_{2}}}(x)\end{pmatrix}}$

\subsubsection{Сферические координаты} 

${\begin{cases}x=r\sin \theta \cos \varphi ,\\y=r\sin \theta \sin \varphi ,\\z=r\cos \theta .\end{cases}} 0 \le \theta \le \pi, \quad 0 \le \varphi \le 2\pi, \quad J=r^{2}\sin \theta $

\subsubsection{Цилиндрические координаты}
${\begin{cases}x=\rho \cos \varphi ,\\y=\rho \sin \varphi ,\\z=z.\end{cases}} \quad 0 \le \varphi \le 2\pi, \quad J = \rho$


\section{Векторное поле}

\begin{itemize}
	\item
	${\mathrm  {grad}}\,\varphi =\nabla \varphi ={\frac  {\partial \varphi }{\partial x}}{\vec  e}_{x}+{\frac  {\partial \varphi }{\partial y}}{\vec  e}_{y}+{\frac  {\partial \varphi }{\partial z}}{\vec  e}_{z}.$
	
	\item
	$\operatorname {div}\,{\mathbf  {F}}= (\nabla , {\mathbf  {F}})= {\frac  {\partial F_{x}}{\partial x}}+{\frac  {\partial F_{y}}{\partial y}}+{\pd{F_{z}}{z}} = \lim _{{V\rightarrow 0}}{\frac{\iint \limits _{S}\!\!\!\!\!\!\!\!\!\!\subset \!\supset \;({\vec  F},d{\vec  S})}{V}}$
	
	\item
	$\operatorname{rot}\; \mathbf{F} = [\mathbf{\nabla} , \mathbf{F}] = \begin{vmatrix} \mathbf{e}_x & \mathbf{e}_y & \mathbf{e}_z \\  
	\frac{\partial}{\partial x} & \frac{\partial}{\partial y} & \frac{\partial}{\partial z} 
	\\  F_x & F_y & F_z 
	\end{vmatrix} 
	= \lim_{\Delta S\to 0}\frac{\oint\limits_{L}\mathbf{ a\cdot , dr}}{\Delta S}$
	
\end{itemize}

\section{Поверхности вращения}

$S=2\pi \int \limits _{a}^{b}f(x){\sqrt  {1+\left(f'(x)\right)^{2}}}dx, \quad y=f(x),\ a\leq x\leq b$ вокруг Ox

$S=2\pi \int \limits _{\alpha }^{\beta }y(t){\sqrt  {\left(x'(t)\right)^{2}+\left(y'(t)\right)^{2}}}dt, \quad x=x(t),\ y=y(t),\ \alpha \leq t\leq \beta  $

$S=2\pi \int \limits _{\alpha }^{\beta }\rho (\varphi )|\sin \varphi |{\sqrt  {\left(\rho (\varphi )\right)^{2}+\left(\rho '(\varphi )\right)^{2}}}d\varphi, \quad r=\rho (\varphi ),\ \alpha \leq \varphi \leq \beta $

$V=\pi \int \limits _{a}^{b}f^{2}(x)dx$


\section{Ряды Фурье}

$f(x)=\frac{a_0}{2} + \sum^{\infty}_{n=1} (a_n \cos \frac{\pi n x}{l} + b_n \sin \frac{\pi n x}{l})$

$a_0= \frac{1}{l}\int\limits_{-l}^{l}f(x)dx, \quad a_n= \frac{1}{l}\int\limits_{-l}^{l}f(x)\cos(nx)dx, \quad b_n= \frac{1}{l}\int\limits_{-l}^{l}f(x)\sin(nx)dx,$

2n - четность относительно $\frac {\pi}{2}$ такая же, как относительно 0

\subsubsection{Равенство Парсеваля} 
$\frac {a_0^2}{2} + \sum _{{n=1}}^{\infty }(a_n^2 +b_n^2)={\frac  {1}{\pi }}\int _{{-\pi }}^{\pi }|f(x)|^{2}\,dx,$

\section{Признаки Даламбера и Коши}

\section{3.9изКудрявцева}

\section{Криволинейные интегралы}
\subsubsection{первого}
$\int\limits_{\Gamma} F\, ds = \int\limits_{\alpha}^{\beta} F(x(t),y(t),z(t))\sqrt{(x'(t))^2+(y'(t))^2+(z'(t))^2}\,dt$
\subsubsection{второго}
$\int\limits_{\Gamma} \left(P\,dx+Q\,dy+R\,dz=\int\limits_{\alpha}^{\beta} P x'(t)+ Q y'(t)+ R z'(t)\right)\,dt$
\subsubsection{грина}


\section{Поверхностные интегралы}
\subsubsection{определение}
\subsubsection{гауссостроградский}
\subsubsection{стокса}

\section{Теория поля}

\section{Все для Фурье}


\section{Вычеты}
\noindent Определение (смотри точные формулировки в соответствующих параграфах):
\begin{itemize}
\item
$\res_a f=\frac{1}{2\pi i} \int_{\Gamma_{\rho}^{+}} f(\xi)\, d\xi,\ a\in\bbC.$
\item
$\res_a f=\frac{1}{2\pi i} \int_{\Gamma_{\rho}^{-}} f(\xi)\, d\xi,\ a=\infty.$
\item
$\sum _{k=1}^{n}\mathop {\mathrm {res} } \limits _{z=a_{k}}f(z) + \mathop {\mathrm {res}} \limits _{\infty } f(z) = 0 $
\end{itemize}

\subsubsection{Формулы для подсчета вычета в конечной точке $a \in \bbC$.}
\begin{itemize}
\item
Общая формула с использованием ряда Лорана:
$$\res_a f = c_{-1}.$$
\item
Если $a$ "--- устранимая особая точка, то
$$\res_a f=0.$$
\item 
Если $a$ "--- полюс первого порядка, то
$$\res_a f=\lim_{z\to a}{f(z)\cdot(z-a)}.$$
\item
Если $a$ "--- полюс первого порядка, причем $f=\frac{g(z)}{h(z)}$, где $g(z),\ h(z)$ "--- регулярные в точке $a$, $g(a)\neq 0,\ h(a)=0. \ h'(a)\neq 0,$ то 
$$\res_a f=\frac{g(a)}{h'(a)}.$$
\item
Если $a$ "--- все еще полюс первого порядка и эта $h(z)=z-a$, т.е. $f=\frac{g(z)}{z-a},$ то
$$\res_a f=g(a).$$
\item
Если $a$ "--- полюс порядка $n\in\bbN$, то
$$
\res_a f = \frac{1}{(n-1)!}\cdot \lim\limits_{z\to a} \frac{d^{n-1}}{dz^{n-1}}\left(f(z)(z-a)^n\right).
$$
\item
Если $a$ "--- полюс порядка $n\in\bbN$, то, в частности, если $f(z)=\frac{h(z)}{(z-a)^m},$ где $h(z)$ "--- регулярная в точке $a$, то 
$$\res_a f(z) = \frac{1}{(m-1)!} h^{(m-1)}(a),$$
т.е. вычет функции $f(z)$ в точке $a$ равен коэффициенту при $(z-a)^{m-1}$ ряда Тейлора $h(z)=\sum\limits_{n=0}^{\infty} c_n (z-a)^n$

\item
Если  $a$ "--- существенно особая точка, то, подчеркну, что 
$$\res_a f = c_{-1}.$$
\end{itemize}

\subsubsection{Формулы для подсчета вычета в бесконечности}
\begin{itemize}
\item
Общая формула с использованием ряда Лорана:
$$\res_{\infty} f = -c_{-1}.$$
\item
Если $a$ "--- устранимая особая точка, то
$$\res_{\infty} f= \lim\limits_{z\to \infty} \left( f(\infty)-f(z)\right)\cdot z,\ \text{где}\ f(\infty)=\lim\limits_{z\to \infty} f(z) .$$
\end{itemize}

\section{Формулы из комбинаторики}

\begin{longtable}[l]{l}
$A^k_n=\frac{n!}{(n-k)!}=n(n-1)(n-2)\dots(n-k+1)$ "--- размещения;
\\
$C^k_n=\binom{n}{k}=\frac{n!}{k!\,(n-k)!}=\frac{n(n-1)(n-2)\dots(n-k+1)}{k!}$ "--- сочетания;
\\
$\binom{n}{k}=\binom{n}{n-k};\qquad \binom{n+1}{k+1}=\binom{n}{k+1}+\binom{k}{n}; $
\\
$\sum^{k}_n \binom{n}{k}=2^n \quad \sum^{k}_n (-1)^k \binom{n}{k}=0.$
\end{longtable}

\section{Суммирование}
\begin{longtable}[l]{l l}
{\normalfont\small\sffamily\bfseries Арифметическая прогрессия}
&
{\normalfont\small\sffamily\bfseries Геометрическая прогрессия}
\\
$a_{n+1}=a_n+d;$ & $b_n=b_1 q^{n-1};$
\\
$a_n=a_1+(n-1)d;$ & $S_n=b_1\frac{1-q^n}{1-q}=\frac{b_n q-b_1}{q-1}, q\ne 1$;
\\ 
$S_n=\frac{a_1+a_n}{2}\cdot n=\frac{2a_1+d(n-1)}{2}\cdot n;$ & $b^2_n=b_{n-1}b_{n+1}, n\geq2,$
\\
$a_n=\frac{a_{n-1}+a_{n+1}}{2}, n\geq 2;$ & $b_kb_{n-k+1}=b_1b_n,\ k\in\overline{1,n};$
\\
$a_k+a_{n-k}=a_1+a_n,\ k\in\overline{1,n}.$ & Если $|q|<1$, то $S=\frac{b_1}{1-q}$
\end{longtable}

\subsubsection{Бином Ньютона}
\begin{longtable}[l]{l}
$(a+b)^n=\sum^{k}_n C^k_n a^k b^{n-k};$
\end{longtable}

\section{Некоторые замечательные пределы.}
\begin{longtable}[l]{l}
$\lim\limits_{x\to0}\frac{\sin x}{x}=1;$
\\
$\lim\limits_{x\to0} (1+x)^{1/x}=e.$
\end{longtable}

\section{Асимптоты графиков}
\textbullet \quad
Прямая $x=x_0$ "--- вертикальная асимптота графика $f(x)$, если $$\lim\limits_{x\to x_0-0}f(x)=\infty\ \text{или}\ \lim\limits_{x\to x_0+0}f(x)=\infty.$$ 

\textbullet \quad
Прямая $y=kx+b$ "--- наклонная асимптота графика $f(x), x\to\pm\infty$, если $$\lim\limits_{x\to\pm\infty}\frac{f(x)}{x}=k; \quad \lim\limits_{x\to\pm\infty} (f(x)-kx)=b.$$

\textbullet \quad
Прямая $y=b$ "--- горизонтальная асимптота графика $f(x), x\to\pm\infty$ (случай $k=0$), если $$ \lim\limits_{x\to\pm\infty} f(x)=b.$$


\section{Дифференциальные уравнения}
\subsubsection {Точки равновесия}

\begin{itemize}
	\item
	Узел: $\lambda _1 \lambda _2 > 0, \quad \rm{Im}\lambda _1 = \rm{Im}\lambda _2 = 0$ (на графике касается прямой с меньшим |$\lambda$|)
	
	\begin{itemize}
		\item Устойчивый $\lambda _{1,2} < 0$)
		
		\item Дикритический $\lambda _1 = \lambda _2$ (кратность 2)
		
		\item Вырожденный $\lambda _1 = \lambda _2$ (кратность 1)
	\end{itemize}
	
	\item
	Седло: $\lambda _1 \lambda _2 < 0, \quad \rm{Im}\lambda _1 = \rm{Im}\lambda _2 = 0$ 
	
	\item
	Фокус: $\rm{Re}\lambda _1 = \rm{Re}\lambda _2 \ne 0$ 
	
	\item
	Центр: $\rm{Re}\lambda _1 = \rm{Re}\lambda _2 = 0$ 
\end{itemize}

\newpage

\section[Площади различных фигур]{Площади различных фигур\footnote{Добавил сюда, только потому что на ГОСе забыл формулу площади эллипса :-)}}
\scriptsize
\begin{longtable}[l]{|A{0.21}{0}|B{0.34}{0}|A{0.45}{0}|}
\hline
Фигура & Формула & Переменные 
\endfirsthead\hline
Фигура & Формула & Переменные 
\endhead\hline
\multicolumn{3}{|c|}{Многоугольники}\tabularnewline
\hline
Правильный треугольник & $a^2\frac{\sqrt{3}}{4}$ & $a$ "--- длина стороны треугольника \tabularnewline\hline
Прямоугольный треугольник & $\frac{ab}{2}$ & $a$ и $b$ "--- катеты треугольника 
\tabularnewline\hline
\multirow{8}{2.5cm}{Произвольный треугольник} & $\frac{1}{2}ah$
&
$a$ "--- сторона треугольника, $h$ "--- высота, проведённая к этой стороне 
\tabularnewline\cline{2-3}
& $\frac{1}{2}ab\sin\alpha$ & $a$ и $b$ "--- любые две стороны, $\alpha$ "--- угол между ними
\tabularnewline\cline{2-3}
&$\sqrt{p(p-a)(p-b)(p-c)}$ \newline (формула Герона) &  $a$, $b$ и $c$ "--- стороны треугольника, $p$ "--- полупериметр $\left(p=\frac{a+b+c}{2}\right)$;
\tabularnewline\cline{2-3}
&$|\frac{1}{2}\begin{vmatrix}x_0&y_0&1\\[-5pt]x_1&y_1&1\\[-5pt]x_2&y_2&1\end{vmatrix}|$ & $(x_0;y_0)$, $(x_1;y_1)$, $(x_2;y_2)$ "--- координаты вершин треугольника (обратите внимание на знак модуля).
\tabularnewline\hline
Квадрат & $a^2$ & $a$ "--- длина стороны квадрата 
\tabularnewline\hline
Прямоугольник & $ab$ & $a$ и $b$ "--- длины сторон прямоугольника (его длина и ширина) 
\tabularnewline\hline
Ромб & $\frac{1}{2}cd$ &  $c$ и $d$ "--- длины диагоналей ромба 
\tabularnewline\hline
Параллелограмм & $ah$ или $ab\sin\alpha$ & $a$ и $h$ "--- длины стороны и опущенной на неё высоты соответственно, $b$ "--- соседняя к $a$ сторона, $\alpha$ "--- угол между ними
\tabularnewline\hline
Трапеция &  $\frac{1}{2}(a+b)h$ & $a$ и $b$ "--- основания трапеции, $h$ "--- высота трапеции 
\tabularnewline\hline
Произвольный четырёхугольник &  $\sqrt{(p-a)(p-b)(p-c)\cdot}$\newline$\overline{\cdot(p-d)-abcd\cos\alpha}$ \newline (формула Брахмагупты) & $a$, $b$, $c$, $d$ "--- стороны четырёхугольника, $p$ "--- его полупериметр, $\alpha$ "--- полусумма противолежащих углов четырёхугольника 
\tabularnewline\hline
Правильный шестиугольник & $a^2\frac{3\sqrt{3}}{2}$ & $a$ "--- длина стороны шестиугольника
\tabularnewline\hline
Правильный восьмиугольник & $2a^2(1+\sqrt{2})$ & $a$ "--- длина стороны восьмиугольника
\tabularnewline\hline
Правильный многоугольник & $\frac{P^2/n}{4\operatorname{tg}(\pi/n)}$ & $P$ "--- периметр, $n$ "--- количество сторон
\tabularnewline\hline
Произвольный многоугольник & $\frac{1}{2}\left|\sum^{n}_{i=1}(x_{i+1}-x_i)(y_{i+1}+y_i)\right|$ (метод трапеций) & $(x_i;y_i)$ "--- координаты вершин многоугольника в порядке их обхода, замыкая последнюю с первой: $(x_{n+1};y_{n+1})=(x_1;y_1)$; при наличии отверстий направление их обхода противоположно обходу внешней границы многоугольника
\tabularnewline\hline
\multicolumn{3}{|c|}{Площади круга, его частей, описанных и вписанных в круг фигур}\tabularnewline\hline
Круг &  $\pi r^2$ или $\frac{\pi d^2}{4}$ & $r$ "--- радиус, $d$ "--- диаметр круга
\tabularnewline\hline
Сектор круга & $\frac{\alpha r^2}{2}$ & $r$ "--- радиус круга, $\alpha$ "--- центральный угол сектора (в радианах)
\tabularnewline\hline
Сегмент круга & $\frac{r^2}{2}(\alpha-\sin\alpha)$ & $r$ "--- радиус круга, $\alpha$ "--- центральный угол сегмента (в радианах)
\tabularnewline\hline
Эллипс & $\pi ab$ & $a$, $b$ "--- большая и малая полуоси эллипса
\tabularnewline\hline
Треугольник, вписанный в окружность & $\frac{abc}{4R}$ & $a$, $b$ и $c$ "--- стороны треугольника, $R$ "--- радиус описанной окружности
\tabularnewline\hline
Четырёхугольник, вписанный в окружность & $\sqrt{(p-a)(p-b)(p-c)(p-d)}$ \newline(формула Брахмагупты) & $a$, $b$, $c$, $d$ "--- стороны четырёхугольника, $p$ "--- его полупериметр
\tabularnewline\hline
Многоугольник, описанный около окружности & $\frac{1}{2}Pr$ & $r$ "--- радиус окружности, вписанной в многоугольник, $P$ "--- периметр многоугольника
\tabularnewline\hline
Прямоугольная трапеция, описанная около окружности & $ab$ & $a$, $b$ "--- основания трапеции
\tabularnewline\hline
\multicolumn{3}{|c|}{Площади поверхностей тел в пространстве}\tabularnewline \hline
Полная поверхность цилиндра & $2\pi r(r+h)$ & $r$ и $h$ "--- радиус и высота соответственно
\tabularnewline\hline
Боковая поверхность цилиндра & $2\pi rh$ & $r$ и $h$ "--- радиус и высота соответственно
\tabularnewline\hline
Полная поверхность конуса & $\pi r (l + r)$ & $r$ и $l$ "--- радиус и образующая боковой поверхности соответственно
\tabularnewline\hline
Боковая поверхность конуса & $\pi rl$ & $r$ и $l$ "--- радиус и образующая боковой поверхности соответственно
\tabularnewline\hline
Поверхность сферы (шара) & $4\pi r^2$ или $\pi d^2$ & $r$ и $d$ "--- радиус и диаметр соответственно
\tabularnewline\hline
\end{longtable}

%%%AFTER FORMULAS COMMANDS
\footnotesize
\setlength{\parindent}{0.6cm}