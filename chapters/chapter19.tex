\chapter{Непрерывность преобразования Фурье абсолютно интегрируемой функции. Преобразование Фурье производной и производная преобразования Фурье.}

\section{Непрерывность преобразования Фурье абсолютно интегрируемой функции}

Аналогом ряда Фурье в данном вопросе будем называть интеграл
$$
\int_{0}^{+\infty} (a(y)\cos(yx) + b(y)\sin(yx))\,dy,\, \text{где}
$$
$$
a(y) = \lim_{l \to +\infty}\frac{1}{\pi}\int_{-l}^{l} f(t)\cos(yt)\,dt;\qquad 
b(y) = \lim_{l \to +\infty}\frac{1}{\pi}\int_{-l}^{l}. f(t)\sin(yt)\,dt
$$
который называется \textit{интегралом Фурье}.

Заметим, что не для всякой функции $f$, которая абсолютно интегрируема на любом конечном интервале, определенные выше пределы существуют. Если же функция $f$ абсолютно интегрируема на $\bbR$, то эти пределы заведомо существуют и
\begin{equation} \label{ch19eq1}
a(f;y) = \frac{1}{\pi}\int_{-\infty}^{+\infty} f(t)\cos(yt)\,dt;
\qquad
b(f;y) = \frac{1}{\pi}\int_{-\infty}^{+\infty} f(t)\sin(yt)\,dt.
\end{equation}

\begin{thm}
Если функция $f$ определена и абсолютно интегрируема на $\bbR$, то функции $a(f)$ и $b(f)$ определены и ограничены на $\bbR$, причем
\begin{equation} \label{ch19eq3}
||a(f;y)||_C \le \frac{1}{\pi}||f||_{L_{1}}, \quad ||b(f;y)||_C \le \frac{1}{\pi}||f||_{L_{1}}.
\end{equation}
Кроме того, они непрерывны на $\bbR$ и
\begin{equation} \label{ch19eq4}
\lim_{y \to \pm\infty} a(f;y) = \lim_{y \to \pm\infty} b(f;y) = 0.
\end{equation}
\end{thm}
\begin{proof}
Из неравенств
$$
|f(t)\cos(yt)| \le |f(t)|, \quad |f(t)\sin(yt)| \le |f(t)|.
$$
и абсолютной интегрируемости функции~$f$ на~$\bbR$ следует, что интегралы~\eqref{ch19eq1} сходятся равномерно на~$\bbR$ относительно $y$, и поэтому функции~$a(f)$ и $b(f)$ непрерывны на $\bbR$.

Неравенства \eqref{ch19eq3} очевидны, а соотношения \eqref{ch19eq4} следуют из теоремы Римана об осцилляции. 
\end{proof}

\begin{defn}
Определенная на $\bbR$ функция называется \textit{локально интегрируемой}\rindex{функция!локально интегрируемая}, если она абсолютно интегрируема на любом конечном интервале.
\end{defn}
\begin{defn}
Для любой локально интегрируемой функции $\phi, x \in \bbR$ предел
$$
\lim_{l \to +\infty}\int_{-l}^{l} \phi(x)\,dx
$$
называют \textit{интегралом от $+\infty$ до $-\infty$ в смысле главного значения (или в смысле Коши)} и обозначают
$$
\v.p.\int_{-\infty}^{+\infty} \phi(x)\,dx
$$
\end{defn}

Для локально интегрируемой на $\bbR$ функции $f$ введем функцию
\begin{equation} \label{ch19eq5}
c(f;y)=\v.p.\frac{1}{2\pi}\int_{-\infty}^{+\infty} f(t)e^{-iyt}\,dt
\end{equation}
являющуюся аналогом коэффициентов Фурье в комплексной форме для периодических функций, и через эту функцию выразим интеграл Фурье функции $f$.

Очевидно,
$$
c(f;y) = \frac12 (a(f;y) - ib(f;y)),
$$
где $a(f;y)$ и $b(f;y)$ "--- функции, определенные в формулах \eqref{ch19eq1} и \eqref{ch19eq2}. Тогда, для любого $\eta > 0$ имеем
$$
\int_{-\eta}^{\eta} c(f;y)e^{iyx}\,dy = \frac12 \int_{-\eta}^{\eta} (a(f;y) - ib(f;y))(\cos yx + i\sin yx)\,dy.
$$
А так как функция $a(f;y)$ четная, а функция $b(f;y)$ "--- нечетная, то
$$
\int_{-\eta}^{\eta} c(f;y)e^{iyx}\,dy = \int_{0}^{\eta}(a(f;y)\cos yx + b(f;y)\sin yx)\,dy.
$$
Отсуда в пределе $\eta \to \infty$ получаем, что интеграл Фурье функции $f$ равен интегралу
\begin{equation} \label{ch19eq6}
\v.p. \int_{-\infty}^{\infty} c(f;y)e^{iyx}\,dy.
\end{equation}
Интеграл \eqref{ch19eq6}, где функция $c(f;y)$ определена по формуле \eqref{ch19eq5}, называется \textit{интегралом Фурье в комплексной форме.}

Пусть абсолютно интегрируемая на $\bbR$ функция $f$ в любой точке $x \in \bbR$ непрерывна и удовлетворяет условию Дини или условию Дирихле. Тогда:
$$
\forall x \in \bbR\quad f(x) = \v.p.\int_{-\infty}^{+\infty} c(f;y)e^{iyx}\,dy,\quad c(f;y)=\frac{1}{2\pi}\int_{-\infty}^{+\infty} f(x)e^{-iyx}\,dx
$$
Здесь функция $f$ принимает действительные значения, а функция $c(f;y)$ принимает, вообще говоря, комплексные значения. Причем, в первом равенстве нет множителя перед интегралом, а во втором "--- стоит множитель $\frac{1}{2\pi}$. Обычно используют более симметричные формулы.
\begin{defn}
Для любой локально интегрируемой комплекснозначной функции $f(x), x \in \bbR$ функция
$$
\widehat{f}(\xi) = \v.p.\frac{1}{\sqrt{2\pi}} \int_{-\infty}^{+\infty} f(x)e^{-i\xi x}\,dx
$$
называется \textit{преобразованием (или образом) Фурье функции f}, а функция
$$
\widetilde{f}(\xi) = \v.p.\frac{1}{\sqrt{2\pi}} \int_{-\infty}^{+\infty} f(x)e^{i\xi x}\,dx
$$
называется \textit{обратным преобразованием (или прообразом) Фурье функции f}.
\end{defn}
Если функция $f$ абсолютно интегрируема, то, как было доказано выше, функции $\widehat{f}$ и $\widetilde{f}$ непрерывны и ограничены на $\bbR$, причем
$$
||\widehat{f}||_C \le \frac{1}{\sqrt{2\pi}}||f||_{L_1}, \quad ||\widetilde{f}||_C \le \frac{1}{\sqrt{2\pi}}||f||_{L_1}.
$$
Кроме того,
$$
\lim_{\xi \to \infty} \widehat{f}(\xi) = \lim_{\xi \to \infty} \widetilde{f}(\xi) = 0.
$$

\section{Преобразование Фурье производной и производная преобразования Фурье}

Функцию $f$, определенную на $\bbR$, будем называть \textit{кусочно непрерывной}, если она кусочно непрерывна на любом конечном интервале. Если же она на любом конечном интервале кусочно дифференцируема, то будем говорить, что она \textit{кусочно дифференцируема на $\bbR$}. Аналогично определяются и \textit{кусочно непрерывно дифференцируемые на $\bbR$} функции. 

Заметим, что если функция $f$ непрерывна и кусочно непрерывно дифференцируема, то она является обобщенной первообразной для производной~$f'$.

\begin{lemm}
Пусть функция $f(x)$ непрерывна и кусочно непрерывно дифференцируема на $\bbR$. Тогда, если $f(x)$ и $f'(x)$ абсолютно интегрируемы на $\bbR$, то $f(x)\to 0$ при $x\to\pm\infty$.
\end{lemm}

Действительно, из равенства
$$
f(x)=f(c)+\int_c^x f'(t)\,dt,\quad c\in\bbR,
$$
и сходимости интеграла $f'(x)$ на $\bbR$ следует, что пределы у $f(x)$ при $x\to\pm\infty$ существуют, а из сходимости интеграла от $f(x)$ на $\bbR$ следует, что эти пределы равны нулю.

\begin{thm}
Пусть функция $f(x)$ непрерывна и кусочно непрерывно дифференцируема на $\bbR$. Тогда, если $f(x)$ и $f'(x)$ абсолютно интегрируемы на $\bbR$, то
$$
F[f']=i\xi\widehat{f}(\xi),\quad F^{-1}[f']=-i\xi \widetilde{f}(\xi).
$$
\end{thm}

\begin{proof}
По формуле интегрирования по частям получаем
\begin{multline*}
F[f']=\frac{1}{\sqrt{2\pi}}\int_{-\infty}^{+\infty} f'(x)e^{-i\xi x}\,dx=\\
={\frac{1}{\sqrt{2\pi}}f(x)e^{-i\xi x}}	
\bigg|^{-\infty}_{+\infty} - \frac{1}{\sqrt{2\pi}}\int_{-\infty}^{+\infty}f(x)(-i\xi)e^{-i\xi x}\,dx.
\end{multline*}
В силу леммы, внеинтегральные члены равны нулю, поэтому
$$
F[f']=i\xi\widehat{f}(\xi).
$$
Аналогично доказывается и вторая формула.
\end{proof}

\begin{cons}
Если $f,f',...,f^{(n)}$ непрерывны и абсолютно интегрируемы на $\bbR$, то
$$
F[f^{(k)}]=(i\xi)^kF[f],
$$
$$
F^{-1}[f^{(k)}]=(-i\xi)^kF^{-1}[f], \quad k=0,1,...,n.
$$
В частности
$$
\widehat{f}(\xi)=o\left(\frac{1}{\xi^n}\right),\quad \widetilde{f}(\xi)=o\left(\frac{1}{\xi^n}\right)
$$
при $\xi\to\pm\infty$.
\end{cons}

\begin{thm}
Если функции $f(x)$ и  $xf(x)$ абсолютно интегрируемы на $\bbR$, то $\widehat{f}(\xi)$ и $\widetilde{f}(\xi)$ непрерывно дифференцируемы на $\bbR$ и 
$$
\frac{d\widehat{f}}{d\xi}=-iF[xf(x)],\quad \frac{d\widetilde{f}}{d\xi}=iF^{-1}[xf(x)].
$$
\end{thm}

\begin{proof}
По признаку Вейерштрасса интегралы
$$
\int_{-\infty}^{+\infty}f(x)e^{-i\xi x}\,dx\quad \text{и} -i\int_{-\infty}^{+\infty}xf(x)e^{-i\xi x}\,dx
$$
сходятся равномерно по $\xi$ на $\bbR$, поэтому они непрерывны и производная по $\xi$ от первого из них равна второму. Следовательно,
$$
\frac{d\widehat{f}}{d\xi}=-iF[xf(x)].
$$

Аналогично доказывается и вторая формула.
\end{proof}

\begin{cons}
Если функции $f(x)$, $xf(x)$, $\dots\,$, $x^nf(x)$ абсолютно интегрируемы на $\bbR$, то $\widehat{f}(\xi)$ и $\widetilde{f}(\xi)$ $n$ раз непрерывно дифференцируемы на $\bbR$ и 
$$
\frac{d^k\widehat{f}}{d\xi^k}=(-i)^kF[x^kf(x)],\quad \frac{d^k\widetilde{f}}{d\xi^k} = i^kF^{-1}[x^kf(x)],\quad k=0,1,...,n.
$$
\end{cons}