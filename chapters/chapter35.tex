\chapter{Разложение функции, регулярной в кольце, в ряд Лорана. Изолированные особые точки однозначного характера.}

\section{Разложение функции, регулярной в кольце, в ряд Лорана}

\begin{leftbar}
\begin{defn}
Рядом Лорана с центром в точке $a \in \bbC$ называется ряд вида
\begin{equation} \label{ch35.1eq1}
\sum\limits_{n = -\infty}^{+\infty} c_n (z - a)^n,
\end{equation}
понимаемый как сумма двух рядов:
\begin{equation} \label{ch35.1eq2}
\sum\limits_{n = 0}^{+\infty} c_n (z - a)^n
\end{equation}
и
\begin{equation} \label{ch35.1eq3}
\sum\limits_{n = -\infty}^{-1} c_n (z - a)^n = \sum\limits_{m = 1}^{+\infty} c_{-m} (z - a)^{-m}.
\end{equation}
\end{defn}


Ряд $\eqref{ch35.1eq2}$ является обычным степенным рядом и в силу теоремы \hyperref[ch34.2Thm1]{Абеля} (билет №34) областью его сходимости является некоторый круг $B_R(a)$, где $R$ "--- радиус сходимости ряда $\eqref{ch35.1eq2}$. Ряд $\eqref{ch35.1eq3}$ заменой $\frac{1}{z - a} = \zeta$ приводится к степенному ряду $\sum\limits_{m = 1}^{+\infty}c_{-m} \zeta^m$, и по той же теореме \hyperref[ch34.2Thm1]{Абеля} его область сходимости "--- тоже некоторый круг $|\zeta| < \alpha_0$. Следовательно, ряд $\eqref{ch35.1eq3}$ сходится в области $|z - a| > \frac{1}{\alpha_0} \triangleq \rho \ge 0$. Если $\rho > R$, то суммарный ряд $\eqref{ch35.1eq1}$ не сходится ни в одной точке, если же $\rho < R$, то ряд $\eqref{ch35.1eq1}$ сходится в кольце: $\rho < |z - a| < R$.

В последнем случае кольцо $\rho < |z - a| < R$, где $R$ "--- радиус сходимости ряда $\eqref{ch35.1eq2}$, а $\frac{1}{\rho}$ "--- радиус сходимости ряда $\sum\limits_{m = 1}^{+\infty} c_{-m} \zeta^m$ называется \textit{кольцом сходимости ряда Лорана} $\eqref{ch35.1eq1}$.
\end{leftbar}

По теореме \hyperref[ch34.2Thm1]{Абеля} ряд $\eqref{ch35.1eq2}$ сходится равномерно в $\overline{B_{R_1}(a)}$ при 
$R_1 \in (0,R)$, а ряд $\eqref{ch35.1eq3}$ сходится равномерно на множестве $|z - a| \ge \rho_1$ при $\rho_1 > \rho$. Следовательно, ряд Лорана сходится равномерно в любом кольце
$$
\rho_1 \le |z - a| \le R_1, \quad \rho < \rho_1 < R_1 < R.
$$

\begin{defn}\label{ch35.1defn2} 
Функциональный ряд 
\begin{equation} \label{ch35.1eq17}
S(z)=\sum\limits_{n=1}^{+\infty} f_n(z), \quad z\in G
\end{equation}
 сходится \textit{равномерно строго внутри области $G$}, если он сходится равномерно в каждом замкнутом круге $\overline{B_r(z)}$, $r>0$, содержащемся в области $G$.
\end{defn}

\begin{thm}[Вейерштрасса] \label{ch35.1Thm5}
Пусть функциональный ряд $\eqref{ch35.1eq17}$, составленный из регулярных функций $f_n: G \to \bbC$, сходится равномерно строго внутри области $G$. Тогда
\begin{enumerate}
	\item[1)] {\rightskip=3.5cm} {сумма $S(z)$ ряда $\eqref{ch35.1eq17}$ есть тоже регулярная функция на $G$ (Первая теорема Вейерштрасса)
	}
	\item[2)] {\rightskip=0cm}{ряд $\eqref{ch35.1eq17}$ можно почленно дифференцировать любое число раз, т.е. для $\forall k \in \bbN$ имеет место формула 
	\begin{equation} \label{ch35.1eq18}
	S^{(k)}(z) = \sum\limits_{n=1}^{+\infty} f_{n}^{(k)}(z), \quad z \in G
	\end{equation}
	причем каждый ряд $\eqref{ch35.1eq18}$ сходится равномерно строго внутри области $G$ (Вторая теорема Вейерштрасса).	}
\end{enumerate}
\end{thm}

Таким образом, по определению \ref{ch35.1defn2} ряд Лорана $\eqref{ch35.1eq1}$ сходится \textit{равномерно строго внутри} его кольца сходимости. Так как к тому же каждый член ряда $\eqref{ch35.1eq1}$ в кольце сходимости является регулярной функцией, то по  \hyperref[ch35.1Thm5]{теореме 1 (Вейерштрасса)} сумма ряда Лорана в кольце сходимости также является регулярной функцией, причем ряд Лорана в этом кольце можно почленно дифференцировать любое число раз.

\begin{thm} \label{ch35.1Thm6}
Пусть в области $G$ заданы непрерывные функции $f_n: G \to \bbC,\quad n \in \bbN$ и кусочно-гладкий контур $\gamma$. Пусть функциональный ряд $\sum\limits_{n=0}^{+\infty} f_n(z)$ сходится к своей сумме $S(z)$ равномерно на контуре $\gamma$. Тогда этот ряд можно почленно интегрировать по контуру $\gamma$, т.е. справедливо равенство
\begin{equation}
\int_{\gamma} S(z)dz = \sum\limits_{n=0}^{+\infty} \int_{\gamma} f_n(z)dz
\end{equation}
\end{thm} 

Имеет место и обратное утверждение, а именно, 

\begin{leftbar}
\begin{thm}[Лорана-Вейерштрасса] \label{ch35.1Thm1}
Всякая функция $\omega = f(z)$, регулярная в кольце $\rho < |z - a| < R$, где $0 \le \rho < R \le +\infty$, представима в этом кольце суммой сходящегося ряда Лорана
$$
f(z) = \sum\limits_{n = -\infty}^{+\infty} c_n(z - a)^n,
$$
коэффициенты которого определяются по формулам
\begin{equation} \label{ch35.1eq4}
c_n = \frac{1}{2\pi i} \int_{\gamma_r} \frac{f(\zeta)}{(\zeta - a)^{n + 1}} \,d\zeta, \quad r \in (\rho, R), \quad n \in \bbZ,
\end{equation}
причем ориентация окружности $\gamma_r \triangleq \{\zeta \: \big| \: |\zeta - a| = r\}$ положительная, т.е. обход производится против хода часовой стрелки.

\end{thm}

\begin{proof}


\begin{enumerate}
	
\item
Покажем, что каждый коэффициент $c_n$ в формуле $\eqref{ch35.1eq4}$ не зависит от выбора $r \in (\rho, R)$ . Функция $\frac{f(\zeta)}{(\zeta - a)^{n+1}}$ регулярна в кольце $\rho < |\zeta - a| < R$. Для любых чисел $r_1, r_2: \rho < r_1 < r_2 < R$ определим окружности $\gamma_k$ с центром в точке $a$ и радиуса $r_k, k \in \overline{1,2}$, ориентированные положительно. По \hyperref[abc28]{обобщенной теореме Коши} получаем равенство
\begin{equation*}
\begin{split}
\int_{\gamma_2 \cup \gamma_{1}^{-1}} \frac{f(\zeta)}{(\zeta - a)^{n + 1}} \,d\zeta = 0, \text{т.е.}\\
\int_{\gamma_2} \frac{f(\zeta)}{(\zeta - a)^{n + 1}} \,d\zeta = \int_{\gamma_1} \frac{f(\zeta)}{(\zeta - a)^{n + 1}} \,d\zeta,
\end{split}
\end{equation*}
что и требовалось для доказательства независимости интеграла $\eqref{ch35.1eq4}$ от выбора $r \in (\rho, R)$ при каждом $n \in \bbZ$.

\item

Зафиксируем произвольную точку $z_0$ в кольце $\rho < |z - a| < R$. Выберем числа $r_1, r_2$ такие, что 
$\rho < r_1 < |z_0 - a| < r_2 < R$, и окружности $\gamma_k = \{ z \: \big| \: |z - a| = r_k\}$ при $k \in \overline{1,2}$, ориентированные положительно. Тогда контур $\Gamma = \gamma_2 \cup \gamma_{1}^{-1}$, является границей кольца $r_1 < |z - a| < r_2$, в котором по  \hyperref[ch34thm1]{интегральной формуле Коши} получаем
\begin{equation} \label{ch35.1eq5}
f(z_0) = \frac{1}{2\pi i} \int_\Gamma \frac{f(\zeta)}{\zeta - z_0} \,d\zeta = \frac{1}{2\pi i} \int_{\gamma_2} \frac{f(\zeta)}{\zeta - z_0} \,d\zeta - \frac{1}{2\pi i} \int_{\gamma_1} \frac{f(\zeta)}{\zeta - z_0} \,d\zeta \triangleq I_2 + I_1.
\end{equation}

Рассмотрим интеграл $I_2$ из равенства $\eqref{ch35.1eq5}$. Преобразовывая подынтегральную функцию $I_2$ (так же, как в теореме \ref{abc29} (билет №34)), для всех $\zeta \in \gamma_2$ получаем сумму геометрической прогрессии (см. \hyperref[exmpl2]{пример 2} из билета №34 ) вида
\begin{equation} \label{ch35.1eq6}
\frac{1}{2\pi i} \frac{f(\zeta)}{\zeta - z_0} = \frac{1}{2\pi i} \frac{f(\zeta)}{(\zeta - a) \left( 1 - \frac{z_0 - a}{\zeta - a}\right)} = \frac{1}{2\pi i}\sum\limits_{n = 0}^{+\infty} \frac{(z_0 - a)^n}{(\zeta - a)^{n + 1}} f(\zeta).
\end{equation}

Из справедливости оценки
$$
\left| f(\zeta) \frac{(z_0 - a)^n}{(\zeta - a)^{n + 1}}\right| \le q_{2}^n \cdot \frac{M}{r_2}, \quad \forall \zeta \in \gamma_2,
$$
где $q_2 \triangleq \frac{|z_0 - a|}{r_2} < 1, \quad M \triangleq \sup \{ |f(z)| \: \big| \: r_1 \le |z - a| \le r_2\} < +\infty$, и из того, что ряд $\sum\limits_{n = 0}^{+\infty} q_{2}^n$ сходится, по признаку Вейерштрасса получаем, что ряд $\eqref{ch35.1eq6}$ сходится абсолютно и равномерно на $\gamma_2$. По теореме \ref{ch35.1Thm6}  ряд $\eqref{ch35.1eq6}$ можно почленно интегрировать по $\gamma_2$, т.е. получим, что 
\begin{equation} \label{ch35.1eq7}
I_2 = \frac{1}{2\pi i} \int_{\gamma_2} \frac{f(\zeta)}{\zeta - z_0} \,d z \xlongequal{\eqref{ch35.1eq6}} \sum\limits_{n = 0}^{+\infty} \frac{1}{2\pi i} \int_{\gamma_2} \frac{f(\zeta)}{(\zeta - a)^{n + 1}} \,d\zeta \cdot (z_0 - a)^n = \sum\limits_{n = 0}^{+\infty} c_n (z_0 - a)^n,
\end{equation}
где 
\begin{equation} \label{ch35.1eq8}
c_n = \frac{1}{2\pi i} \int_{\gamma_2}  \frac{f(\zeta)}{(\zeta - a)^{n + 1}} \,d\zeta, \quad n = 0,1,2,\ldots
\end{equation}

\item

Рассмотрим интеграл $I_1$ из $\eqref{ch35.1eq5}$. Представим $-\frac{1}{\zeta - z_0}$ в виде ряда (см. \hyperref[exmpl2]{пример 2} из билета №34 )
\begin{equation} \label{ch35.1eq9}
-\frac{1}{\zeta - z_0} = \frac{1}{(z_0 - a) \left( 1 - \frac{\zeta - a}{z_0 - a}\right)} = \sum_{n = 0}^{+\infty} \frac{(\zeta - a)^n}{(z_0 - a)^{n + 1}}.
\end{equation}

По признаку Вейерштрасса ряд $\eqref{ch35.1eq9}$ сходится равномерно по $\zeta$ на $\gamma_1$, так как
$$
\left| \frac{\zeta - a}{z_0 - a}\right| = \frac{r_1}{|z_0 - a|} \triangleq q_1 < 1, \quad \forall \zeta \in \gamma_1.
$$

Так как $|f(\zeta)| \le M$ при $\zeta \in \gamma_1$, то ряд
\begin{equation} \label{ch35.1eq10}
-\frac{1}{2\pi i} \frac{f(\zeta)}{(\zeta - z_0)} = \sum\limits_{n = 0}^{+\infty} \frac{1}{2\pi i} \frac{f(\zeta)(\zeta - a)^n}{(z_0 - a)^{n + 1}}, \quad \zeta \in \gamma_1,
\end{equation}
также сходится равномерно на $\gamma_1$, и аналогично случаю вычисления $I_2$ его можно почленно интегрировать. После интегрирования равенства $\eqref{ch35.1eq10}$ получаем
\begin{equation} \label{ch35.1eq11}
I_1 = \sum\limits_{n = 0}^{+\infty} \left( \frac{1}{2\pi i} \int_{\gamma_1} f(\zeta)(\zeta - a)^n \,d\zeta\right) \frac{1}{(z_0 - a)^{n + 1}}.
\end{equation}

Заменяя в формуле $\eqref{ch35.1eq11}$ номера $(n + 1)$ на $(-m)$, получаем равенство
\begin{equation} \label{ch35.1eq12}
I_1 = \sum\limits_{m = -\infty}^{-1} c_m (z_0 - a)^m,
\end{equation}
где
\begin{equation} \label{ch35.1eq13}
c_m = \frac{1}{2\pi i} \int_{\gamma_1} \frac{f(\zeta)}{(\zeta - a)^{m + 1}} \,d\zeta, \quad m = -1,-2,\ldots
\end{equation}

В силу пункта 1. в формулах $\eqref{ch35.1eq8}$, $\eqref{ch35.1eq13}$ контуры $\gamma_1, \gamma_2$ можно заменить на любую окружность $\gamma_r = \{ z \: \big| \: |z - a| = r\}$, где $\rho < r < R$, т.е.
верна общая формула коэффициентов $\eqref{ch35.1eq4}$. Так как точка $z_0$ была выбрана в данном кольце произвольно, то, складывая ряды $\eqref{ch35.1eq7}$ и $\eqref{ch35.1eq12}$, получаем ряд Лорана с коэффициентами $\eqref{ch35.1eq4}$, сходящийся во всем кольце $\rho < |z - a| < R$.	

\end{enumerate}

\end{proof}
\end{leftbar}

\begin{cons} \label{ch35.1cons1}
Если функция $f : B_R(a) \to \bbC$ регулярна на $B_R(a)$, то ее ряд Лорана с центром в точке а совпадает с ее рядом Тейлора с центром в точке $a$.
\end{cons}
\begin{proof}
В самом деле, по формуле $\eqref{ch35.1eq4}$ при $m = -1,-2, \ldots$ функция $\frac{f(\zeta)}{(\zeta - a)^{m+1}}$ будет регулярной в круге $B_R(a)$, и по теореме Коши интеграл от нее по замкнутому контуру равен нулю, т.е. $c_m = 0 \quad \forall m = -1,-2, \ldots$.
\end{proof}

\begin{thm}[о единственности разложения в ряд Лорана] \label{Ch35.1Thm2}
Разложение регулярной в кольце $\rho < |z - a| < R$ функции $f$ в сходящийся ряд Лорана с центром в точке $a$ единственно.
\end{thm}

\begin{proof}
Пусть регулярная функция $f$ представлена в кольце $\rho < |z - a| < R$ в виде некоторого ряда
\begin{equation} \label{ch35.1eq14}
f(z) = \sum_{n = -\infty}^{-1} b_n (z - a)^n + \sum_{n = 0}^{+\infty} b_n (z - a)^n.
\end{equation}

Выберем число $r \in (\rho, R)$, и пусть окружность $\gamma_r = \{ z \: \big| \: |z - a| = r\}$ ориентирована положительно. Как показано в \hyperref[exmpl1]{примере 1} (билет №34), справедлива формула
\begin{equation} \label{ch35.1eq15}
I_k \triangleq \int_{\gamma_r} \frac{\,dz}{(z - a)^{k + 1}} = 
\begin{cases}
2\pi i, & k = 0, \\
0, & k = \pm1, \pm2, \ldots
\end{cases}
\end{equation}

Как было отмечено в начале параграфа, ряд $\eqref{ch35.1eq14}$ на окружности $\gamma_r$ сходится равномерно. Зафиксируем любое число $k \in \bbZ$. Умножив ряд $\eqref{ch35.1eq14}$ на ограниченную по модулю на кривой $\gamma_r$ функцию $\frac{1}{2\pi i(z - a)^{k + 1}}$, получаем равномерно сходящийся на окружности $\gamma_r$ ряд
\begin{equation} \label{ch35.1eq16}
\frac{1}{2\pi i} \frac{f(z)}{(z - a)^{k + 1}} = \sum_{n = -\infty}^{+\infty} \frac{1}{2\pi i} b_n \frac{(z-a)^n}{(z - a)^{k + 1}}.
\end{equation}

Следовательно, по теореме \ref{ch35.1Thm6} его можно почленно интегрировать по окружности $\gamma_r$, и, учитывая формулу $\eqref{ch35.1eq4}$, получаем
$$
c_k \xlongequal{\eqref{ch35.1eq4}} \frac{1}{2\pi i} \int_{\gamma_r} \frac{f(z)}{(z - a)^{k + 1}} \,dz \xlongequal{\eqref{ch35.1eq16}} \sum_{n = -\infty}^{+\infty} \frac{1}{2\pi i} b_n \int_{\gamma_r} \frac{\,dz}{(z - a)^{k - n + 1}} \xlongequal{\eqref{ch35.1eq15}} b_k,
$$
т.е. ряд $\eqref{ch35.1eq14}$ совпадает с рядом Лорана $\eqref{ch35.1eq1}$, $\eqref{ch35.1eq4}$.
\end{proof}

Из следствия \ref{ch35.1cons1} и теоремы \ref{Ch35.1Thm2} получаем

\begin{cons}
Представление регулярной функции $f : B_R(a) \to \bbC$ в виде сходящегося степенного ряда по степеням $(z - a)$ единственно. Оно совпадает с рядом Тейлора этой функции с центром в точке $a$.
\end{cons}

\begin{cons} [неравенство Коши для коэффициентов ряда Лорана] \label{ch35.1cons3}
Пусть функция $f$ регулярна в кольце $\rho < |z - a| < R$ и на каждой окружности $\gamma_r = \{ z \: \big| \: |z - a| = r\}$ справедлива оценка $|f(z)| \le A_r \: \forall z \in \gamma_r$. Тогда для коэффициентов $\eqref{ch35.1eq4}$ ряда Лорана $\eqref{ch35.1eq1}$ справедлива оценка
\begin{equation} \label{ch35.1eq20}
|c_n| \le \frac{A_r}{r^n}, \quad \forall n \in \bbZ.
\end{equation}

\begin{proof}
Из формулы $\eqref{ch35.1eq4}$ сразу следует
$$
|c_n| = \left| \frac{1}{2\pi i} \int_{\gamma_r} \frac{f(\zeta)\,d\zeta}{(\zeta - a)^{n + 1}} \right| \le \frac{A_r}{2\pi r^{n + 1}} \int_{\gamma_r} |\,d\zeta| = \frac{A_r}{r^n},
$$
что и требовалось доказать.
\end{proof}
\end{cons}

\section{Изолированные особые точки однозначного характера}

\begin{leftbar}
\begin{defn}\label{ch35.1defn3}
Пусть функция $f$ не регулярна в точке $a \in \overline{\bbC}$, но регулярна в некоторой проколотой окрестности этой точки $a$ (т.е. на множестве $\overset{\circ}{B}_\rho(a), \rho > 0$). Тогда точку $a$ называют \textit{изолированной особой точкой (однозначного характера) функции $f$}.
\end{defn}

В определении \ref{ch35.1defn3} точка $a$ может быть как конечной точкой (тогда $\overset{\circ}{B}_\rho(a) = \{ z \: \big| \: 0 < |z - a| < \rho\}$), так и бесконечной (тогда $\overset{\circ}{B}_\rho(\infty) = \{ z \: \big| \: |z| > \rho\}$).

В зависимости от поведения функции $f$ около особой точки будем различать три типа особых точек.

\begin{defn}
Изолированная особая точка $a \in \overline{\bbC}$ функции $f : \overset{\circ}{B}_\rho(a) \to \bbC$ называется
\begin{enumerate}
\item \textit{устранимой особой точкой}, если существует конечный предел $\lim\limits_{z \to a} f(z) \in \bbC$;
\item	\textit{полюсом}, если существует $\lim\limits_{z \to a} f(z) = \infty$;
\item \textit{существенно особой точкой}, если не существует конечного или бесконечного предела $\lim\limits_{z \to a} f(z)$.
\end{enumerate}
\end{defn}
\end{leftbar}

В случае, когда особая точка $a$ конечна, регулярную в кольце $\overset{\circ}{B}_\rho(a)$ функцию $f$ по теореме \ref{ch35.1Thm1} можно представить в виде сходящегося ряда Лорана с центром в точке $a$, т.е.
\begin{equation} \label{ch35.2eq1}
f(z) = \sum_{n = -\infty}^{+\infty} c_n (z - a)^n, \quad z \in \overset{\circ}{B}_\rho(a).
\end{equation}

Тогда будем различать две части ряда Лорана
$$
I_{\text{пр}} \triangleq \sum_{n = 0}^{+\infty} c_n (z - a)^n \quad \text{и} \quad I_{\text{гл}} \triangleq \sum_{n = -\infty}^{-1} c_n (z - a)^n,
$$
которые называют соответственно \textit{правильной и главной частями ряда Лорана} $\eqref{ch35.2eq1}$ с центром в особой точке $a$.

В случае, когда особая точка $a = \infty$, функцию $f$ можно представить в виде сходящегося в кольце $\overset{\circ}{B}_\rho(\infty)$ ряда Лорана
\begin{equation} \label{ch35.2eq2}
f(z) = \sum_{n = -\infty}^{+\infty} c_n z^n, \quad z \in \overset{\circ}{B}_\rho(\infty),
\end{equation}

и теперь будем различать части ряда $\eqref{ch35.2eq2}$
$$
I_{\text{пр}} \triangleq \sum_{n = -\infty}^{0} c_n z^n \quad \text{и} \quad I_{\text{гл}} \triangleq \sum_{n = 1}^{+\infty} c_n z^n,
$$
которые называются соответственно \textit{правильной и главной частями ряда Лорана} $\eqref{ch35.2eq2}$ с центром в $\infty$.

\begin{leftbar}
\begin{thm}\label{Ch35.2Thm5}
Пусть точка $a \in \overline{\bbC}$ есть изолированная особая точка функции $f$. Пусть функция $f$  представлена своим рядом Лорана с центром в точке $a$.

\begin{enumerate}
\item[1)] Для того, чтобы точка $a \in \overline{\bbC}$ была устранимой особой точкой, необходимо и достаточно, чтобы главная часть ряда Лорана отсутствовала (т.е. $I_{\text{гл}}(z) \equiv 0$).
\item[2)]	Чтобы точка $a \in \overline{\bbC}$ была полюсом, необходимо и достаточно, чтобы главная часть ряда Лорана $I_{\text{гл}}(z)$ содержала конечное число ненулевых слагаемых.
\item[3)]	Чтобы точка $a \in \overline{\bbC}$ была существенно особой точкой, необходимо и достаточно, чтобы главная часть ряда Лорана $I_{\text{гл}}(z)$ содержала бесконечное число ненулевых слагаемых.
\end{enumerate}

\end{thm}
\end{leftbar}

\begin{proof}

I. Пусть точка $a \in \bbC$ конечна.

\begin{enumerate}

\item \textit{Необходимость}. Пусть $a$ "--- устранимая особая точка функции $f$, т.е. существует конечный предел $\lim\limits_{z \to a} f(z)$. Тогда функция $f$ ограничена в некоторой окрестности точки $a$, т.е. существуют числа $\rho_1 \in (0, \rho)$ и $A > 0$ такие, что $|f(z)| < A$ при $\forall z \in \overset{\circ}{B}_{\rho_1}(a)$. Воспользуемся\hyperref[ch35.1cons3]{ неравенством Коши для коэффициентов ряда Лорана} функции $f$
$$
|c_n| \le \frac{A}{r^n}, \quad \forall r \in (0, \rho_1).
$$

Для каждого целого $n < 0$ получаем, что $|c_n| \le A \cdot r^{|n|} \to 0$ при $r \to 0$, т.е. $c_n = 0$ для всех $n < 0$, т.е. $I_{\text{гл}}(z) = 0$.

\item \textit{Достаточность}. Пусть $I_{\text{гл}}(z) \equiv 0$, т.е. $c_n = 0 \quad \forall n < 0$. Тогда $f(z) = \sum\limits_{n = 0}^{+\infty} c_n (z - a)^n \triangleq S(z), \quad \forall z \in \overset{\circ}{B}_{\rho}(a)$.

Так как сумма сходящегося степенного ряда $S(z)$ есть регулярная функция на круге $\overset{\circ}{B}_{\rho}(a)$, причем $f(z) = S(z)$ при $z \not= a$, то существует предел
$$
\lim_{z \to a} f(z) = S(a) = c_0.
$$

\item \textit{Необходимость}. Пусть точка $a$ "--- полюс функции $f$, т.е. существует предел $\lim\limits_{z \to a} f(z) = \infty$. В силу этого можно выбрать $\delta > 0$ такое, что при всех $z \in \overset{\circ}{B}_{\delta}(a)$ справедливо неравенство $|f(z)| > 1$. Определим функцию $g(z)\triangleq\frac{1}{f(z)}$ при $z \in \overset{\circ}{B}_{\delta}(a)$.

Очевидно, что функция $g$ регулярна в проколотой окрестности $\overset{\circ}{B}_{\delta}(a)$, причем $g(z) \not= 0$ и $|g(z)| < 1$ при всех $z \in \overset{\circ}{B}_{\delta}(a)$. Так как точка $a$ есть полюс функции $f$, то существует предел $\lim\limits_{z \to a} g(z) = \lim\limits_{z \to a} \frac{1}{f(z)} = 0$, т.е. получаем, что точка $a$ есть устранимая особая точка функции $g$. Следовательно, доопределив $g(a) = 0$, получаем, что функция $g$ регулярна в круге $B_{\delta}(a)$, т.е. по \hyperref[abc29]{теореме 5} (билет №34) она представима в виде сходящегося степенного ряда
\begin{equation} \label{ch35.2eq3}
g(z) = b_m (z - a)^m + b_{m + 1} (z - a)^{m + 1} + \ldots, \quad \forall z \in B_\delta(a).
\end{equation}

Так как функция $g(z) \not\equiv 0$, в равенстве $\eqref{ch35.2eq3}$ существует номер $m \ge 1$, при котором $b_m \not= 0$. Таким образом, $g(z) = (z - a)^{m} h(z)$, где $h(z) = b_m + b_{m + 1} (z - a) + \ldots ,$ т.e. функция $h$ как сумма сходящегося степенного ряда регулярна в круге $B_{\delta}(a)$, причем $h(a) \not= 0$. Поэтому $h(z) \not= 0$ при всех $z$ из некоторой окрестности $B_{\rho_1}(a)$, где $0 < \rho_1 < \delta$. Следовательно, функция $\frac{1}{h(z)}$ тоже регулярна в $B_{\rho_1}(a)$, и  по \hyperref[abc29]{теореме 5} (билет №34) она также представима в виде сходящегося степенного ряда
$$
\frac{1}{h(z)} = d_0 + d_1 (z - a) + d_2 (z - a)^2 + \ldots , \quad z \in B_{\rho_1}(a),
$$

причем здесь $d_0 = \frac{1}{b_m} \not= 0$. В итоге получаем в $\overset{\circ}{B}_{\rho_1}(a)$
\begin{multline} \label{ch35.2eq4}
f(z) = \frac{1}{g(z)} = \frac{1}{(z - a)^m} \cdot \frac{1}{h(z)} = \frac{d_0}{(z - a)^m} + \frac{d_1}{(z - a)^{m - 1}} + \ldots \\ \ldots+ \frac{d_{m - 1}}{(z - a)} + d_m + d_{m + 1} (z - a) + \ldots 
\end{multline}

Таким образом, правая часть в равенстве $\eqref{ch35.2eq4}$ есть ряд Лорана функции $f$ с центром в точке $a$, причем главная часть $I_{\text{гл}}(z)$, очевидно, содержит конечное число ненулевых слагаемых.

\item \textit{Достаточность}. Пусть справедливо представление функции $f$ в проколотой окрестности $\overset{\circ}{B}_{\rho_1}(a)$ в виде сходящегося ряда Лорана $\eqref{ch35.2eq4}$, причем $d_0 \not= 0$. Тогда, вынося общий множитель, получаем
\begin{equation} \label{ch35.2eq5}
f(z) = \frac{1}{(z - a)^m} (d_0 + d_1 (z - a) + \ldots) = \frac{p(z)}{(z - a)^m}.
\end{equation}

В формуле $\eqref{ch35.2eq5}$ функция $p$ как сумма сходящегося степенного ряда регулярна в круге $B_{\rho_1}(a)$ и $\lim\limits_{z \to a} p(z) = p(a) = d_0 \not= 0$. С другой стороны $\frac{1}{(z - a)^m} \to \infty$ при $z \to a$. Отсюда получаем, что $\lim\limits_{z \to a} f(z) = \infty$.

\item	\textit{Эквивалентность} покажем методом исключения. Предел может существовать в $\overline{\bbC}$ или не существовать. У главной части ряда $I_{\text{гл}}(z)$ может быть конечное число слагаемых или бесконечное. Эквивалентность существования предела в $\overline{\bbC}$ и конечности числа ненулевых слагаемых в ряде $I_{\text{гл}}(z)$ уже доказаны в пп. 1) и 2). Следовательно, если не существует предела функции $f$, то это эквивалентно бесконечному числу слагаемых в $I_{\text{гл}}(z)$.

\end{enumerate}

II. Пусть функция $f$ имеет особую точку $a = \infty$. Заменой аргумента $\zeta = \frac{1}{z}$ приходим к функции $\widetilde{f}(\zeta) = f(\frac{1}{\zeta})$, у которой особой точкой является точка $\zeta = 0$, причем существование предела функции $\widetilde{f}(\zeta)$ в точке $\zeta = 0$ эквивалентно существованию предела функции $f(z)$ в $\infty$, т.е. тип особой точки $a = \infty$ у функции $f(z)$ и тип особой точки $\zeta = 0$ у функции $\widetilde{f}(\zeta)$ одинаков. В свою очередь, главная часть ряда Лорана функции $f(z)$ с центром в точке $\infty$ при замене аргумента переходит в главную часть ряда Лорана функции $\widetilde{f}(\zeta)$ с центром в точке $\zeta = 0$. Так как необходимое соответствие в конечной точке $\zeta = 0$ уже установлено в пункте I, то это влечет требуемое соответствие при $a  =\infty$.

\end{proof}

\begin{cons} \label{ch35.2cons1}
Точка $a \in \bbC$ является полюсом функции $f$ тогда и только тогда, когда существуют окрестность $\overset{\circ}{B}_\rho(a)$, натуральное число $m \ge 1$ и регулярная в круге $B_\rho(a)$ функция $p$ такие, что $p(a) \not = 0$ и
\begin{equation} \label{ch35.2eq6}
f(z) = \frac{p(z)}{(z - a)^m}, \quad \forall z \in \overset{\circ}{B}_\rho(a).
\end{equation}

В свою очередь, точка $a = \infty$ является полюсом функции $f$ тогда и только тогда, когда существуют окрестность $\overset{\circ}{B}_\rho(\infty)$, число $m \ge 1$, регулярная в $\overset{\circ}{B}_\rho(\infty)$ функция $h$, у которой существует конечный предел $h(\infty) \not= 0$, такие, что
\begin{equation} \label{ch35.2eq7}
f(z) = z^m h(z), \quad z \in \overset{\circ}{B}_\rho(\infty).
\end{equation}

\end{cons}

\begin{proof}
Очевидно следует из только что доказанной теоремы \ref{Ch35.2Thm5}.
\end{proof}

\begin{defn}
Пусть $a \in \overline{\bbC}$ "--- полюс функции $f$. Тогда число $m$ в формулах $\eqref{ch35.2eq6}$ или $\eqref{ch35.2eq7}$ соответственно называется порядком полюса $a$ функции $f$.
\end{defn}

\begin{defn} \label{ch35.2defn4}
Пусть $a \in \bbC,$ $p > 0,$ $m \ge 1$, пусть функция $g : B_{\rho}(a) \to \bbC$ регулярна и
$$
g(a) = g'(a) = \ldots = g^{(m - 1)}(a) = 0, \quad g^{(m)}(a) \not= 0.
$$

Тогда говорят, что функция $g$ имеет в точке $a$ \textit{нуль $m$-го порядка (или $m$-й кратности)}. Если же $g(a) \not= 0$, то говорят, что точка $a$ не является нулем функции $g$ (или для следствия \ref{ch35.2cons2}: функция $g$ имеет в точке $a$ нуль нулевого порядка).
\end{defn}

\begin{cons} \label{ch35.2cons2}
Пусть функции $g,h : B_\rho(a) \to \bbC$ регулярны, причем функция $h$ имеет в точке $a$ нуль $k$-го порядка $(k \ge 0)$, а функция $g$ имеет в точке $a$ нуль $m$-го порядка $(m \ge 1)$. Тогда, если
$m > k$, то функция $f(z) = \frac{h(z)}{g(z)}$ имеет в точке $a$ полюс $(m - k)$-го порядка, а если $m \le k$, то функция $f$ имеет в точке $a$ устранимую особую точку.
\end{cons}

\begin{proof}
В силу определения $\eqref{ch35.2defn4}$ имеет для функций $h$ и $g$ представление
$$
h(z) = (z - a)^k h_1(z), \quad g(z) = (z - a)^m g_1(z),
$$

где функции $h_1,g_1 : B_\rho(a) \to \bbC$ регулярны и $h_1(a) \not= 0, \: g_1(a) \not= 0$.
Определим функцию $p(z) = \frac{h_1(z)}{g_1(z)}$. Эта функция регулярна в некоторой окрестности точки $a$. В итоге для функции $f$ получили формулу $\eqref{ch35.2eq6}$ и по следствию $\eqref{ch35.2cons1}$ получаем утверждение следствия $\eqref{ch35.2cons2}$.

\end{proof}

