\part[Аналитическая геометрия]{Аналитическая \protect\linebreak геометрия}

\chapter{Прямые и плоскости в пространстве.  Формулы расстояния от точки до прямой и плоскости, между прямыми в пространстве. Углы между прямыми и плоскостями.}
\section{Уравнения прямой на плоскости и в пространстве, плоскости в пространстве}
\subsection{Уравнения прямой на плоскости и в пространстве}
\begin{axiome} [Постулат Евклида]
  Через любую точку $P_0$ плоскости (пространства) можно провести единственную прямую, параллельную заданной прямой
\end{axiome}
  Рассмотрим точку $P_0$ на плоскости (в пространстве) и вектор $\vv{a}$. Построим уравнение, описывающее множество точек $P$, принадлежащих прямой $l$, проходящей через точку $P_0$ и параллельной вектору $\vv{a}$.

  Заметим, что условие $P \in l$ эквивалентно $ \vv{P_0P} \parallel \vv{a} $, или, если обозначить через $\vv{r}$ и $\vv{r_0}$ радиус-векторы точек $P$ и $P_0$ соответственно, $\vv{r}-\vv{r_0} = t\vv{a}$, что даёт
\begin{equation}\label{20.1.par1}
\vv{r}=\vv{r_0}+t\vv{a},\ t \in \bbR
\end{equation}
"--- \textit{параметрическое уравнение прямой на плоскости (в пространстве}), которое в координатной записи для $\vv{r}(x,y,z)$ и $\vv{a}(a_1,a_2,a_3)$ принимает вид

\begin{equation}\label{20.1.par2}\tag{\ref{20.1.par1}'}
  \left\lbrace\begin{array}{crl}
  x=x_0+a_1t\\
  y=y_0+a_2t\\
  z=z_0+a_3t,\\ 
  \end{array}\right.
\end{equation}

что можно переписать в виде
\begin{equation}\label{20.1.can1}
t=\frac{x-x_0}{a_1}=\frac{y-y_0}{a_2}=\frac{z-z_0}{a_3}
\end{equation}
"--- \textit{каноническое уравнение прямой в пространстве}.
  Уравнения \eqref{20.1.par1}, \eqref{20.1.can1} записываются аналогично для прямой на плоскости отбрасыванием условия на $z$.
  \begin{notion}
  Для соотношения \eqref{20.1.can1} действует следующая договорённость: если какие-то из коэффициентов $a_k$ равны нулю, соответствующие им числители приравниваются нулю. Заметим, что все коэффициенты не могут быть нулевыми, потому что направляющий вектор не может быть ноль-вектором.
  \end{notion}
  
Заметим, что каноническое уравнение \eqref{20.1.can1} в плоском случае представимо в виде $a_2(x-x_0)-a_1(y-y_0)=0$. Переобозначая $A=a_2, B=-a_1, C=a_1y_0-a_2x_0$, получим
  \begin{equation}\label{20.1.lin1}
  Ax+By+C=0
  \end{equation}
  "--- \textit{общее линейное уравнение прямой на плоскости}. Если $B\neq0$, его можно переписать в виде \textit{уравнения с угловым коэффициентом}:
  \begin{equation}\label{20.1.ang1}
  y=kx+b,
  \end{equation}
  где $k=-\frac AB, b=-\frac CB$.
    
  Пусть $P_0$ "--- точка прямой на плоскости, a \vv{n} "--- вектор нормали. Тогда имеет место соотношение $(\vv{r}-\vv{r_0},\vv{n})=0$, или
  \begin{equation}\label{20.1.norm1}
  (\vv{r},\vv{n})+D=0
  \end{equation}
  "--- \textit{нормальное уравнение прямой на плоскости}. Если система координат ортонормированна, раскрывая скалярное произведение, получим
  \begin{equation}
  n_1x+n_2y+D=0,
  \end{equation}
откуда и из сравнения с \eqref{20.1.lin1} видно, что $\vv{n}\parallel\left(\begin{array}{crl}
A\\
B
\end{array}\right)$.   

  В трёхмерном пространстве условие \eqref{20.1.par1} может быть переписано с использованием векторного произведения:
  \begin{equation}\label{20.1.vect1}
  [\vv{r}-\vv{r_0},\vv{a}]=\vv{0}, 
  \end{equation}
  что равносильно
  \begin{equation}\label{20.1.vect2}
  [\vv{r},\vv{a}]=\vv{b}, \vv{b}\bot\vv{a}
  \end{equation}
  "--- \textit{векторное уравнение прямой в пространстве}.
  
\subsection{Уравнения плоскости в пространстве}
\begin{axiome}[Постулат Евклида]
Через любую заданную точку пространства можно провести ровно одну плоскость, параллельную заданной плоскости.
\end{axiome}

Пусть \vv{a} и \vv{b} "--- два неколлинеарных направленных отрезка, лежащих в плоскости $\pi_0$, и задана точка $P_0$ в пространстве. Условием принадлежности точки $P$ плоскости $\pi$, проходящей через $P_0$ и параллельной плоскости $\pi_0$, будет $\vv{P_0P}=t_1\vv{a}+t_2\vv{b}$, или, если обозначить через $\vv{r}$ и $\vv{r_0}$ радиус-векторы точек $P$ и $P_0$ соответственно,
\begin{equation}\label{20.1.par3}
\vv{r}=\vv{r_0}+t_1\vv{a}+t_2\vv{b}, t_1,t_2 \in \bbR
\end{equation}
"--- \textit{параметрическое уравнение плоскости}.

Иначе записать условие можно как линейную зависимость векторов \vv{P_0P}, \vv{a} и \vv{b}, или, с использованием смешанного произведения:
\begin{equation}\label{20.1.mix1}
(\vv{P_0P},\vv{a}, \vv{b})=\begin{vmatrix}
x-x_0 & y-y_0 & z-z_0 \\
a_1 & a_2 & a_3 \\
b_1 & b_2 & b_3 \\
\end{vmatrix}(\vv{e_1},\vv{e_2},\vv{e_3})=0.
\end{equation}
Раскрывая определитель по первой строке и вводя обозначения $A=\begin{vmatrix}
a_2 & a_3 \\
b_2 & b_3 \\
\end{vmatrix}, B=\begin{vmatrix}
a_3 & a_1 \\
b_3 & b_1 \\
\end{vmatrix}, C=\begin{vmatrix}
a_1 & a_2 \\
b_1 & b_2 \\
\end{vmatrix}, D=-Ax_0-By_0-Cz_0$, получим
\begin{equation}\label{20.1.lin2}
A(x-x_0)+B(y-y_0)+C(z-z_0)=0,
\end{equation}
\begin{equation}\label{20.1.lin3}
Ax+By+Cz+D=0.
\end{equation}
Уравнение \eqref{20.1.lin3} называется \textit{общим линейным уравнением плоскости}, \eqref{20.1.lin2} "--- общим линейным уравнением плоскости, проходящей через $P_0(x_0,y_0,z_0)$.

Пусть $P_0$ "--- точка прямой в пространстве, a \vv{n} "--- вектор нормали к плоскости. Тогда соотношение \eqref{20.1.norm1} служит \textit{нормальным уравнением плоскости в пространстве}. Если система координат ортонормированна, получаем $\vv{n}\parallel\left(\begin{array}{crl}
A\\
B\\
C\\
\end{array}\right)$.

\section{Углы между прямыми и плоскостями}

\insertpicture[5]{ch20pict1}{0.18}
Пусть две прямые на плоскости заданы параметрическими уравнениями вида \eqref{20.1.par1} с использованием направляющих векторов \vv{a_1} и \vv{a_2}. Поскольку $(\vv{a_1},\vv{a_2})=\abs{a_1}\abs{a_2}\cos \phi$, где $\phi \in \left[0;\pi\right]$ "--- угол между векторами \vv{a_1} и \vv{a_2}, угол между прямыми, лежащий в интервале $\left[0;\dfrac \pi 2\right]$, находится по формуле
\begin{equation}\label{20.2.directions}
\widetilde\phi=\arccos\dfrac{\abs{(\vv{a_1},\vv{a_2})}}{\abs{a_1}\abs{a_2}}.
\end{equation}
Искомый угол также равен углу между нормалями, поэтому при задании прямой нормальным уравнением \eqref{20.1.norm1} имеем:
\begin{equation}\label{20.2.normals}
\widetilde\phi=\arccos\dfrac{\abs{(\vv{n_1},\vv{n_2})}}{\abs{n_1}\abs{n_2}}.
\end{equation}
  
\insertpicture{ch20pict4}{0.18}
Пусть две прямые заданы в виде \eqref{20.1.ang1} с угловыми коэффициентами $k_1$ и $k_2$. Тангенс угла с осью абсцисс $\tg \phi = \dfrac{a_2}{a_1}=-\dfrac{A}{B}=k$. Для угла между двумя заданными прямыми имеем:
\begin{equation}
\tg \phi=\tg (\alpha-\beta)=\dfrac{\tg \alpha - \tg \beta}{1+\tg \alpha \tg \beta}=\dfrac{k_1-k_2}{1+k_1k_2}
\end{equation}
\begin{equation}\label{20.2.angles}
\widetilde \phi = \arctg \abs{\dfrac{k_1-k_2}{1+k_1k_2}}
\end{equation}

Теперь перейдём в трёхмерное пространство. Угол между двумя прямыми по-прежнему можно найти по формуле~\eqref{20.2.directions}. 

Рассмотрим угол между плоскостями $\pi_1$ и $\pi_2$. Этот угол равен углу между их нормалями \vv{n_1} и \vv{n_2} (или является смежным, если найденный угол тупой), поэтому формула повторяет выражение~\eqref{20.2.normals}.
  
  
Пусть теперь даны плоскость $\pi$ с нормалью \vv{n} и прямая $l$ с направляющим вектором \vv{a}. Искомый угол выражается через смежный угол между векторами~\vv{a} и~\vv{n}:
\begin{equation}
\sin  \widetilde \phi = \cos  \widetilde \alpha = \dfrac{\abs{(\vv{a},\vv{n})}}{\abs{\vv{a}} \abs{\vv{n}}} \Rightarrow
\widetilde \phi = \arcsin \dfrac{\abs{(\vv{a},\vv{n})}}{\abs{\vv{a}} \abs{\vv{n}}}
\end{equation}

\section{Формулы расстояния от точки до прямой и плоскости, между прямыми в пространстве}
\insertpicture{ch20pict2}{0.25}
Пусть заданы точка $P_1$ плоскости и прямая $l$. Расстояние между ними равно длине проекции отрезка, соединяющего $P_1$ с любой точкой прямой, на нормаль прямой:
\begin{equation}\label{20.3.normals1}
\rho_2 (P_1,l) = \dfrac{ \abs{(\vv{r_1}-\vv{r},\vv{n})} }{ \abs{\vv{n}} }.
\end{equation}   
Используя нормальное уравнение \eqref{20.1.norm1}, получим:
\begin{equation}\label{20.3.normals2}
\rho_2 (P_1,l) = \dfrac{\abs{(\vv{r_1},\vv{n})+D}}{\abs{\vv{n}}}.
\end{equation}
В ортонормированном базисе уравнение \eqref{20.1.norm1} приобретает вид~\eqref{20.1.lin1}, а формула выглядит следующим образом:
\begin{equation}
\rho_2 (P_1,l) = \dfrac{\abs{Ax_1+By_1+C}}{\sqrt{A^2+B^2}}.
\end{equation}

При рассмотрении точки $P_1$ и прямой $l$ в трёхмерном пространстве удобно пользоваться векторным произведением и вычислять расстояние как модуль проекции отрезка, соединяющего точку $P_1$ с произвольной точкой $P$ прямой, на плоскость, нормальную направляющему вектору \vv{a}:
\begin{equation}
\rho_3 (P_1,l) = \dfrac{\abs{[\vv{a},\vv{r_1}-\vv{r}]}}{\abs{\vv{a}}}.
\end{equation}
Если используется представление \ref{20.1.vect2}, можно записать:
\begin{equation}
\rho_3 (P_1,l) = \dfrac{\abs{[\vv{a},\vv{r_1}]-\vv{b}}}{\abs{\vv{a}}}.
\end{equation}

Пусть теперь даны точка $P_1$ в пространстве и плоскость $\pi$. Повторяя рассуждения, получим формулы, аналогичные~\eqref{20.3.normals1} и~\eqref{20.3.normals2}, приобретающие в ортонормированном базисе вид
\begin{equation}
\rho_3 (P_1,\pi) = \dfrac{\abs{Ax_1+By_1+Cz_1+D}}{\sqrt{A^2+B^2+C^2}},
\end{equation}
если плоскость задаётся общим линейным уравнением~\eqref{20.1.lin3}.

Теперь рассмотрим случай двух скрещивающихся прямых в пространстве ($l_1|\cdot l_2$). Построим параллелепипед, двумя из рёбер которого являются направляющие векторы \vv{a_1} и \vv{a_2}, приложенные соответственно в точках $P_1$ и $P_2$ (см. рис.~\ref{ch20pict3}). Расстояние между ними будет равно расстоянию между гранями, параллельными обеим плоскостям, т.е., из определений скалярного и векторного произведения,
\begin{equation}
\rho_3(l_1,l_2)=h_{par}=\dfrac{V \lbrace \vv{a_1},\vv{a_2},\vv{P_1P_2} \rbrace}{S \lbrace \vv{a_1},\vv{a_2} \rbrace}=\dfrac{\abs{(\vv{a_1},\vv{a_2},\vv{r_2}-\vv{r_1})}}{\abs{[\vv{a_1},\vv{a_2}]}}.
\end{equation}   
\usepict{ch20pict3}{0.2}