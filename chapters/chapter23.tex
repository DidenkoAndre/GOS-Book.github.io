\chapter{Самосопряженные преобразования евклидовых пространств, свойства их собственных значений и собственных векторов.}

\font\Large=cmr10 at 20pt
\def\fudge#1{\smash{\hbox{\Large#1}}}

\section[Самосопряженные преобразования евклидовых пространств, свойства их собственных значений и собственных векторов.]{Самосопряженные преобразования евклидовых пространств, свойства их собственных значений и собственных векторов.\footnote{Рекомендую ознакомиться с написанными самим Чубаровым И.А. материалами по этому билету по этой ссылке: \href{https://drive.google.com/drive/u/0/folders/0BzuzEyNkpwYDYjVNcE0wa3hqWjA}{$drive.google.com/drive/...$}}}

\begin{defn}
Линейное пространство $E$ над полем вещественных чисел называется \textit{евклидовым}, если в нём введено \textit{скалярное произведение} $(\bullet,\bullet): E\times E \rightarrow \bbR$ -- операция удовлетворяющая следующим условиям:

(1) $(\lambda x,y)=\lambda(x,y) \fa x,y \in E \fa \lambda \in \bbR$

(2) $(x+y,z)=(x,z)+(y,z) \fa x,y,z \in E$

(3) $(x,y)=(y,x) \fa x,y \in E$

(4) $(x,x)>0 \fa x \in E: x \neq 0$
\end{defn}
  \begin{notion}
  Иначе говоря, скалярное произведение -- симметричная положительно определённая билинейная форма (см. билет 24).
  \end{notion}
  Рассмотрим, как происходит вычисление скалярного произведения через координаты в базисе $e=\norm{e_1 \ldots e_n}:x=\sum\limits_{i=1}^nx_ie_i=eX, y=\sum\limits_{j=1}^ny_je_j=eY$
  \begin{equation}\label{23.1.sc.pr}
  (x,y)=(\sum\limits_{i=1}^nx_ie_i,\sum\limits_{j=1}^ny_je_j)=\sum\limits_{i=1}^n\sum\limits_{j=1}^nx_i(e_i,e_j)y_j
  \end{equation}
  \begin{defn}
  \textit{Матрицей Грама} базиса $e$ называется матрица попарных скалярных произведений его элементов:
  \begin{equation}
  G_e=\norm{(e_i,e_j)}=\begin{pmatrix}
  (e_1,e_1)&&(e_1,e_2)&&\cdots&&(e_1,e_n)\\
  (e_2,e_1)&&(e_2,e_2)&&\cdots&&(e_2,e_n)\\
  \vdots   && \vdots  &&\ddots&& \vdots  \\
  (e_n,e_1)&&(e_n,e_2)&&\cdots&&(e_n,e_n)\\
  \end{pmatrix}
  \end{equation}
  \end{defn}
  Теперь выражение \ref{23.1.sc.pr} можно записать в виде:
  \begin{equation}
  (x,y)=X^TG_eY
  \end{equation}
  \begin{notion}
  Матрица Грама симметрична в силу симметричности скалярного произведения, однако не любая симметричная матрица может служить в качестве матрицы Грама: поскольку матрица Грама является матрицей билинейной формы скалярного произведения, она должна быть положительно определена (см. билет 24).
  \end{notion}
  
\begin{defn}
  Линейное преобразование $\phi: E \rightarrow E$ евклидова пространства называется \textit{самосопряжённым}, если
  \begin{equation}\label{23.1.selfadjoint}
  \fa x,y \in E \hookrightarrow (\phi(x),y)=(x,\phi(y))
  \end{equation}
\end{defn}
  \begin{thm}[Условие самосопряжённости преобразования в терминах матрицы Грама] Пусть матрица $A_\phi$ преобразования $\phi$ записана в некотором базисе $e$. $\phi$ является самосопряжённым преобразованием в том и только том случае, если
  \begin{equation}\label{23.1.criteria}
  A^T_{\phi,e}G_e=G_eA_{\phi,e}.
  \end{equation}  
  \end{thm}    
  \begin{proof}
  Пусть $X_e^\uparrow=X=
  \begin{pmatrix}
  x_1 \\ \vdots \\ x_n
  \end{pmatrix},Y_e^\uparrow=Y=
  \begin{pmatrix}
  y_1 \\ \vdots \\ y_n
  \end{pmatrix}$ - столбцы координат векторов $x=eX$ и $y=eY$. Тогда
  \begin{equation*}
  \left\lbrace \begin{array}{crl}
    (\phi(x),y)=(A_\phi X)^TG_eY=X^TA_\phi^TG_eY \\
    (\phi(x),y)=(x,\phi(y))=X^TG_eA_\phi Y \\
    \end{array}\right\rbrace \Rightarrow  \begin{array}{crl}
    X^TA_\phi^TG_eY=X^TG_eA_\phi Y, \\ \fa X,Y\in \bbR^n
  \end{array}     
  \end{equation*}   
  Обозначим $B=A_\phi^TG_e, C=G_eA_\phi$. Подставляя $X=E_i, Y=E_j$, где $E_k$ -- k-й столбец единичный матрицы, получим $E_i^TBE_j=E_i^TCE_j=b_{ij}=c_{ij}$, то есть $B=C$, что равносильно \eqref{23.1.criteria}.
  \end{proof}
  \begin{cons}
  В ортонормированном базисе $(G_e=E)$ условие самосопряжённости преобразования $\phi$ приобретает вид: $A=A^T$, то есть матрица должна быть симметричной.
  \end{cons}

  \begin{defn}
  Подпространство $U \subseteq L$ называется \textit{инвариантным подпространством} преобразовния $\phi$, или \textit{$\phi$-инвариантным}, если
  \begin{equation}\label{23.1.invariant}
  \fa u \in U \hookrightarrow \phi(u) \in U
  \end{equation}
  \end{defn} 
  
  \begin{defn}
  \textit{Ортогональным дополнением} подпространства $U$ евклидова пространства $E$ называется множество векторов из $E$, ортогональных каждому из векторов подпространства $U$, т.е.
  \begin{equation}\label{23.1.orthogonal}
  U^\perp=\{v \in E: \fa u \in U \hookrightarrow (u,v)=0\}
  \end{equation}
  \end{defn}
  
  Пусть $\phi$ -- самосопряжённое преобразование евклидова \linebreak пространства $E$.
  \begin{stt}\label{23.1.stt_orth}
  Если $U$ -- $\phi$-инвариантное подпространство в $E$, то его ортогональное дополнение $U^\perp$ также $\phi$-инвариантно.
  \end{stt}
  \begin{proof}
  $\fa x \in U \fa y \in U^\perp \hookrightarrow (x,\phi(y))=(\phi(x),y)=0$, т.к. $\phi(x)\in U$. Следовательно, $\phi(y)\in U^\perp \fa y\in U^\perp$.
  \end{proof}
  
  \begin{stt}
  Собственные векторы $\phi$, отвечающие различным собственным значениям, ортогональны.
  \end{stt}
  \begin{proof}
  Если $\phi(x_1)=\lambda_1 x_1, \phi(x_2)=\lambda_2 x_2, \lambda_1 \neq \lambda_2$, то $\lambda_1(x_1,x_2)=(\phi(x_1),x_2)=(x_1,\phi(x_2))=\lambda_2(x_1,x_2) \Rightarrow (\lambda_1-\lambda_2)(x_1,x_2)=0 \Rightarrow (x_1,x_2)=0$
  \end{proof}    
    
  \begin{lemm}\label{23.1.lemm}
  Любое линейное преобразование конечномерного действительного линейного пространства обладает одномерным или двумерным инвариантным подпространством $U$.
  \end{lemm}
  \begin{notion}
  Одномерное инвариантное подспространство порождается собственным вектором, а двумерное соответствует существенно комплексному характеристическому корню. В силу основной теоремы алгебры характеристический многочлен имеет хотя бы один комплексный корень.
  \end{notion}
  
  \begin{thm}\label{23.1.thm1}
  Все характеристические корни самосопряжённого преобразования действительные.
  \end{thm}
  \begin{proof} Проведём доказательство по индкуции по $n=\dim E$:
  \linebreak\vspace*{-\baselineskip}
  \begin{itemize}
  \item[\underline{$n=1:$}] случай очевиден.
  \item[\underline{$n=2:$}] в ортонормированном базисе
    \begin{equation}
    \bigchi_A(\lambda)=\begin{vmatrix}
    a_{11}-\lambda & a_{12} \\
    a_{12} & a_{22}-\lambda
    \end{vmatrix}=\lambda^2-(a_{11}+a_{22})\lambda+a_{11}a_{22}-a_{12}^2=0 \end{equation}
    Дискриминант этого уравнения $D=(a_{11}-a_{22})^2+4a_{12}^2 \ge 0 \Rightarrow \lambda_{1,2} \in \bbR$.
  \item[\underline{$n>2:$}] сделаем индуктивное прдеположение, что у любой симметрической матрицы порядка меньше $n$ все характеристические корни вещественные. Допустим, хотя бы один корень матрицы $A$ мнимый. Согласно замечанию к лемме \ref{23.1.lemm}, ему соответствует двумерное инвариантное пространство $U$. По утверждению \ref{23.1.stt_orth}, $U^\perp$ тоже инвариантно.
  
  В ортонормированном базисе, составленном из базисов подпространств, матрица преобразования имеет блочный вид (из определения матрицы преобразования) $A'=\begin{Vmatrix}
  A_1 & 0 \\
  0 & A_2 \\
  \end{Vmatrix}$, где $A_1$ и $A_2$ -- симметрические матрицы $2$-го и $(n-2)$-го порядков соответственно.
    \begin{equation}
    \bigchi_A(\lambda)=\begin{vmatrix}
    A_1-\lambda E & 0 \\
    0 & A_2-\lambda E
    \end{vmatrix}=\abs{A_1-\lambda E}\abs{A_2-\lambda E} \end{equation}
    По предположению индукции  $\abs{A_2-\lambda E}$ имеет все действительные корни, с учётом случая \underline{$n=2$} $\abs{A_1-\lambda E}$ -- тоже. Противоречие с предположением. 
    
    Т.о., теорема верна для всех $n$.
  \end{itemize}
  \vspace{-1.65\baselineskip}  
  \end{proof}
  
  \begin{defn}
  Преобразование $\phi|_U:U\rightarrow U \subseteq L$ называется \textit{ограничением} преобразования $\phi:L\rightarrow L$ на инвариантное подпространство $U$, если $\fa u \in U \hookrightarrow \phi|_U(u)=\phi(u)$.
  \end{defn}  
  \begin{notion}
  Ограничение самосопряжённого преобразования на инвариантное подпространство $U$ евклидова пространства $E$ остаётся самосопряжённым, если рассматривать на подпространстве склярное произведение, заданное во всём пространстве $E$.
  \end{notion}  
  
  \begin{thm}
  Для любого самосопряжённого преобразования существует ортонормированный базис из его собственных векторов. Матрица преобразования в этом базисе диагональна: $A=
  \begin{pmatrix}
  \lambda_1 &           &        & \\
            & \lambda_2 &        & \fudge {0 }  \\
            &			& \ddots & \\
  \fudge{ 0}&			&		 & \lambda_n \\
  \end{pmatrix}$, где $\lambda_1,...,\lambda_n$ -- собственные значения матрицы этого преобразования.
  \end{thm}
  \begin{proof} Проведём доказательство по индкуции по $n=\dim E$:
  \linebreak\vspace*{-\baselineskip}
  \begin{itemize}
  \item[\underline{$n=1:$}] случай очевиден.
  \item[\underline{$n>1:$}] пусть теорема верна для $\dim E<n$. Пусть $\lambda_1$ -- какой-либо характеристический корень, действительный по теореме \ref{23.1.thm1}, -- ему соответствует собственный вектор $h_1$ (сразу нормируем его: $\abs{h_1}=1$). $U=\langle h_1\rangle$ -- одномерное инвариантное пространство, натянутое на этот вектор. Согласно утверждению \ref{23.1.stt_orth} подпространство $U^\perp$, имеющее размерность $n-1$, инвариантно, и для ограничения $\phi|_{U^\perp}$ в силу предположения индукции существует ортонормированный базис из собственных векторов $h_2,...,h_n$. Тогда $h_1,...,h_n$ -- искомый базис.

  \end{itemize}
  \vspace{-1.65\baselineskip}  
  \end{proof}
