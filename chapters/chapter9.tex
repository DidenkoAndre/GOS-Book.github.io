\chapter[Теорема о неявной функции, заданной одним уравнением.]{Теорема о неявной функции, заданной одним уравнением\footnotemark.}
\footnotetext{На всякий случай предупрежу, что материал этого билета и следующего билетов читался в 3 семестре, на лекциях по курсу ``Кратные интегралы и теории поля'', однако по разумным соображениям я отношу этот материал к многомерному анализу.}
\section{Теорема о неявной функции, заданной одним уравнением}
\begin{defn}
Функция $y=f(x)$ называется \textit{неявной функцией, заданной уравнением} $F(x,y)=0$, если $F(x,f(x))=0$ для любого $x\in D_f$. (Здесь, как обычно, $D_f$ "--- область определения функции $f$.)
\end{defn}

\begin{thm}[о существовании и единственности неявной функции, заданной одним уравнением]\label{yaa14th1}
Пусть функция $F(x,y)$ определена и непрерывна в некоторой $\delta$-окрестности точки $(x_0,y_0)$, и пусть $F(x_0,y_0)=0$. Тогда, если $F(x,y)$ при каждом фиксированном $x$ строго монотонна по $y$, то у точек $x_0$ и $y_0$ существуют окрестности $\Delta$ и $(a;b)$ такие, что на множестве $\Delta\times(a;b)$ уравнение $F(x,y)=0$ определяет единственную неявную функцию $y=f(x),\; x\in\Delta$, и эта функция  $f$ непрерывна на $\Delta$. 
\end{thm}

\begin{proof}

По условию функция $F(x_0,y)$ строго монотонна и равна нулю при  $y_0$.  Пусть для определенности, она строго возрастает. Тогда $F(x_0,y)>0$ для всех допустимых $y>y_0$ и $F(x_0,y)<0$ для всех допустимых $y<y_0$.
Выберем некоторые $a$ и $b$ такие, что $a<y_0<b$ и точки $(x_0,a)$, $(x_0,b)$ лежат в $\delta$-окрестности точки $(x_0,y_0)$. Тогда
$$
F(x_0,a)<0<F(x_0,b).
$$

Функции $F(x,a)$ и $F(x,b)$ непрерывны в точке $x_0$, поэтому существуют окрестности $\Delta'$ и $\Delta''$ точки $x_0$ такие, что (рис.\ref{yu})
$$
F(x,a)<0 \quad \forall x\in \Delta', \qquad F(x,b)>0\quad \forall x\in \Delta''.
$$

Отсюда следует, что $F(x,a)<0<F(x,b)$ для любого $x$ из интервала $\Delta = \Delta'\cap\Delta''$. А так как функция $F(x,y)$ при каждом фиксированном $x\in\Delta$ по $y$ непрерывна и строго монотонна, то для каждого $x\in \Delta$ существует единственное $y$, которое обозначим $f(x)$, такое что $f(x)\in(a;b)$ и $F(x,f(x))=0$. Следовательно, на прямоугольнике $\Delta\times(a;b)$ уравнение $F(x,y)=0$ определяет единственную неявную функцию $y=f(x)$. Докажем, что она непрерывна в точке $x_0$.

\addpicture{pictures/ch9pict1.png}{0.4}\label{yu}
Выберем некоторую окрестность $(\alpha;\beta)$ точки $y_0$. Не ограничивая общности, будем считать, что $(\alpha;\beta)\subset (a;b)$. Тогда точно так же, как и для интервала $(a,b)$, строится окрестность $\Delta=\Delta(\alpha;\beta)$ точки $x_0$ такая, что $\forall x\in\Delta \quad f(x)\in(\alpha;\beta)$. А это и означает, что функция $f$ непрерывна в точке $x_0$.

Непрерывность функции $y=f(x)$ в любой точке $x_1\in\Delta$ следует из того, что в точке с координатами $x_1$ и $y_1=f(x_1)$ выполнены все условия теоремы, поэтому, согласно доказанному, у точки $(x_1,y_1)$ существует прямоугольная окрестность, в которой уравнение $F(x,y)=0$ определяет единственную функцию $y=f_1(x),\;x\in\Delta_1$, которая непрерывна в точке $x_1$. Очевидно, что $f_1(x)=f(x)\quad \forall x\in\Delta\cap\Delta_1$, и поэтому функция $f(x)$ непрерывна в точке $x_1\in\Delta$.

Теорема доказана.
\end{proof}
\begin{cons}
Пусть функция $F(x,y)$ в некоторой окрестности точки $(x_0,y_0)$ непрерывна и имеет частную производную $F'_y(x,y)$, непрерывную в точке $(x_0,y_0)$. Тогда, если $F(x_0,y_0)=0$ и $F'_y(x_0,y_0)\ne 0 $, то справедливы утверждения теоремы \ref{yaa14th1}.
\end{cons}

\begin{proof}
Пусть для определенности $F'_y(x_0,y_0)>0$. Тогда из непрерывности функции $F'_y(x,y)$ в точке $(x_0,y_0)$ следует, что $F'_y(x,y)>0$ в некоторой $\delta$-окрестности этой точки, и поэтому для уравнения $F(x,y)=0$ в точке $(x_0,y_0)$ выполнены все условия теоремы \ref{yaa14th1}. Случай $F'_y(x_0,y_0)<0$ рассматривается аналогично.

Следствие доказано.
\end{proof}

\begin{thm} \label{yaa12th2}
Пусть функция $F(x,y)$ в некоторой окрестности точки $(x_0,y_0)$ непрерывна и имеет производные $F'_x(x,y)$ и $F'_y(x,y)$, непрерывные в точке $(x_0,y_0)$. Тогда если $F(x_0,y_0)=0$ и $F'_y(x_0,y_0)\ne 0$, то имеют место утверждения теоремы \ref{yaa14th1} и, кроме того, функция $y=f(x)$ в точке $x_0$ имеет производную
\begin{equation}\label{yaa12e2}
\frac{dy}{dx} = -\frac{F'_x(x_0,y_0)}{F'_y(x_0,y_0)}.
\end{equation}
\end{thm}

\begin{proof}
В силу непрерывности производных $F'_x$ и $F'_y$ в точке $(x_0,y_0)$ функция $F(x,y)$ дифференцируема в этой точке, поэтому
$$
F(x,y)-F(x_0,y_0)=F'_x(x_0,y_0)\Delta x+F'_y(x_0,y_0)\Delta y+\alpha\Delta x+\beta\Delta y,
$$
где $\Delta x=x-x_0,\, \Delta y = y-y_0$, а функции $\alpha$ и $\beta$ такие, что $\alpha\to 0$ и $\beta \to 0$ при $(x,y)\to (x_0,y_0)$. Положим здесь $y=f(x)$. Тогда $F(x,y) = F(x_0,y_0)=0$, следовательно,
\begin{gather*}
F'_x(x_0,y_0)\Delta x+F'_y(x_0,y_0)\Delta y + \alpha\Delta x+\beta\Delta y=0,\\
\frac{\Delta y}{\Delta x} = -\frac{F'_x(x_0,y_0)+\alpha}{F'_y(x_0,y_0)+\beta}.
\end{gather*}
Отсюда в пределе при $x \to x_0$ получаем, что функция $y=f(x)$ в точке $x_0$ имеет производную и справедлива формула \eqref{yaa12e2}.

Теорема доказана.
\end{proof}

\begin{cons}\label{yaa12c2}
Если выполнены все условия теоремы \ref{yaa12th2} и, кроме того, производные $F'_x$ и $F'_y$ непрерывны в окрестности точки $(x_0,y_0)$, то неявная функция $y=f(x),\,x\in\Delta$, имеет непрерывную производную.
\end{cons}

\begin{proof}
В теореме \ref{yaa14th1} доказано, что функция $y=f(x)$ непрерывна. Из теоремы~\ref{yaa12th2} следует, что она имеет производную, причем
$$
f'(x)=-\frac{F'_x(x,f(x))}{F'_y(x,f(x))},\quad x\in\Delta.
$$
Поэтому, в силу теоремы о непрерывности композиции непрерывных функций, производная $f'(x)$ непрерывна на $\Delta$.

Следствие доказано.
\end{proof}

\begin{cons}
Если выполнены все условия следствия \ref{yaa12c2} и, кроме того, функция $F(x,y)$ непрерывно дифференцируема $l$ раз в окрестности точки $(x_0,y_0)$, то неявная функция $y=f(x)$ на интервале $\Delta$ имеет непрерывные производные до $l$-го порядка включительно.
\end{cons}
