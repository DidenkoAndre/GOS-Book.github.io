\part{Введение в математический анализ}
\chapter[Теорема Больцано-Вейерштрасса и критерий Коши сходимости числовой последовательности.]{Теорема Больцано-Вейерштрасса и критерий Коши сходимости числовой последовательности.}
\section{Теорема Больцано-Вейерштрасса}

\subsection{Последовательности и пределы}
\begin{defn}
Пусть имеется правило, которое каждому натуральному числу $n$ ставит в соответствие некоторое число $x_n$. Тогда множество всевозможных упорядоченных пар $(n;x_n), n \in \bbN$, называется \textit{числовой последовательностью}\rindex{числовая последовательность} и обозначается либо $\{x_n\}$, либо $x_n, n \in \bbN$, либо $x_1,x_2,\dots,x_n,\dots$
\end{defn}

\begin{defn}
Последовательность действительных чисел $\{x_n\}$ называется \textit{ограниченной сверху}, если существует число $M$ такое, что $x_n \le M$ для любого $n \in \bbN$. Аналогично, последовательность $\{x_n\}$ называется \textit{ограниченной снизу}, если выполняется условие:
$$
\ex m \in \bbR: \fa n \in \bbN\quad x_n \ge m.
$$
Последовательность называется \textit{ограниченной}\rindex{последовательность!ограниченная}, если она ограничена и сверху, и снизу: 
$$
\ex M\in [0;+\infty): \fa n \in \bbN\quad |x_n| \le M.
$$
\end{defn}

\begin{defn}
Последовательность $\{x_n\}$ называется \textit{строго возрастающей (убывающей)}, если 
$$
\fa n \in \bbN \quad x_n < x_{n+1} \quad (\text{соотв., \:} x_n > x_{n+1}).
$$
\end{defn}

\begin{defn}
Пусть $c \in \bboR = \{-\infty\}\cup\{\bbR\}\cup\{+\infty\}$, тогда 
\begin{itemize}
\item
\textit{окрестностью} числа $c \in \bbR$ называется любой интервал $(a;b)\ni c,$ $(a;b) \subset \bbR$.
\item
\textit{окрестностью} элемента $+\infty$ называется любой луч      $(a;+\infty),\: a \in \bbR$.
\item
\textit{окрестностью} элемента $-\infty$ называется любой луч $(-\infty;b),\: b \in \bbR$.
\end{itemize}
\end{defn}

\begin{defn}
Пусть $\epsilon > 0$ и $c \in \bboR $, тогда если $c \in \bbR$, то
\begin{itemize}
\item
\textit{$\epsilon$-окрестностью}~$O_\epsilon$ числа $c$ называется интервал $(c-\epsilon;c+\epsilon) = O_\epsilon$
\item
Если $c=+\infty$, то \textit{$\epsilon$-окрестностью} $+\infty$ называется луч $(\epsilon;+\infty)$.
\item
Если $c=-\infty$, то $O_\epsilon(-\infty)=(-\infty;-\epsilon)$.
\end{itemize}
\end{defn}
Всякая $\epsilon$-окрестность элемента $c \in \bboR$ является его окрестностью, но не наоборот.

\begin{defn}
Число или бесконечно удаленная точка $c \in \bboR$ называется \textit{пределом}\rindex{предел!последовательности} последовательности $\{x_n\}$, если выполняется условие:
$$
\fa O(c): \ex M \in \bbN: \fa n \ge M:\quad x_n \in O(c).
$$
Обозначается $\lim_{n \to \infty}\limits x_n = c$.
\end{defn}

\begin{lemm}
Число $x_0$ является пределом последовательности тогда и только тогда, когда выполняется условие:
\begin{equation}
\label{eq:ch1:predel}
\fa \epsilon>0: \ex N_\epsilon \in \bbN: \fa n \ge N_\epsilon,\quad x_n \in O_\epsilon(x_0)
\end{equation}
\end{lemm}
Заметим, что условие~\eqref{eq:ch1:predel} часто записывают так:
$$
\fa \epsilon>0: \ex N_\epsilon \in \bbN: \fa n \ge N_\epsilon,\quad |x_n-x_0|<\epsilon.
$$
\begin{defn}
Последовательность называется \textit{сходящейся}, если она имеет конечный предел. Если же последовательность не имеет конечного предела, то она называется \textit{расходящейся}.
\end{defn}
В дальнейшем будем говорить <<последовательность сходится>>, имея в виду, что она имеет конечный предел. Если же ее предел будет равен $\pm\infty$, будем отдельно отмечать <<последовательность сходится к $\pm\infty$>>, однако такие последовательности являются расходящимися, поэтому иногда говорят, что они расходятся к $\pm\infty$. 
\begin{thm}[о трех последовательностях] \label{th:ch1:otrehposled}  
Пусть числовые последовательности $\{x_n\}$, $\{y_n\}$ и $\{z_n\}$ удовлетворяют условиям:
$$
\ex N_0 \in \bbN:\fa n\ge N_0 \quad x_n \le y_n \le z_n 
$$
Тогда, если $\{x_n\}$ и $\{z_n\}$ сходятся и их пределы равны, то $\{y_n\}$ тоже сходится к тому же пределу.
\end{thm}

\begin{defn}
Последовательность отрезков $[a_n;b_n],\; n \in \bbN$, называется \textit{последовательностью вложенных отрезков}\rindex{последовательность!вложенных отрезков}, если
$$
[a_{n+1};b_{n+1}]\subset [a_n;b_n] \fa n \in \bbN.
$$
\end{defn}

\begin{defn}
Последовательность вложенных отрезков называется \textit{стягивающейся}, если последовательность длин этих отрезков сходится к нулю.
\end{defn}

\begin{thm}
\label{th:ch1:poslstyag}
Любая последовательность стягивающихся отрезков действительной прямой имеет единственную общую точку.
\end{thm}

Теорему~\ref{th:ch1:poslstyag} можно сформулировать следующим образом: 
\begin{thmn}
Любая последовательность стягивающихся отрезков стягивается к некоторой точке.
\end{thmn}

\subsection{Частичный предел последовательности}
\begin{defn}
Последовательность $\{y_k\}$ называется \textit{подпоследовательностью} последовательности $\{x_n\}$, если 
$$
\fa k \in \bbN \quad \ex n=n_k:\quad y_k=x_{n_k},
$$
где последовательность $\{n_k\}$ строго возрастающая. Эта подпоследовательность обозначается $\{x_{n_k}\}$.
\end{defn}

\begin{defn}
Предел любой подпоследовательности данной последовательности называется \textit{частичным пределом} этой последовательности.
\end{defn}

\begin{thm}[Больцано-Вейерштрасса]\label{th:ch1:TBV}
\rindex{теорема!Больцано-Вейерштрасса} У любой ограниченной последовательности существует сходящаяся подпоследовательность.
\end{thm}
\begin{proof}
Пусть последовательность $\{x_n\}$ ограничена, т.е. существуют числа $a$ и $b$ такие, что $a \le x_n \le b \fa n$. Точкой
$c_0 = (a + b)/2$ отрезок $[a; b]$ разделим на два равных по длине отрезка $[a; c_0]$ и $[c_0,b]$. Тогда хотя бы в одном из них лежат значения бесконечного множества элементов последовательности $\{x_n\}$. Через $[a_1;b_1]$ обозначим отрезок $[c_0;b]$, если он содержит значения бесконечного множества элементов последовательности, в противном случае через $[a_1; b_1]$ обозначим отрезок $[a; c_0]$. Отрезок $[a_1, b_1]$ точкой $c_1 = (a_1 + b_1)/2$ снова разделим на два отрезка $[a_1;c_1]$ и $[c_1;b_1]$, и через $[a_2;b_2]$ обозначим отрезок $[c_1;b_1]$, если он содержит значения бесконечного множества элементов последовательности $\{x_n\}$, и отрезок $[a_1; c_1]$ в противном случае. Таким образом, делением пополам, строится последовательность вложенных отрезков $[a_k;b_k],\; k \in \bbN$, каждый из которых содержит значения бесконечного множества членов последовательности $\{x_n\}$, причем
$$
\lim_{k\to \infty}(b_k-a_k)=\lim_{k \to \infty}\frac{(b-a)}{2^k}=0.
$$
Следовательно (см. теорему \ref{th:ch1:poslstyag}), эти отрезки имеют одну общую точку
$$
c = \lim_{k \to \infty} a_k = \lim_{k \to \infty} b_k.
$$
Построим подпоследовательность $\{x_{n_k}\}$ следующим образом:
\begin{gather*}
\begin{aligned}
& x_{n_1} \in [a_1;b_1];\\
& x_{n_2} \in [a_2;b_2],\quad n_2 \ge n_1;\\
& ...\\
& x_{n_k} \in [a_k;b_k],\quad n_k \ge n_{k-1};\\
& x_{n_{k+1}} \in [a_{k+1};b_{k+1}],\quad n_{k+1} \ge n_k;\\
& ...
\end{aligned}
\end{gather*}
Так как $a_k \le x_{n_k} \le b_k \;\; \forall k$, то из \hyperref[th:ch1:otrehposled]{теоремы о трех последовательностях} следует, что $\lim_{k \to \infty}\limits x_{n_k} = c$.

Теорема доказана.
\end{proof}

\hyperref[th:ch1:TBV]{Теорему Больцано-Вейерштрасса} можно сформулировать следующим образом: 
\begin{thmn}[Больцано-Вейерштрасса] Любая ограниченная последовательность имеет хотя бы один частичный предел.
\end{thmn}

\section{Критерий Коши сходимости числовой последовательности}

\begin{defn}
Последовательность $\{x_n\}$ \textit{фундаментальна}\rindex{последовательность!фундаментальная}, если она удовлетворяет \textit{условию Коши}:
\begin{equation}
\label{eq:ch1:fundamen}
\forall \epsilon >0 \ex N_{\epsilon} \in \bbN: \;\forall m,n > N_{\epsilon}: |x_n-x_m|<\epsilon  
\end{equation}
\end{defn}

\begin{lemm}
\label{lm:ch1:fundamendal}
Если последовательность $\{x_n\}$ фундаментальна, то она ограничена.
\end{lemm}
\begin{proof}
Пусть $\{x_n\}$ фундаментальна, т.е. удовлетворяет \eqref{eq:ch1:fundamen}. Положим $\epsilon = 1,\; m = N_1$. Тогда $\forall n > N_1:\\ |x_n-x_{N_1}| <\epsilon = 1 \Leftrightarrow \fa n > N_1: |x_n - x_{N_1}| < 1$, т.е. $x_{N_1} - 1< x_n < x_{N_1}+1$. Если теперь через $a$ и $b$ обозначим наименьшее и наибольшее из чисел $x_1,\dots,x_{N_1},x_{N_1}-1,x_{N_1}+1$, то, очевидно, $a \le x_n \le b \fa n$.

Лемма доказана.
\end{proof}

\begin{thm}[Критерий Коши\rindex{критерий!Коши}] 
$\{x_n\}$ сходится $\Longleftrightarrow$ $\{x_n\}$ фундаментальна, где $\{x_n\}$ "--- числовая последовательность.
\end{thm}
\begin{proof}\leavevmode
\begin{itemize}[wide, labelwidth=!, labelindent=0pt]
\item[$\Longrightarrow$:]

Пусть последовательность $\{x_n\}$ сходится, тогда $\exists x_{0} \in \bbR\colon  \lim_{n \to \infty}\limits x_n = x_0$. Тогда $\\ \fa \epsilon > 0 \ex N_{\epsilon/2} \in \bbN: \fa n \ge N_{\epsilon/2} : |x_n-x_0|<\epsilon/2$. Отсюда следует, что если $n \ge N_{\epsilon/2}$ и $m \ge N_{\epsilon/2}$, то
$$
|x_n-x_m| \le |x_n-x_0|+|x_0-x_m| <\frac{\epsilon}{2}+\frac{\epsilon}{2}=\epsilon
$$

\item[$\Longleftarrow$:]
Пусть $\{x_n\}$ фундаментальна. Тогда, согласно лемме~\ref{lm:ch1:fundamendal} $\{x_n\}$ ограничена. Следовательно, по \hyperref[th:ch1:TBV]{теореме Больцано-Вейерштрасса} у нее есть сходящаяся подпоследовательность $\{x_{n_k}\}: \;\exists x_0 \in \bbR\colon$ $\lim_{k \to \infty}\limits x_{n_k} =x_0 $. Докажем, что $\lim_{n \to \infty}\limits x_{n}=x_0 $.

Зададим некоторое $\epsilon > 0$. Тогда
\begin{gather*}
\begin{aligned}
& \ex N_\epsilon : \fa n,m \ge N_\epsilon &&|x_n-x_m|<\epsilon/2\\
& \ex K_\epsilon : \fa k \ge K_\epsilon   &&|x_{n_k}-x_0|<\epsilon/2.
\end{aligned}
\end{gather*}
Положим $p=\max\{N_\epsilon;K_\epsilon\}$. Тогда, очевидно, $p \ge K_\epsilon,\; n_p \ge p \ge N_\epsilon$ и, следовательно, для любого $n \ge N_\epsilon$
$$
|x_n-x_0| \le |x_n-x_{n_p}|+|x_{n_p}-x_0|<\frac{\epsilon}{2}+\frac{\epsilon}{2}=\epsilon.
$$
А так как $\epsilon > 0$ любое, то этим доказано, что $\lim_{n \to +\infty}\limits x_n = x_0$. \qedhere
\end{itemize}
\end{proof}
