\part{Введение в математический анализ}
\chapter{Теорема Больцано"--~Вейерштрасса и критерий Коши сходимости числовой последовательности.}

\section[Аксиоматика множества действительных чисел]{Аксиоматика множества действительных чисел}

\begin{defn}
Будем говорить, что на множестве $G$ определена операция сложения <<$+$>> (умножения <<$\cdot$>>), если любой упорядоченной паре элементов ($a$,~$b$) элементов $G$ поставлен в соответствие элемент $y = a + b$ (соотв., $y = a\cdot b$)
\end{defn}

Зачастую знак умножения <<$\cdot$>> опускают и пишут $ab$ вместо $a\cdot b$.

\begin{defn}
Будем говорить, что на множестве $G$ задано отношение порядка <<$\le$>>, если для любых двух элементов $a,b\in G$ выполняется хотя бы одно из условий $a\le b$, $b \le a$.
\end{defn}

Другими словами, для любых $a, b \in \bbR$ заранее установлено, верно или неверно неравенство $a \le b$.

\begin{defn}
Множество $\bbR$, состоящее более, чем из одного элемента, называется \textit{множеством действительных чисел}, а его элементы "--- \textit{действительными числами}, если на $\bbR$ определены операции сложения~<<$+$>> и умножения~<<$\cdot$>> и отношение порядка~<<$\le$>>, удовлетворяющие следующим~$15$ аксиомам\rindex{аксиомы!множества действительных чисел}:
\end{defn}

\begin{enumerate}[label=\Roman*.]
\item
\textbf{Аксиомы сложения} (\,$+\colon a,b \to a+b$\,)
\begin{enumerate}[label=\arabic*.]
\item 
$a + b = b + a\quad\forall a,b \in \bbR$ (\textit{коммутативность});
\item
$a + (b + c) = (a + b) + c\quad \forall a,b,c \in \bbR$ \textit{(ассоциативность});
\item
$\exists 0 \in \bbR\cquad a + 0 = a\quad \forall a \in \bbR$ (\textit{нейтральность нуля});
\item 
$\forall a \in \bbR \quad \exists (-a) \in \bbR\cquad a + (-a) = 0$, $(-a)$ называется \textit{противоположным} числом для $a$ (\textit{существование противоположного}).
\end{enumerate}
\item
\textbf{Аксиомы умножения} (\,$\cdot\colon a,b \to a\cdot b$\,)
\begin{enumerate}[resume, label=\arabic*.]
\item
$a\cdot b = b \cdot a \quad \forall a,b\in \bbR$ (\textit{коммутативность});
\item
$a \cdot (b \cdot c) = (a \cdot b) \cdot c\quad \forall a,b,c \in \bbR$ (\textit{ассоциативность});
\item 
$\exists 1 \in \bbR$, $1\ne 0\cquad a\cdot1 = a\quad \forall a\in \bbR$ (\textit{нейтральность единицы});
\item 
$\forall a\in \bbR$, $a\ne 0,\quad \exists \dfrac{1}{a} \in \bbR\cquad a\cdot \dfrac{1}{a} = 1$, $\dfrac{1}{a}$ называется \textit{обратным} числом для $a$ (\textit{существование обратного}).
\end{enumerate}
\item
\textbf{Аксиома связи сложения и умножения}
\begin{enumerate}[resume, label=\arabic*.]
\item
$(a+b) \cdot c = a\cdot c +  b\cdot c\quad \forall a,b,c \in \bbR$ (\textit{дистрибутивность умножения относительно сложения}).
\end{enumerate}
\item
\textbf{Аксиомы порядка} 
\begin{enumerate}[resume, label=\arabic*.]
\item 
$ a \le a\quad \forall a \in \bbR$ (\textit{рефлексивность});
\item
если $ a\le b$ и $b\le a$, то $a=b \quad \forall a,b\in\bbR$ (\textit{антисимметричность});
\item
если $a\le b$ и $b\le c$, то $a\le c\quad \forall a,b,c\in \bbR$ (\textit{транзитивность}).
\end{enumerate}
\item
\textbf{Аксиома связи сложения и порядка}
\begin{enumerate}[resume, label=\arabic*.]
\item 
если $ a \le b$, то $ a + c \le b + c\quad \forall a,b,c \in \bbR$ (\textit{монотонность}).
\end{enumerate}
\item
\textbf{Аксиома связи умножения и порядка}
\begin{enumerate}[resume, label=\arabic*.]
\item 
если $ 0 \le a$ и $0\le b$, то $ 0 \le a\cdot b\quad \forall a,b \in \bbR$ (\textit{монотонность}).
\end{enumerate}
\item
\textbf{Принцип непрерывности}
\begin{enumerate}[resume, label=\arabic*.]
\item 
Пусть $A$, $B$ "--- непустые подмножества $\bbR$ такие, что
$$
a\le b\quad \forall a \in A, \,  \forall b \in B.
$$
Тогда $\exists c \in \bbR$ такое, что $a\le c\le b\quad \forall a \in A,\ \forall b\in B.$
\end{enumerate}
\end{enumerate}
\enlargethispage{\baselineskip}
\begin{defn}
Определим отношения порядка <<$<$>>, <<$>$>>, <<$\ge$>> и операции вычитания~<<$-$>> и деления~<<$/$>> на множестве $\bbR$:
\begin{itemize}[noitemsep,  topsep=0pt]
\item 
$a < b \Longleftrightarrow a \le b \text{ и } a\neq b$;
\item
$a > b \Longleftrightarrow b < a$, аналогично $a \ge b \Longleftrightarrow b \le a$;
\item
$a - b \triangleq a + (-b)$;
\item
$a / b =  \dfrac{a}{b} \triangleq a \cdot \dfrac{1}{b}$; 
\end{itemize}
\end{defn}

Напоследок, определим некоторые важнейшие числовые множества:
\begin{itemize}[wide, labelwidth=!, labelindent=0pt, nolistsep, topsep=0pt]
\item 
Множество \textit{натуральных} чисел 
$$\bbN = \{1,\,1+1=2,\,\ldots,\,n = 1+\ldots+1,\,\ldots\}.$$
\item 
Множество $\bbN_0 \triangleq \bbN \cup \{0\}$.
\item 
Множество \textit{целых} чисел 
$$
\bbZ = \{ 0,\,1,\,-1,\,2,\,-2,\,3,\,-3,\,\dots\},
$$ 
т.е.~множество чисел $x$ таких, что $x\in \bbN$ или $-x \in \bbN$, или $x = 0$.
\item Пусть $a,b\in\bbZ$, $a<b$. Тогда обозначим за $\overline{a,b}$ следующее множество
$$
\overline{a,b} = \{a,\,a+1,\,\dots,\,b-1,\,b\}.
$$
\item
Множество \textit{рациональных} чисел 
$$
\bbQ=\{x\,\big|\,x=p/q,\, q\in\bbN,\ p\in\bbZ\}.
$$
\item 
Множество \textit{иррациональных чисел} $\bbR \setminus \bbQ$.
\item 
Множество действительных чисел $\bbR$ часто называют \textit{числовой прямой}, а числа "--- \textit{точками числовой прямой}. А сами действительные числа часто будем называть \textit{вещественными числами}, или просто \textit{числами}, подразумевая именно элементы из множества~$\bbR$.
\item 
Наряду с числовой прямой определим \textit{расширенную числовую прямую:} множество $\bboR=\bbR\cup \{-\infty,\,+\infty\}$. В нем определены отношения порядка, операции  <<$+$>>, <<$-$>>, <<$\cdot$>> и <<$/$>> для элементов из $\bbR$. Элементы $-\infty$, $+\infty$ не содержатся в $\bbR$. Будем их называть \textit{бесконечно удаленными точками (числами)} в противопоставление точкам числовой прямой $\bbR$, которые также называют \textit{конечными точками (точками)}. 

Для плюс и минус бесконечностей определены отношения порядка: $\forall x \in \bbR$ $-\infty < x < +\infty$. Частично определены операции <<$+$>>, <<$-$>>, <<$\cdot$>> и <<$/$>>
\begin{alignat*}{1}
&{+\infty} + a = +\infty \text{ и }  {+\infty}- a = +\infty\quad \forall a \in \bbR\cup\{+\infty\}; \\
&{-\infty} - a = -\infty \text{ и } {-\infty} + a = -\infty\quad \forall a \in \bbR\cup\{-\infty\}; \\
&{+\infty}\cdot a = +\infty \text{ и } {-\infty}\cdot a = -\infty\quad \forall a > 0,\,  a\in\bbR\cup\{+\infty\};\\ 
&{+\infty}\cdot a = -\infty \text{ и } {-\infty}\cdot a = +\infty \quad \forall a < 0,\,  a\in\bbR\cup\{-\infty\};\\
&a/ (-\infty) = 0 \text{ и } a / (+\infty) = 0 \quad \forall a \in \bbR;
\end{alignat*}
Но, например, не определены: сумма $+\infty + (-\infty)$, произведение $0\cdot (\pm \infty)$, частное $(\pm \infty )/ (\pm \infty)$.
\item
Пусть заданы действительные числа $a,b\in \bbR$, $a < b$. \textit{Числовыми промежутками} называются следующие множества:
\begin{itemize}[nolistsep, label = $\scriptstyle\blacktriangleright$, topsep=0pt]
\item 
\textit{интервал} $(a;b)\triangleq \{x\in\bbR\,\big|\,a < x < b\}$;
\item 
\textit{отрезок} $[a;b] \triangleq \{x\,\big|\,a\le x \le b\}$;
\item 
\textit{полуинтервалы} $(a,b]$, $[a,b)$ c аналогичным определением;
\item 
\textit{лучи} $(-\infty; a) \triangleq \{x\,\big|\,x < a \} $, $(a; +\infty)$, $(-\infty;a]$, $[a;+\infty]$ с аналогичным определением;
\item 
\textit{точка} $\{a\};$
\item 
\textit{числовая прямая} $(-\infty; +\infty) = \bbR$
\end{itemize}
\end{itemize}

Множество $\bbQ$ рациональных чисел удовлетворяет аксиомам~I--VI, но не удовлетворяет аксиоме VII. Покажем последнее. Пусть 
$A = \{a\,\big|\,a\in\bbQ,\ a>0,\ a^2<2\},$ $B = \{b\,\big|\,b\in\bbQ,\ b>0,\ b^2>2\}.$ Тогда во множестве $\bbQ$ не существует числа $c\in\bbQ$ со свойством: $a\le c\le b \quad \forall a \in A,~\forall b\in B$.

\section{Точные грани числовых множеств}
\begin{defn}
Число $M \in \bbR$ "--- \textit{верхняя (нижняя) грань множества}\rindex{грань! множества} $G\subset \bbR$, если выполняется условие 
$$
\forall x \in G \quad x \le M \quad (\text{соотв., }x \ge M)
$$
\end{defn}
\begin{defn}
Множество $G\subset \bbR$ называется \textit{ограниченным сверху (снизу)}\rindex{множество!ограниченное}, если существует верхняя (нижняя) грань этого множества:
$$
\exists M \in G \cquad \forall x \in G \quad x \le M \quad (\text{соотв., } x \ge M)
$$
Множество $G$ называется \textit{ограниченным}, если $G$ ограничено сверху и снизу.
\end{defn}
\begin{defn}
\textit{Модулем} числа $x$ называется число 
$$
|x|=\begin{cases}x \text{, если }x\ge 0;\\-x\text{, если }x<0.\end{cases}
$$
\end{defn}
Тогда, с введенным определением, можно показать, что множество $G$ ограничено тогда и только тогда, когда 
$$
\exists M \in \bbR \cquad \forall x \in G \quad |x| \le M.
$$
\begin{defn}
Число $M \in \bbR$ "--- \textit{точная верхняя грань}\rindex{грань! множества, точная} или \textit{супремум множества} $G \subset \bbR$ и обозначается $\sup G$, если 
\begin{enumerate}
\item
$\forall x \in G\cquad x \le M$;
\item
$\forall M' < M \quad \exists x\in G\cquad x>M'$.
\end{enumerate}
Аналогично, число $M \in \bbR$ "--- \textit{точная нижняя грань} или \textit{инфимум множества} $G \subset \bbR$ и обозначается $\inf G$, если 
\begin{enumerate}
\item
$\forall x \in G\quad x\ge M$;
\item
$\forall M' > M\quad \exists x\in G\cquad x<M'$.
\end{enumerate}
\end{defn}

Eсли множество $G \subset \bbR$ не ограничено сверху (снизу), то, по определению, $\sup G=+\infty$ (соотв., $\inf G=-\infty$).  

\begin{defn}
Число $M \in \bbR$ называется \textit{максимальным} (\textit{минимальным}) \textit{элементом множества} $G \subset \bbR$ и обозначается $\max G$ $(\min G)$, если 
\begin{enumerate}
\item
$M\in G$;
\item
$\forall x \in G\quad x \le M \quad (\text{соотв., } x \ge M)$.
\end{enumerate}
\end{defn}

\section{Последовательности и пределы}
\begin{defn}
Пусть имеется правило, которое каждому натуральному числу $n$ ставит в соответствие некоторое $x_n$ из множества $G$. Тогда множество всевозможных упорядоченных пар $(n, x_n)$, $n \in \bbN$, называется \textit{последовательностью} и обозначается либо $\{x_n\}$, либо $x_n$, $n \in \bbN$, либо $x_1$,~$x_2$,~\dots, $x_n$,~\dots
\end{defn}

Пара $(n, x_n)$ называется \textit{$n$-м элементом} этой последовательности и обозначается просто $x_n$. Число $n$ называется \textit{номером}, а число $x_n$ "--- \textit{значением} $n$-го элемента. Множество элементов последовательности всегда бесконечно. Два различных элемента последовательности могут иметь одно и то же значение, но заведомо отличаются номерами, которых бесконечно много. Множество же значений элементов последовательности может быть бесконечным, так и конечным, в частности, состоять из одного элемента. 

Пока мы будем рассматривать лишь последовательности со значениями из $\bbR$ и называть их \textit{числовыми последовательностями}\rindex{последовательность!числовая} или просто последовательностями, поэтому можно считать $G=\bbR$ в данном определении. 
 
\begin{defn}
Последовательность действительных чисел $\{x_n\}$ называется \textit{ограниченной сверху}, если существует число $M$ такое, что $x_n \le M$ для любого $n \in \bbN$. Аналогично, последовательность $\{x_n\}$ называется \textit{ограниченной снизу}, если выполняется условие:
$$
\exists m \in \bbR\cquad \forall n \in \bbN\quad x_n \ge m.
$$
Последовательность называется \textit{ограниченной}\rindex{последовательность!ограниченная}, если она ограничена и сверху, и снизу: 
$$
\exists M\in [0;+\infty)\cquad \forall n \in \bbN\quad |x_n| \le M.
$$
\end{defn}

\begin{defn}
Последовательность $\{x_n\}$ называется \textit{монотонно возрастающей (убывающей)}, если 
$$
\forall n \in \bbN \quad x_n \le x_{n+1} \quad (\text{соотв., } x_n \ge x_{n+1}).
$$
Последовательность $\{x_n\}$ называется \textit{строго возрастающей \textup{(}убывающей\textup{)}}, если 
$$
\forall n \in \bbN \quad x_n < x_{n+1} \quad (\text{соотв., } x_n > x_{n+1}).
$$

Будем называть последовательность \textit{монотонной}\rindex{последовательность!монотонная}, если она либо монотонно возрастает, либо монотонно убывает. 
\end{defn}

\begin{defn}
Пусть $c \in \bboR$, тогда 
\begin{itemize}[wide, labelwidth=!, labelindent=0pt, nolistsep]
\item
\textit{окрестностью} числа $c \in \bbR$ называется любой интервал $(a;b)$, содержащий в себе элемент $c$: $(a;b)\ni c$, $(a;b) \subset \bbR$.
\item
\textit{окрестностью} элемента $+\infty$ называется любой луч $(a;+\infty)$, $a \in \bbR$.
\item
\textit{окрестностью} элемента $-\infty$ называется любой луч $(-\infty;b)$, $b \in \bbR$.
\end{itemize}
\end{defn}

\begin{defn}
Пусть $\epsilon > 0$ и $c \in \bboR $, тогда если $c \in \bbR$, то
\begin{itemize}[wide, labelwidth=!, labelindent=0pt, nolistsep]
\item
\textit{$\epsilon$-окрестностью}~$O_\epsilon$ числа $c$ называется интервал $(c-\epsilon;c+\epsilon) = O_\epsilon.$
\item
Если $c=+\infty$, то \textit{$\epsilon$-окрестностью} $+\infty$ называется луч $(\epsilon;+\infty)$.
\item
Если $c=-\infty$, то $O_\epsilon(-\infty)=(-\infty;-\epsilon)$.
\end{itemize}
\end{defn}
Всякая $\epsilon$-окрестность элемента $c \in \bboR$ является его окрестностью, но не наоборот.

\begin{defn}
Число или бесконечно удаленная точка $c \in \bboR$ называется \textit{пределом}\rindex{предел!последовательности} последовательности $\{x_n\}$, если выполняется условие:
$$
\forall O(c)\quad \exists M \in \bbN\cquad \forall n \ge M\quad x_n \in O(c).
$$
Обозначается $\lim_{n \to \infty}\limits x_n = c$.
\end{defn}

Заметим, что число $c$ не будет являться пределом $\{x_n\}$, если 
$$
\exists O(c)\cquad \forall M \in \bbN\quad \exists n > M\cquad x_n \notin O(c).
$$
\begin{lemm}
Число $x_0$ является пределом последовательности $\{x_n\}$ тогда и только тогда, когда выполняется условие:
\begin{equation}
\label{ch1:eq:predel}
\forall \epsilon>0\quad \exists N_\epsilon \in \bbN\cquad \forall n \ge N_\epsilon\quad x_n \in O_\epsilon(x_0).
\end{equation}
\end{lemm}
Заметим, что условие~\eqref{ch1:eq:predel} часто записывают так:
$$
\forall \epsilon>0\quad \exists N_\epsilon \in \bbN\cquad \forall n \ge N_\epsilon\quad |x_n-x_0|<\epsilon.
$$
\begin{defn}
Последовательность называется \textit{сходящейся}, если она имеет конечный предел. Если же последовательность не имеет конечного предела, то она называется \textit{расходящейся}.
\end{defn}
В дальнейшем будем говорить <<последовательность сходится>>, имея в виду, что она имеет конечный предел. Если же ее предел будет равен $\pm\infty$, будем отдельно отмечать <<последовательность сходится к $\pm\infty$>>, однако такие последовательности являются расходящимися, поэтому иногда говорят, что они расходятся к $\pm\infty$. 
\begin{thm}[о трех последовательностях] \label{ch1:th:otrehposled}  
Пусть числовые последовательности $\{x_n\}$, $\{y_n\}$ и $\{z_n\}$ удовлетворяют условиям:
$$
\exists N_0 \in \bbN\cquad \forall n\ge N_0 \quad x_n \le y_n \le z_n.
$$
Тогда, если $\{x_n\}$ и $\{z_n\}$ сходятся и их пределы равны, то $\{y_n\}$ тоже сходится к тому же пределу.
\end{thm}

\begin{defn}
Последовательность отрезков $\{[a_n;b_n]\}$, $n \in \bbN$, называется \textit{последовательностью вложенных отрезков}\rindex{последовательность!вложенных отрезков}, если
$$
[a_{n+1};b_{n+1}]\subset [a_n;b_n]\quad \forall n \in \bbN.
$$
\end{defn}

\begin{thm}[Кантора\rindex{теорема!Кантора}]
Любая последовательность вложенных отрезков имеет общую точку.
\end{thm}

\begin{defn}
Последовательность вложенных отрезков называется \textit{стягивающейся}, или \textit{последовательностью стягивающихся отрезков}\rindex{последовательность!стягивающихся отрезков} если последовательность длин этих отрезков сходится к нулю.
\end{defn}

\begin{thm}
\label{ch1:th:poslstyag}
Любая последовательность стягивающихся отрезков имеет единственную общую точку.
\end{thm}

Теорему~\ref{ch1:th:poslstyag} можно сформулировать следующим образом: 
\begin{thmn}
Любая последовательность стягивающихся отрезков стягивается к некоторой точке.
\end{thmn}

\section{Теорема Больцано"--~Вейерштрасса}
\begin{defn}
Последовательность $\{y_k\}$ называется \textit{подпоследовательностью}\rindex{подпоследовательность} последовательности $\{x_n\}$, если 
$$
\forall k \in \bbN \quad \exists n=n_k\cquad y_k=x_{n_k},
$$
где последовательность $\{n_k\}$ строго возрастающая. Эта подпоследовательность обозначается $\{x_{n_k}\}$.
\end{defn}

\begin{defn}
Предел любой подпоследовательности данной последовательности называется \textit{частичным пределом}\rindex{предел!частичный} этой последовательности.
\end{defn}

\begin{thm}[Больцано"--~Вейерштрасса]\label{ch1:th:TBV}
\rindex{теорема!Больцано"---Вейерштрасса} У любой ограниченной последовательности существует сходящаяся подпоследовательность.
\end{thm}
\begin{proof}
Пусть последовательность $\{x_n\}$ ограничена, т.е.~существуют числа $a$ и $b$ такие, что $a \le x_n \le b$\quad$\forall n$. Точкой
$c_0 = (a + b)/2$ отрезок~$[a; b]$ разделим на два равных по длине отрезка~$[a; c_0]$ и~$[c_0,b]$. Тогда хотя бы в одном из них лежат значения бесконечного множества элементов последовательности $\{x_n\}$. Через $[a_1;b_1]$ обозначим отрезок $[c_0;b]$, если он содержит значения бесконечного множества элементов последовательности, в противном случае через $[a_1; b_1]$ обозначим отрезок $[a; c_0]$. Отрезок $[a_1, b_1]$ точкой $c_1 = (a_1 + b_1)/2$ снова разделим на два отрезка $[a_1;c_1]$ и $[c_1;b_1]$, и через $[a_2;b_2]$ обозначим отрезок $[c_1;b_1]$, если он содержит значения бесконечного множества элементов последовательности $\{x_n\}$, и отрезок $[a_1; c_1]$ в противном случае. Таким образом, делением пополам, строится последовательность вложенных отрезков $[a_k;b_k]$, $k \in \bbN$, каждый из которых содержит значения бесконечного множества членов последовательности~$\{x_n\}$, причем
$$
\lim_{k\to \infty}(b_k-a_k)=\lim_{k \to \infty}\frac{(b-a)}{2^k}=0.
$$
Следовательно (см.~теорему~\ref{ch1:th:poslstyag}), эти отрезки имеют одну общую точку
$$
c = \lim_{k \to \infty} a_k = \lim_{k \to \infty} b_k.
$$
Построим подпоследовательность $\{x_{n_k}\}$ следующим образом:
\begin{gather*}
\begin{aligned}
& x_{n_1} \in [a_1;b_1];\\
& x_{n_2} \in [a_2;b_2],\quad n_2 \ge n_1;\\
& \ldots\\
& x_{n_k} \in [a_k;b_k],\quad n_k \ge n_{k-1};\\
& x_{n_{k+1}} \in [a_{k+1};b_{k+1}],\quad n_{k+1} \ge n_k;\\
& \ldots
\end{aligned}
\end{gather*}
Так как $a_k \le x_{n_k} \le b_k$\quad$\forall k$, то из \hyperref[ch1:th:otrehposled]{теоремы о трех последовательностях} следует, что $\lim_{k \to \infty}\limits x_{n_k} = c$.

Теорема доказана.
\end{proof}

\hyperref[ch1:th:TBV]{Теорему Больцано"--~Вейерштрасса} можно сформулировать следующим образом: 
\begin{thmn}[Больцано"--~Вейерштрасса] Любая ограниченная последовательность имеет хотя бы один частичный предел.
\end{thmn}

\section{Критерий Коши сходимости числовой последовательности}

\begin{defn}
Последовательность $\{x_n\}$ \textit{фундаментальна}\rindex{последовательность!фундаментальная}, если она удовлетворяет \textit{условию Коши}:
\begin{equation}
\label{ch1:eq:fundamen}
\forall \epsilon >0 \quad \exists N_{\epsilon} \in \bbN\cquad \forall m,n > N_{\epsilon}\quad |x_n-x_m|<\epsilon  
\end{equation}
\end{defn}

\begin{lemm}
\label{ch1:lm:fundamendal}
Если последовательность $\{x_n\}$ фундаментальна, то она ограничена.
\end{lemm}
\begin{proof}
Пусть $\{x_n\}$ фундаментальна, т.е.~удовлетворяет \eqref{ch1:eq:fundamen}. Положим $\epsilon = 1$, $m = N_1$. Тогда $$
\forall n > N_1\quad |x_n-x_{N_1}| <\epsilon = 1,
$$ 
т.е.~$x_{N_1} - 1< x_n < x_{N_1}+1$. Если теперь через $a$ и $b$ обозначим соответственно наименьшее и наибольшее из чисел $x_1$,~\dots, $x_{N_1}$,~$x_{N_1}-1$,~$x_{N_1}+1$, то, очевидно, $a \le x_n \le b$\quad$\forall n \in \bbN$.

Лемма доказана.
\end{proof}

\begin{thm}[Критерий Коши\rindex{критерий!Коши}] 
$\{x_n\}$ сходится $\Longleftrightarrow$ $\{x_n\}$ фундаментальна, где $\{x_n\}$ "--- числовая последовательность.
\end{thm}
\begin{proof}\leavevmode
\begin{itemize}[wide, labelwidth=!, labelindent=0pt]
\item[$\Longrightarrow$:]

Пусть последовательность $\{x_n\}$ сходится, т.е.~ $\exists x_{0} \in \bbR$:  $\lim_{n \to \infty}\limits x_n = x_0$. Тогда 
$$
\forall \epsilon > 0\quad \exists N_{\epsilon/2} \in \bbN\cquad \forall n \ge N_{\epsilon/2}\quad |x_n-x_0|<\epsilon/2.
$$
Отсюда следует, что если $n \ge N_{\epsilon/2}$ и $m \ge N_{\epsilon/2}$, то
$$
|x_n-x_m| \le |x_n-x_0|+|x_0-x_m| <\frac{\epsilon}{2}+\frac{\epsilon}{2}=\epsilon
$$

\item[$\Longleftarrow$:]
Пусть $\{x_n\}$ фундаментальна. Тогда, согласно лемме~\ref{ch1:lm:fundamendal} последовательность $\{x_n\}$ ограничена. Следовательно, по \hyperref[ch1:th:TBV]{теореме Больцано"--~Вейерштрасса} у нее есть сходящаяся подпоследовательность $\{x_{n_k}\}$, причем $\exists x_0 \in \bbR$: $\lim_{k \to \infty}\limits x_{n_k} =x_0 $. Докажем, что $\lim_{n \to \infty}\limits x_{n}=x_0 $.

Зададим некоторое $\epsilon > 0$. Тогда
\begin{gather*}
\begin{aligned}
& \exists N_\epsilon\cquad \forall n,m \ge N_\epsilon &&|x_n-x_m|<\epsilon/2\\
& \exists K_\epsilon\cquad \forall k \ge K_\epsilon   &&|x_{n_k}-x_0|<\epsilon/2.
\end{aligned}
\end{gather*}
Положим $p=\max\{N_\epsilon;K_\epsilon\}$. Тогда, очевидно, $p \ge K_\epsilon$, $n_p \ge p \ge N_\epsilon$ и, следовательно, для любого $n \ge N_\epsilon$
$$
|x_n-x_0| \le |x_n-x_{n_p}|+|x_{n_p}-x_0|<\frac{\epsilon}{2}+\frac{\epsilon}{2}=\epsilon.
$$
А так как $\epsilon > 0$ любое, то этим доказано, что $\lim_{n \to +\infty}\limits x_n = x_0$. \qedhere
\end{itemize}
\end{proof}
