\chapter{Системы линейных однородных дифференциальных уравнений с постоянными коэффициентами, методы их решения.}
\section{Системы линейных однородных дифференциальных уравнений с постоянными коэффициентами, методы их решения}
\begin{defn}\label{ch27defn1}
\textit{Нормальной линейной системой с постоянными коэффициентами} порядка $n\ge 2$ называют систему линейных дифференциальных уравнений вида
\begin{equation}\label{27eq1.1}
\dot x_i(t)=\sum\limits_{j=1}^n a_{ij}x_j(t)+f_i(t),\quad i=\overline{1,n}.
\end{equation}
Здесь: $t$ "--- аргумент; $x_1(t),...,x_n(t)$ "--- неизвестные функции; $a_{ij}$ "--- заданные комплексные или действительные числа, называемые \textit{коэффициентами} системы, $i,j=\overline{1,n}$; $f_1(t),...,f_n(t)$ "--- заданные комплексные функции, называемые свободными членами системы; $\dot x_i(t)=\frac{dx_i}{dt},\; i=\overline{1,n}$. Будем всегда считать, что функции $f_1(t),...,f_n(t)$ "--- заданные непрерывные функции на некотором промежутке $T$ числовой оси $\bbR$.
\end{defn}
Заметим, что число уравнений системы \eqref{27eq1.1} равно числу неизвестных функций.

Упростим запись. Пусть
$$
A=||a_{i,j}||,\quad i,j=\overline{1,n},
$$
$$
x(t)=\begin{pmatrix}
x_1(t)\\ \vdots \\ x_n(t)
\end{pmatrix},\;
\dot x(t)=\begin{pmatrix}
\dot x_1(t)\\ \vdots \\ \dot x_n(t)
\end{pmatrix},\;
f(t)=\begin{pmatrix}
f_1(t)\\ \vdots \\ f_n(t)
\end{pmatrix}.
$$
Тогда нормальная линейная система \eqref{27eq1.1} записывается в виде одного уравнения 
\begin{equation} \label{27eq1.2}
\dot x(t)=Ax(t)+f(t).
\end{equation}
Линейная система \eqref{27eq1.2} называется \textit{линейной однородной системой}, если $f(x)\equiv 0$ на промежутке $T$. В противном случае она будет называться \textit{линейной неоднородной системой}. \textit{Решением} нормальной линейной системы \eqref{27eq1.2} будем называть всякую вектор-функцию  $x=\phi(t)$ с $n$ комплекснозначными непрерывно дифференцируемыми на промежутке $T$ компонентами $\phi_1(t),...,\phi_n(t)$, если $\dot \phi(t)\equiv A\phi(t)+f(t)$ на промежутке $T$.

Рассмотрим нормальную линейную однородную систему
\begin{equation}\label{27eq2.1}
\dot x(t) = Ax(t),
\end{equation}
где $t\in\bbR$, $A$ - квадратная комплексная матрица порядка $n$, $x(t)$ "--- неизвестная вектор-функция с $n$ компонентами.

\begin{lemm}\label{27lemm1} (принцип суперпозиции)
Если $x^{(1)}(t), x^{(2)}(t)$ "--- решения системы \eqref{27eq2.1}, а  $C_1,$ $C_2$ "--- произвольные комплексные числа, то вектор-функция $x(t)=C_1x^{(1)}(t)+C_2x^{(2)}(t)$ также решение системы \eqref{27eq2.1}.
\end{lemm}
\begin{proof}
В силу условий леммы имеем
\begin{multline*}
\dot x(t)-Ax(t)=C_1\dot x^{(1)}(t)+C_2\dot{x}^{(2)}(t)-A\left[C_1{x}^{(1)}(t)+C_2{x}^{(2)}(t)\right]=\\=C_1\left[\dot{x}^{(1)}-Ax^{(1)}\right]+C_2\left[\dot{x}^{(2)}-Ax^{(2)}\right]=0.\tag*{\qedhere}
\end{multline*}
\end{proof}
Будем считать в дальнейшем, что матрица $A$ является матрицей линейного преобразования $\mathcal A$ в комплексном унитарном $n$-мерном пространстве $\bbR^n$ столбцов с $n$ компонентами в ортонормированном базисе $e_1,e_2,...,e_n$. При заданном базисе можно отождествить преобразование $\mathcal A$ и его матрицу $A$.

Очевидно, что система \eqref{27eq2.1} имеет тривиальное решение $x=0$. Будем искать нетривиальные решения \eqref{27eq2.1} в виде $x(t)=e^{\lambda t}h$, где $h\neq 0$ "--- числовой $n$-мерный вектор. Подставляя $x(t)$ в систему \eqref{27eq2.1}, получим $\lambda e^{\lambda t}h=Ae^{\lambda t}h$ или $Ah=\lambda h$.

Напомним, что собственный вектор $h$ преобразования $A$ для собственного значения $\lambda$ определяется условием
$$
Ah=\lambda h,\quad h\neq 0,
$$
и что все собственные значения $\lambda$ преобразования $A$ являются корнями уравнения
$$
\det(A-\lambda E)=0,
$$
где $E$ "--- единичная матрица порядка $n$. Таким образом, установлено следующие утверждение.

\begin{lemm}\label{27lemm2}
Для того, чтобы вектор-функция $x(t)=e^{\lambda t}h$ была нетривиальным решением линейной однородной системы \eqref{27eq2.1}, необходимо и достаточно, чтобы $\lambda$ было собственным значением, а $h$ "--- соответствующим ему собственным вектором преобразования $A$.
\end{lemm}

\begin{thm}\label{27thm1}
Пусть существует базис $\bbR^n$ из собственных векторов $h_1,...,h_n$ линейного преобразования $A$ и пусть $\lambda_1,...,\lambda_n$ "--- соответствующие им собственные значения (среди них могут быть одинаковые)

Тогда:

а) вектор-функция $x(t)$ вида
\begin{equation}\label{27eq2.2}
x(t)=C_1e^{\lambda_1 t}h_1+...+C_ne^{\lambda_n t}h_n,
\end{equation}
где  $C_1,...,C_n$ "--- произвольные комплексные постоянные, являются решением системы \eqref{27eq2.1};

б) если $x(t)$ "--- какое-либо решение системы \eqref{27eq2.1}, то найдутся такие значения постоянных $C_1,...,C_n$, при которых $x(t)$ задается формулой \eqref{27eq2.2}.
\end{thm}
\begin{proof}
П. а)  следует из леммы \ref{27lemm1} и леммы \ref{27lemm2}. 

Докажем п. б). Пусть $x(t)$ "--- какое-либо решение \eqref{27eq2.1}. Так как $h_1,...,h_n$ "--- базис $\bbR^n$, то $\forall t\in \bbR$
$$
x(t)=\zeta_1(t)h_1+...+\zeta_n(t)h_n.
$$
Подставим $x(t)$ в систему \eqref{27eq2.1}. Имеем
$$
\dot{\zeta}_1(t)h_1+...+\dot\zeta_n(t)h_n=\zeta_1(t)Ah_1+...+\zeta_n(t)Ah_n=\lambda_1\zeta_1(t)h_1+...+\lambda_n\zeta_n(t)h_n.
$$
Так как $h_1,...,h_n$ "--- линейно независимые векторы, то отсюда
$$
\dot{\zeta}_1(t)=\lambda_1\zeta_1(t),...,\dot{\zeta}_n(t)=\lambda_n\zeta_n(t).
$$
Из этих уравнений находим, что $\zeta_1(t)=C_1e^{\lambda_1t},...,\zeta_n(t)=C_ne^{\lambda_n t}$. Подстановка найденных $\zeta_1(t),...,\zeta_n(t)$ в формулу для $x(t)$ дает \eqref{27eq2.2}.
\end{proof}

Как известно из курса алгебры, базис пространства $\bbR^n$ из собственных векторов преобразования $A$ существует, например, тогда, когда все собственные значения $\lambda_1,...,\lambda_n$ преобразования $A$ попарно различны или когда преобразование $A$ является нормальным (независимо от кратности $\lambda$), в частности, симметрическим.

Если несимметрическое преобразование $A$ имеет собственное значение  $\lambda$ кратности $k$, то, вообще говоря, линейно независимых собственных векторов $A$, соответствующих этому значению $\lambda$, оказывается меньше $k$. Таким образом, в таком случае нельзя построить базис $\bbR^n$ из собственных векторов преобразования $A$. Но можно построить \textit{жорданов базис} пространства $\bbR^n$

\begin{defn}
Пусть $\lambda_0$ "--- собственное значение преобразования $A$ и пусть векторы $h_1,h_2,...,h_k$ таковы, что 

\begin{equation} \label{27eq2.3}
\begin{aligned}
&Ah_1=\lambda_0h_1,\quad h_1\neq 0,\\
&Ah_2=\lambda_0h_2+h_1,\\
&..................................\\
&Ah_k=\lambda_0h_k+h_{k-1}.
\end{aligned}
\end{equation}

Тогда $h_1$ "--- собственный вектор преобразования $A$, а векторы $h_2,...,h_k$ называют \textit{присоединенными векторами} к вектору $h_1$. Система векторов $h_1,...,h_k$ называется \textit{жордановой цепочкой} для собственного значения $\lambda_0$, а число $k$ называется длиной жордановой цепочки.

Если собственное значение $\lambda_0$ "--- простое и $h_1$ "--- соответствующий ему собственный вектор, то присоединенных векторов к $h_1$ в этом случае не существует. Если же $\lambda_0$ "--- кратное собственное значение, то для него может существовать несколько жордановых цепочек, содержащих линейно независимые собственные векторы преобразования.   

\end{defn}

\begin{thm}\label{27thmJ} (Жордана)
Каково бы ни было линейное преобразование $A$ в комплексном пространстве $\bbR^n$, всегда существует базис $\bbR^n$, составленный из жордановых цепочек для всех собственных значений.
\end{thm}

Отметим, что 
\begin{itemize} 
\item в жордановом базисе $\bbR^n$ (базисе, составленном из жордановых цепочек) число различных жордановых цепочек равно числу линейно независимых векторов преобразования $A$;
\item в жордановом базисе сумма длин всех жордановых цепочек для каждого кратного собственного значения $\lambda$ равна кратности $\lambda$;
\item в общем случае жорданов базис $\bbR^n$ является комплексным даже для действительного преобразования $A$;
\item жорданов базис может быть базисом, состоящим только из собственных векторов $A$;
\item жорданов базис $\bbR^n$ строится не единственным образом.
\end{itemize}

Пусть $A$ "--- произвольная квадратная матрица, $\lambda$ "--- собственное значение $A$ и пусть $h_1,...,h_k$ "--- некоторая жорданова цепочка для $\lambda$. Покажем, что каждой жордановой цепочке длины $k$ соответствует $k$ решений системы \eqref{27eq2.1} вида

\begin{equation}\label{27eq2.4}
\begin{aligned}
&x_1(t)=e^{\lambda t}h_1\equiv e^{\lambda t}\cdot P_1(t),\\
&x_2(t)=e^{\lambda t}\left( \frac{t}{1!}h_1+h_2\right)\equiv e^{\lambda t}\cdot P_2(t),\\
&x_3(t)=e^{\lambda t}\left( \frac{t^2}{2!}h_1+\frac{t}{1!}h_2+h_3\right)\equiv e^{\lambda t}\cdot P_3(t),\\
&.................................................................\\
&x_k(t)=e^{\lambda t}\left(\frac{t^{k-1}}{(k-1)!}h_1+...+\frac{t}{1!}h_{k-1}+h_k\right)\equiv e^{\lambda t}\cdot P_k(t),\\
\end{aligned}
\end{equation}

\begin{lemm}\label{27lemm3}
Каждая из вектор-функций $x_r(t)=e^{\lambda t}\cdot P_r(t),\; r=\overline{1,k}$, является решением системы \eqref{27eq2.1}
\end{lemm}

\begin{proof}
При $k=1$ утверждение леммы доказано в лемме \ref{27lemm2}. Пусть $k\geq 2$. Тогда $\dot{P}_r(t)=P_{r-1}(t)$, а из определения жордановой цепочки \eqref{27eq2.3} следует, что $AP_r(t)=\lambda P_r(t)+P_{r-1}(t)$. Подставляя $x_r(t)$ в систему \eqref{27eq2.1}, получаем, что
\begin{multline*}
\dot{x}_r(t)-Ax_r=\lambda e^{\lambda t}P_r+e^{\lambda t}\dot{P}_r-e^{\lambda t}AP_r=\\=\lambda e^{\lambda t} P_r+e^{\lambda t}P_{r-1}-e^{\lambda t}(\lambda P_r+P_{r-1})=0
\end{multline*}
Здесь был использован тот факт, что формула производной произведения скалярной функции и вектор-функции аналогична формуле производной произведения двух скалярных функций.
\end{proof}


\begin{thm}\label{27thm2}
Пусть жорданов базис $\bbR^n$ состоит из $S$ жордановых цепочек $h_1^{(j)},..,h_{k_j}^{(j)}$ длин $k_j\; (k_1+..+k_S=n)$ для собственных значений $\lambda_j$ (среди $\lambda_j$ могут быть одинаковые) преобразования $A$, $j=\overline{1,S}$. Тогда:

а) вектор-функция $x(t)$ вида
\begin{equation}\label{27eq2.5}
x(t)=\sum\limits_{j=1}^S e^{\lambda_j t}\left[C_1^{(j)}P_1^{(j)}(t)+...+C_{k_j}^{(j)}P_{k_j}^{(j)}(t)\right],
\end{equation}
где $P_1^{(j)}(t),...,P_{k_j}^{(j)}(t)$ "--- многочлены вида \eqref{27eq2.4} и $C_1^{(j)},...,C_{k_j}^{(j)},\; j=\overline{1,S}$ "--- произвольные комплексные постоянные, является решением системы \eqref{27eq2.1}.

б) если $x(t)$ "--- какое-либо решение системы \eqref{27eq2.1}, то найдется такой набор значений постоянных $C_1^{(j)},...,C_{k_j}^{(j)},\; j=\overline{1,S}$, при котором $x(t)$ задается формулой \eqref{27eq2.5}.
\end{thm}

\begin{proof}
П. а) следует из леммы \ref{27lemm3} и леммы \ref{27thm1}. 

Докажем п. б).Пусть $x(t)$ "--- какое-либо решение системы \eqref{27eq2.1}. Покажем, что оно имеет вид \eqref{27eq2.5}. При каждом $t\in\bbR$ решение $x(t)$ можно разложить по базису $\bbR^n$. Пусть
$$
x(t)=\sum\limits_{j=1}^S\left[\zeta_1^{(j)}(t)h_1^{(j)}+...+\zeta_{k_j}^{(j)}(t)h_{k_j}^{(j)}\right].
$$
Подставим $x(t)$ в систему \eqref{27eq2.1} и воспользуемся определением жордановой цепочки \eqref{27eq2.3}. Имеем:
\begin{multline*}
\sum\limits_{j=1}^S\left[\dot\zeta_1^{(j)}(t)h_1^{(j)}+...+\dot\zeta_{k_j}^{(j)}(t)h_{k_j}^{(j)}\right]=\sum\limits_{j=1}^S\left[\zeta_1^{(j)}(t)Ah_1^{(j)}+...+\zeta_{k_j}^{(j)}(t)Ah_{k_j}^{(j)}\right]=\\=\sum\limits_{j=1}^S\left[\zeta_1^{(j)}(t)\lambda_jh_1^{(j)}+\zeta_2^{(j)}(t)\left(\lambda_jh_2^{(j)}+h_1^{(j)}\right)+...+\zeta_{k_j}^{(j)}(t)\left(\lambda_jh_{k_j}^{(j)}+h_{k_j-1}^{(j)}\right)\right].
\end{multline*}
Из единственности разложения $x(t)$ по жордановому базису отсюда находим $S$ систем вида
$$
\begin{cases}
\dot{\zeta}_1^{(j)}=\lambda_j\zeta_1^{(j)}+\zeta_2^{(j)},\\
..................................\\
\dot{\zeta}_{k_j-1}^{(j)}=\lambda_j\zeta_{k_j-1}^{(j)}+\zeta_{k_j}^{(j)},\\
\dot{\zeta}_{k_j}^{(j)}=\lambda_j\zeta_{k_j}^{(j)},\quad j=\overline{1,S}
\end{cases}
$$
Решая каждую из этих систем снизу вверх, получаем:
$$
\begin{aligned}
\zeta_{k_j}^{(j)}(t)&=C_{k_j}^{(j)}e^{\lambda_jt},\\
\zeta_{k_j-1}^{(j)}(t)&=\left[C_{k_j-1}^{(j)}+C_{k_j}^{(j)}\frac{t}{1!}\right]e^{\lambda_j t},\\
...........&...........................................\\
\zeta_1^{(j)}(t)&=\left[C_1^{(j)}+C_2^{(j)}\frac{t}{1!}+...+C_{k_j}^{(j)}\frac{t^{k_j-1}}{(k_j-1)!}\right]e^{\lambda_jt},\quad j=\overline{1,S}.
\end{aligned}
$$
Подставляя найденные значения $\zeta_1^{(j)}(t),...,\zeta_{k_j}^{(j)}(t)$ в разложение $x(t)$ и собирая члены возле каждого $C_1^{(j)},...,C_{k_j}^{(j)}$, получим представление $x(t)$ в виде \eqref{27eq2.5}.
\end{proof}







