\chapter{Линейные обыкновенные дифференциальные уравнения с переменными коэффициентами. Фундаментальная система решений. Формула Лиувилля-Остроградского. Определитель Вронского.}

\section{Линейные обыкновенные дифференциальные уравнения с переменными коэффициентами}

Линейным дифференциальным уравнением порядка $n$ с переменными коэффициентами называется уравненеие 
\begin{equation} \label{ch28.1eq1}
y^{(n)} + a_1(x)y^{(n - 1)} + \ldots + a_n(x)y = f(x),
\end{equation}
где $x \in [\alpha, \beta], \: a_j(x), \: j = \overline{1,n},$ "--- заданные непрерывные функции на $[\alpha, \beta]$, называемые коэффициентами уравнения $(\ref{ch28.1eq1})$, и $f(x)$ "--- заданная непрерывная на $[\alpha, \beta]$ функция, называемая правой частью уравнения $(\ref{ch28.1eq1})$. При $f(x) \equiv 0$ на $[\alpha, \beta]$, уравнение "--- однородное, в противном случае "--- неоднородное. $a_j(x), \: j = \overline{1,n}, \; f(x)$ могут быть комплексными.

Функция $y = \phi(x)$ называется решением уравения $(\ref{ch28.1eq1})$ на $[\alpha, \beta]$, если $\phi(x) \; n$ раз непрерывно дифференцируема на $[\alpha, \beta]$ и обращает $(\ref{ch28.1eq1})$ в тождество на всем $[\alpha, \beta]$.

\begin{lemm}
Если $f(x) = f_1(x) + f_2(x)$ и $y_i(x)$ "--- решение уравнения $(\ref{ch28.1eq1})$ при $f(x) \equiv f_i(x)$ на $[\alpha, \beta], \: i = 1,2$, то функция $y(x) = y_1(x) + y_2(x)$ является решением уравнения $(\ref{ch28.1eq1})$.
\end{lemm}  

\begin{cons}
Если $y_1(x), y_2(x)$ "--- решения линейного однородного уравнения и $c_1, c_2$ "--- произвольные числа, то линейная комбинация $y = c_1y_1(x) + c_2y_2(x)$ также является решением линейного однородного уравнения.
\end{cons}

Решение уравнения \eqref{ch28.1eq1} всегда можно свести к решению линейной системы дифференциальных уравнений порядка $n$ следующего вида:

\begin{equation} \label{ch28.1eq2}
y'(x) = A(x)y(x) + f(x),
\end{equation}
где
\begin{multline}
y(x) = \begin{pmatrix}
y_1(x) \\
\vdots \\
y_n(x)
\end{pmatrix},
A(x) = \begin{pmatrix}
0 & 1 & 0 & \ldots & 0 \\
0 & 0 & 1 & \ldots & 0 \\
\hdotsfor{5} \\
0 & 0 & 0 & \ldots & 1 \\
-a_n(x) & -a_{n - 1}(x) & -a_{n - 2}(x) & \ldots & -a_1(x)
\end{pmatrix}, \\
f(x) = \begin{pmatrix}
0 \\
\vdots \\ 
0 \\
f(x)
\end{pmatrix}.
\end{multline}

\begin{lemm} \label{ch28.1lemm2}
Уравнение \eqref{ch28.1eq1}) эквивалентно системе $(\ref{ch28.1eq2})$.
\end{lemm}

\begin{proof}
Пусть $y = \phi(x)$ "--- решение \eqref{ch28.1eq1}. 

Положим $y_1(x) = \phi(x), \ y_2(x) = \phi'(x), \ldots, y_n(x) = \phi^{(n - 1)}(x)$. Тогда вектор-функция с компонентами $\phi(x),\ \phi'(x),\ldots,\ \phi^{(n - 1)}(x)$ удовлетворяет системе $(\ref{ch28.1eq2})$. Наоборот, если вектор-функция с компонентами $\phi(x), \phi'(x), \ldots, \phi^{(n - 1)}(x)$ "--- решение системы $(\ref{ch28.1eq1})$, то, исключив из $(\ref{ch28.1eq2})$ переменные $y_2, \ldots, y_n$, получаем, что $y_1 = \phi(x)$ "--- решение уравнения $(\ref{ch28.1eq1})$.
\end{proof}

Лемма \ref{ch28.1lemm2} позволяет  перенести все результаты для линейных систем на случай уравнения $(\ref{ch28.1eq1})$.

Рассмотрим для уравнения $(\ref{ch28.1eq1})$ начальные условия
\begin{equation} \label{ch28.1eq3}
y(x_0) = y_1^{(0)}, \; y'(x_0) = y_2^{(0)}, \; \ldots, \; y^{(n - 1)}(x_0) = y_n^{(0)},
\end{equation}
где $x_0 \in [\alpha, \beta]$ и $y_1^{(0)}, \ldots, y_n^{(0)}$ "--- заданные числа.

\begin{thm}
Пусть все функции $a_j(x), \: j = \overline{1,n}$ и $f(x)$ "--- непрерывны на $[\alpha, \beta]$ и пусть $x_0 \in [\alpha, \beta]$. Тогда при произвольных начальных значениях $y_1^{(0)}, \ldots, y_n^{(0)}$ решение задачи Коши $(\ref{ch28.1eq1}), \: (\ref{ch28.1eq3})$ существует и единственно на всем $[\alpha, \beta]$.
\end{thm}

\begin{proof}
Сделав замену 
$$
y_1(x) = y(x), \; y_2(x) = y'(x), \; \ldots, \; y_n(x) = y^{(n - 1)}(x),
$$
сведем уравнение \eqref{ch28.1eq1} к системе \eqref{ch28.1eq2}. При этом начальные условия примут вид
\begin{equation} \label{ch28.1eq4}
y(x_0) = y^{(0)},
\end{equation}
где $y^{(0)}$ "--- вектор с компонентами $y_1^{(0)}, \ldots, y_n^{(0)}$. В силу леммы $\ref{ch28.1lemm2}$ задача Коши $(\ref{ch28.1eq1}), \: (\ref{ch28.1eq3})$ эквивалентна задаче Коши $(\ref{ch28.1eq2}), \: (\ref{ch28.1eq4})$. В силу условий теоремы $A(x)$ и $f(x)$ "--- непрерывны на $[\alpha, \beta]$. Следовательно, для задачи Коши $(\ref{ch28.1eq2}), \: (\ref{ch28.1eq4})$ выполнены все условия теоремы о существовании и единственности задачи Коши для линейной системы уравнений. Значит, и решение задачи Коши $(\ref{ch28.1eq1}), \: (\ref{ch28.1eq3})$ существует и единственно на $[\alpha, \beta]$.
\end{proof}
\section{Фундаментальная система решений}

Рассмотрим линейное однородное уравнение порядка $n$
\begin{equation} \label{ch28.2eq1}
y^{(n)} + a_1(x)y^{(n - 1)} + \ldots + a_n(x)y = 0,
\end{equation} 

где $a_j(x), \: j = \overline{1,n}$, заданные непрерывные функции на $[\alpha, \beta]$.

\begin{defn}
Решения $y_1(x), \ldots, y_k(x)$ уравнения $(\ref{ch28.2eq1})$ называются \textit{линейно зависимыми} на $[\alpha, \beta]$, если $\exists$ числа $c_1, \ldots, c_k$, одновременно не равные нулю и такие, что 
$$
c_1y_1(x) + \ldots + c_ky_k(x) \equiv 0, \quad \forall x \in [\alpha, \beta].
$$

В противном случае решения $y_1(x), \ldots, y_k(x)$ называются \textit{линейно независимыми} на $[\alpha, \beta]$.
\end{defn}

Рассмотрим линейную однородную систему, которая эквивалентна уравнению $(\ref{ch28.1eq1})$:
\begin{multline} \label{ch28.2eq2}
y'(x) = A(x)y(x), \\
y(x) = \begin{pmatrix}
y_1(x) \\
\vdots \\
y_n(x)
\end{pmatrix},
A(x) = \begin{pmatrix}
0 & 1 & 0 & \ldots & 0 \\
0 & 0 & 1 & \ldots & 0 \\
\hdotsfor{5} \\
0 & 0 & 0 & \ldots & 1 \\
-a_n & -a_{n - 1} & -a_{n - 2} & \ldots & -a_1
\end{pmatrix}.
\end{multline}

\begin{lemm} \label{ch28.2lemm1}
Решения $y_1(x), \ldots, y_k(x)$ уравнения $\eqref{ch28.2eq1}$ линейно зависимы на $[\alpha, \beta]$ тогда и только тогда, когда соответствующие им решения $Y_1(x), \ldots, Y_k(x)$ системы $\eqref{ch28.2eq2}$ линейно зависимы на $[\alpha,\beta]$ (здесь $Y_j(x)$ "--- вектор-функция с компонентами $y_j(x), y'_j(x), \ldots, y^{(n - 1)}_j(x), \: j = \overline{1,k})$.
\end{lemm}

\begin{proof}
Пусть решения $y_1(x), \ldots, y_k(x)$ уравнения $\eqref{ch28.2eq1}$ линейно зависимы на $[\alpha, \beta]$. Тогда найдутся такие числа $c_1, \ldots, c_k, \: |c_1| + \ldots + |c_k| > 0$, что 
$$
c_1y_1(x) + \ldots + c_ky_k(x) \equiv 0, \quad \forall x \in [\alpha, \beta].
$$

Дифференцируя последовательно это тождество $(n - 1)$ раз, получаем тождество на $[\alpha, \beta]$ для решений системы $\eqref{ch28.2eq2}$:
$$
c_1Y_1(x) + \ldots + c_kY_k(x) \equiv 0,
$$
т.е. решения $Y_1(x), \ldots, Y_k(x)$ системы $\eqref{ch28.2eq2}$ линейно зависимы на $[\alpha, \beta]$.

Обратно, если выполнено последнее тождество на $[\alpha, \beta]$ с некоторыми, одновременно не равными нулю, числами $c_1, \ldots, c_k$, то первая компонента этого векторного тождества означает линейную зависимость решений $y_1(x), \ldots, y_k(x)$ уравнения $\eqref{ch28.2eq1}.$
\end{proof}

\begin{cons}
Решения $y_1(x), \ldots, y_k(x)$ уравнения $(\ref{ch28.2eq1})$ линейно независимы на $[\alpha, \beta]$ тогда и только тогда, когда решения $Y_1(x), \ldots, Y_k(x)$ системы $(\ref{ch28.2eq2})$ линейно независимы на $[\alpha, \beta]$.
\end{cons}

\begin{defn}
Совокупность произвольных $n$ независимых решений $\phi_1(x), \ldots, \phi_n(x)$ уравнения $(\ref{ch28.2eq1})$ называется \textit{фундаментальной системой решений уравнения} $(\ref{ch28.2eq1})$.
\end{defn}

Из леммы $\ref{ch28.2lemm1}$ в качестве следствия получаем следующее утверждение.

\begin{lemm} \label{ch28.2lemm2}
Решения $\phi_1(x), \ldots, \phi_n(x)$ уравнения $(\ref{ch28.2eq1})$ образуют фундаментальную систему решений уравнения $(\ref{ch28.2eq1})$ в том и только в том случае, когда вектор-функция $\Phi_j(x)$ с компонентами $\phi_j(x), \ldots, \phi^{(n - 1)}_j(x),\ j =\ \overline{1,n}$, образуют фундаментальную систему решений линейной однородной системы $\eqref{ch28.2eq2}$.
\end{lemm}

С помощью леммы $\ref{ch28.2lemm2}$ все утверждения о фундаментальных системах решений линейной однородной системы переносятся на фундаментальные системы решений линейных однородных уравнений порядка $n$.

\begin{thm} \label{ch28.2thm1}
Для уравнения $(\ref{ch28.2eq1})$ существует бесконечное множество фундаментальных систем решений.
\end{thm}

\begin{proof}
Уравнение $(\ref{ch28.2eq1})$ эквивалентно системе $(\ref{ch28.2eq2})$, для которой справедлив аналог теоремы $\ref{ch28.2thm1}$ для линейной системы с переменными коэффициентами. В силу леммы $\ref{ch28.2lemm2}$ тогда справедлива и теорема $\ref{ch28.2thm1}$.
\end{proof}

\begin{thm} \label{ch28.2thm2}
Если $\phi_1(x), \ldots, \phi_n(x)$ "--- фундаментальная система решений уравнения $(\ref{ch28.2eq1})$, то каждое решение $y(x)$ уравнения $(\ref{ch28.2eq1})$ представимо единственным образом в виде
$$
y(x) = c_1\phi_1(x) + \cdots + c_n\phi_n(x),
$$
где $c_1, \ldots, c_n$ "--- постоянные.
\end{thm}

\begin{proof}
По лемме $\ref{ch28.2lemm2}$ вектор-функции $\Phi_j(x)$ с компонентами $\phi_j(x),\\ \phi'_j(x), \ldots, \phi^{(n - 1)}_j(x), \: j = \overline{1,n}$, образуют фундаментальную систему решений системы $(\ref{ch28.2eq2})$, эквивалетной уравнению $(\ref{ch28.2eq1})$. По теореме, аналогичной теореме $\ref{ch28.2thm2}$, для линейной системы с переменными коэффициентами любое решение $Y(x)$ системы $(\ref{ch28.2eq2})$ единственным образом представимо в виде
$$
Y(x) = c_1 \Phi_1(x) + \cdots + c_n \Phi_n(x).
$$
Первая строка этого векторного равенства и дает утверждение теоремы $\ref{ch28.2thm2}$.
\end{proof}

\section{Определитель Вронского}

Рассмотрим систему 
\begin{equation} \label{ch28.4eq1}
y'(x) = A(x)y(x),
\end{equation}
где $A(x)$ "--- заданная непрерывная на $[\alpha, \beta]$ комплекснозначная квадратная матрица порядка $n$.

\begin{thm} \label{ch28.3thm1}
Пусть $y_j(x), j = \overline{1,k}$ "--- решения линейной однородной системы $(\ref{ch28.4eq1})$. Решения $y_j(x), j = \overline{1,k}$ "--- линейно независимы на $[\alpha, \beta]$ тогда и только тогда, когда $\forall x_0 \in [\alpha, \beta]$ числовые векторы $y_j(x_0), j = \overline{1,k}$ линейно независимы.
\end{thm}

\begin{proof}
Пусть решения $(\ref{ch28.4eq1}) \; y_j(x), j = \overline{1,k}$ "--- линейно независимы. Если $\exists x_0 \in [\alpha, \beta]$, такое, что $y_j(x_0), j = \overline{1,k}$ "--- линейно зависимые векторы, то найдутся числа $c_1, \ldots, c_k, \; |c_1| + \ldots + |c_k| > 0$, такие, что
$$
c_1y_1(x_0) + \ldots + c_k y_k(x_0) = 0
$$

Вектор-функция
$$
y(x) = c_1y_1(x) + \ldots + c_k y_k(x)
$$
является решением системы $(\ref{ch28.4eq1})$ и удовлетворяет начальному условию $y(x_0) = 0$. Тогда $y(x) \equiv 0$ на $[\alpha, \beta]$, т.е. $y_j(x), j = \overline{1,k}$, линейно зависимы. Противоречие.

Наоборот. Пусть $\forall x_0 \in [\alpha, \beta]$ векторы $y_j(x_0), j = \overline{1,k}$, линейно независимы. Если бы вектор-функции $y_j(x), j = \overline{1,k}$, были линейно зависимыми на $[\alpha,\beta]$, то следовало бы, что векторы $y_j(x_0), j = \overline{1,k}$, "--- линейно зависимы. Противоречие.
\end{proof}

Пусть $y_1(x), \ldots, y_n(x)$ "--- система вектор-функций с $n$ компонентами на $[\alpha, \beta]$
\begin{defn}
\textit{Определителем Вронского} (или сокращенно \textit{вронскианом}) системы $y_1(x), \ldots, y_n(x)$ называется определитель
$$
W(x) \equiv W[y_1(x), \ldots, y_n(x)] = \det ||y_1(x)\ \ldots\ y_n(x)||.
$$

Если $y_1(x), \ldots, y_n(x)$ "--- решения линейной однородной системы $(\ref{ch28.4eq1})$, то из теоремы $\ref{ch28.3thm1}$ вытекает следующая связь между линейной зависимостью $y_1(x), \ldots, y_n(x)$ на $[\alpha, \beta]$ и обращением в нуль их определителя Вронского $W(x)$.

\begin{itemize}
\item
Решения $y_1(x), \ldots, y_n(x)$ системы $(\ref{ch28.4eq1})$ линейно зависимы на $[\alpha, \beta]$, тогда и только тогда, когда $W[y_1, \ldots, y_n] \equiv 0$ на $[\alpha, \beta]$.
\item
Решения $y_1(x), \ldots, y_n(x)$ системы $(\ref{ch28.4eq1})$ линейно независимы на $[\alpha, \beta]$, тогда и только тогда, когда $W[y_1, \ldots, y_n] \not = 0$, $\forall x \in [\alpha, \beta]$. Отсюда получаем, что не существует таких $x_1, x_2 \in [\alpha, \beta], \; x_1 \not= x_2$, что $W(x_1) = 0, \; W(x_2) \not= 0$. 
\end{itemize}
\end{defn}

\begin{defn}
\textit{Определителем Вронского} (или сокращенно \textit{вронскианом}) решений $y_1(x), \ldots, y_n(x)$ уравнения $(\ref{ch28.2eq1})$ называется определитель вида
\begin{equation}
\begin{vmatrix}
y_1(x) & \ldots & y_n(x) \\
y'_1(x) & \ldots & y'_n(x) \\
\hdotsfor{3} \\
y^{(n - 1)}_1(x) & \ldots & y^{(n - 1)}_n(x)
\end{vmatrix}
\end{equation}
и обозначается $W(x)$ или $W[y_1(x), \ldots, y_n(x)]$.
\end{defn}

\begin{thm} \label{ch28.3thm3}
Решения $y_1(x), \ldots, y_n(x)$ уравнения $(\ref{ch28.2eq1})$ линейно зависимы тогда и только тогда, когда $W(x) \equiv 0$ на $[\alpha, \beta]$. Решения $y_1(x), \ldots, y_n(x)$ уравнения $(\ref{ch28.2eq1})$ линейно независимы тогда и только тогда, когда $W(x) \not= 0$ для всех $x \in [\alpha, \beta]$.
\end{thm}

\begin{proof}
Сведем уравнение $(\ref{ch28.2eq1})$ к эквивалентной системе $(\ref{ch28.2eq2})$. Тогда столбцы $W(x)$ "--- решения системы $(\ref{ch28.2eq2})$ и, значит, $W(x)$ является определителем Вронского и для решений $(\ref{ch28.2eq2})$. Но для него утверждения теоремы $\ref{ch28.3thm3}$ уже установлены для линейной однородной системы , как свойства определителя Вронского.
\end{proof}

\section{Формула Лиувилля-Остроградского}

\begin{thm}
Пусть $W(x)$ "--- вронскиан решений $y_1(x), \ldots, y_n(x)$ уравнения $(\ref{ch28.4eq1})$ и пусть $x_0 \in [\alpha, \beta]$. Тогда для всех $x \in [\alpha, \beta]$ справедлива формула Лиувилля-Остроградского
\begin{equation} \label{ch28.4eq6}
W(x) = W(x_0) \exp \left\lbrace \int\limits_{x_0}^{x} \sp A(\zeta) \,d\zeta \right\rbrace,
\end{equation}
где $\sp A(\zeta) = a_{11}(\zeta) + \cdots + a_{nn}(\zeta)$ называется следом матрицы $A(\zeta)$.
\end{thm}

\begin{proof}
Покажем, что $W(x)$ удовлетворяет дифференциальному уравнению
$$
W'(x) = \sp A(x) \cdot W(x), \quad x \in [\alpha, \beta].
$$
Пусть $y_{ij}(x),\ i = \overline{1,n}$ компоненты решения $y_j(x),\ j = \overline{1,n}$. Тогда $W(x)$ является функцией всех этих компонент:
$$
W(x) = W[y_{11}(x), y_{21}(x), \ldots, y_{nn}(x)].
$$
По формуле производной сложной функции получаем, что 
$$
W'(x) = \sum\limits_{p,q = 1}^{n} \frac{\partial W(x)}{\partial y_{pq}(x)} y'_{pq}(x).
$$
Если $W_{pr}(x)$ "--- алгебраическое дополнение $y_{pr}(x)$ в $W(x)$, то разложение $W(x)$ по $p$-й строке дает
$$
W(x) = \sum\limits_{r = 1}^{n} y_{pr}(x) \cdot W_{pr}(x).
$$
Отсюда находим, что 
$$
\frac{\partial W(x)}{\partial y_{pq}} = W_{pq}(x).
$$
Каждая вектор-функция $y_q(x)$ удовлетворяет системе $(\ref{ch28.4eq1})$, т.е.
$$
y'_q(x) = A(x)y_q(x), \quad q = \overline{1,n}, \quad x \in [\alpha, \beta].
$$
Отсюда находим, что 
$$
y'_{pq}(x) = \sum\limits_{r = 1}^{n} a_{pr}(x)y_{rq}(x),
$$
где $a_{pr}(x)$ "--- элементы матрицы $A(x)$.

Подставляя найденные выражения $\frac{\partial W(x)}{\partial y_{pq}}$ и $y'_{pq}(x)$ в формулу $W'(x)$, получаем, что
$$
W'(x) = \sum\limits_{p,q = 1}^{n} W_{pq}(x) \sum\limits_{r = 1}^{n} a_{pr}(x)y_{rq}(x) = \sum\limits_{p,r = 1}^{n} a_{pr}(x) \sum\limits_{q = 1}^{n} y_{rq}(x) W_{pq}(x).
$$

Но из курса алгебры известно, что 
$$
\sum\limits_{q = 1}^{n}y_{rq}(x) W_{pq}(x) = W(x) \cdot \delta_{rp},
$$
где $\delta_{rp}$ "--- символ Кронекера. Тогда
$$
W'(x) = W(x) \sum\limits_{p,r = 1}^{n} a_{pr}(x) \delta_{pr} = W(x) \sum\limits_{p = 1}^{n} a_{pp}(x) = W(x) \cdot \sp A(x).
$$
Интегрирование этого линейного однородного уравнения первого порядка дает требуемую формулу $(\ref{ch28.4eq6})$.
\end{proof}

\begin{thm}
Пусть $W(x)$ "--- определитель Вронского решений $y_1(x), \ldots, y_n(x)$ уравнения $(\ref{ch28.2eq1})$ и пусть $x_0 \in [\alpha, \beta]$. Тогда для всех $x \in [\alpha, \beta]$ справедлива формула Лиувилля-Остроградского
$$
W(x) = W(x_0) \exp \left\lbrace - \int\limits_{x_0}^{x} a_1(\zeta) \,d\zeta \right\rbrace.
$$
\end{thm}

\begin{proof}
Уравнение $(\ref{ch28.2eq1})$ эквивалентно линейной системе $(\ref{ch28.2eq2})$. Для них определитель Вронского $W(x)$ один и тот же, и для системы $(\ref{ch28.2eq2})$ формула Лиувилля-Остроградского доказана. Остается заметить, что для системы $(\ref{ch28.2eq2})$ $\sp A(\zeta) = - a_1(\zeta)$.
\end{proof}

\begin{notion}
Рассмотрев уравнение $\eqref{ch28.2eq1}$, заметим, что коэффициент при старшей производной $y^{(n)}$:  $a_0(x) = 1$. Если $a_0(x) \not= 1, a_0(x) \not= 0$, то 
$$
W(x) = W(x_0) exp \left\lbrace - \int\limits_{x_0}^{x} \frac{a_1(\zeta)}{a_0(\zeta)} \,d\zeta \right\rbrace.
$$
\end{notion}