\chapter{Простейшая задача вариационного исчисления. Необходимые условия локального экстремума.}
\section{Простейшая задача вариационного исчисления} 
Обозначим через $C^1[a,b]$ множество всех непрерывно дифференцируемых функций, заданных на $[a,b]$. Для $\forall y_1(x),y_2(x)\in C^1[a,b]$ введем \textit{расстояние} между ними по формуле
$$
||y_1(x)-y(x)||_{C^1[a,b]}=\max_{x\in [a,b]}\limits |y_1(x)-y_2(x)|+\max_{x\in[a,b]}\limits |y'_1(x)-y'_2(x)|. 
$$ 
Множество функций $C^1[a,b]$ с введенной метрикой является линейным нормированным пространством.

Пусть $F(x,y,p)$ "--- заданная непрерывно дифференцируемая функция для $\forall x\in [a,b]$ и $\forall(y,p)\in \bbR^2_{(y,p)}$ "--- плоскости с декартовыми прямоугольными координатами $y,p$. Рассмотрим интеграл
\begin{equation}\label{ch29eq1}
J(y)=\int_a^b\limits F[x,y(x),y'(x)]\,dx
\end{equation}
на множестве $M$ тех функций $y(x)\in C^1[a,b]$, которые удовлетворяют граничным условиям
\begin{equation}\label{ch29eq2}
y(a)=A,\quad y(b)=B,
\end{equation}
где $A$ и $B$ "--- заданные числа. Функции $y(x)\in M$ будем называть \textit{допустимыми}.

Очевидно, что $\forall y(x)\in M$ интеграл \eqref{ch29eq1} определен и задает функционал (отображение $F \colon M \to \bbC$) с областью определения $M$ в пространстве $C^1[a,b]$.

\begin{defn}
Говорят, что функция $\hat{y}(x)\in M$ дает \textit{слабый локальный минимум (максимум)} функционала \eqref{ch29eq1}, если $\exists$ число $\epsilon>0$ такое, что для $\forall y(x)\in M$, для которой $||y(x)-\hat{y}(x)||_{C^1[a,b]}<\epsilon$, выполняется неравенство $J(y)\geq J(\hat{y})$ ($J(y)\leq J(\hat{y})$).
\end{defn}

Оба понятия "--- слабый локальный минимум и слабый локальный максимум объединяются единым термином: \textit{слабый локальный экстремум}.

\begin{defn}
Задача нахождения слабого локального экстремума функционала \eqref{ch29eq1} называется \textit{простейшей вариационной задачей}
\end{defn}

Простейшую вариационную задачу иногда называют задачей с закрепленными концами в силу того, что допустимые кривые обязаны проходить через две закрепленные точки $M_1(a,A)$ и $M_2(b,B)$ на плоскости $\bbR^2_{(x,y)}$

Обозначим через $\coci[1]{[a,b]}$ множество всех тех функций $y(x)\in C^1[a,b]$, для которых $y(a)=y(b)=0$.

Пусть $y(x)\in M$ и $\eta (x)\in \coci[1]{[a,b]}$. Рассмотрим семейство функций, зависящих от действительного параметра $\alpha$: $y(x,\alpha)=y(x)+\alpha\cdot\eta(x)$. Поскольку $y(x,\alpha)\in M$ при $\forall\alpha$, то можно рассмотреть интеграл 
\begin{equation} \label{ch29eq3}
J(y+\alpha\eta)=\int\limits_a^b F[x,y(x)+\alpha\eta(x),y'(x)+\alpha\eta'(x)]\,dx.
\end{equation}

При фиксированных $y(x)$ и $\eta(x)$ интеграл $\eqref{ch29eq3}$ является собственным интегралом $\Phi(\alpha)$, зависящим от параметра $\alpha$. Если взять некоторое $\epsilon>0$, то при $|\alpha|\leq\epsilon,\;x\in[a,b]$ подынтегральная функция F и ее производная по $\alpha$ в силу наложенных на $F$ условий являются непрерывными. Тогда по известной из курса анализа теореме $\Phi(\alpha)=J(y+\alpha\eta)$ является дифференцируемой функцией $\alpha$ при $|\alpha|\leq \epsilon$ и по правилу Лейбница
\begin{multline*}
\Phi'(0)=\frac{d}{d\alpha}J[y(x)+\alpha\eta(x)]\biggr|_{\alpha=0}=\\=\int\limits_a^b\left\{ \frac{\partial F[x,y(x),y'(x)]}{\partial y}\cdot \eta(x)+\frac{\partial F[x,y(x),y'(x)]}{\partial y'}\cdot \eta'(x)\right\}\,dx.
\end{multline*}

\begin{defn}
\textit{Допустимым приращением (вариацией)} функции $y(x)\in M$ называется любая функция $\eta (x)\in \coci[1]{[a,b]}$. Выражение $\frac{d}{d\alpha}J[y(x)+\alpha\eta(x)]\bigr|_{\alpha=0}$, где $\eta(x)$ "--- любая функция из $\coci[1]{[a,b]}$,  называется \textit{первой вариацией} функционала $J(y)$ на функции $y(x)$ и обозначается $\delta J[y,\eta(x)]$, $\forall\eta(x)\in\coci[1]{[a,b]}$.
\end{defn}

Таким образом, вариация функционала \eqref{ch29eq1}
\begin{equation} \label{ch29eq4}
\delta J[y,\eta(x)]=\int\limits_a^b\left\{\frac{\partial F[x,y(x),y'(x)]}{\partial y}\cdot\eta(x) + \frac{\partial F[x,y(x),y'(x)]}{\partial y'}\cdot \eta'(x)\right\}\,dx,
\end{equation}
где $\eta(x)$ "--- любая допустимая вариация функции  $y(x)\in M$.

Отметим, что первая вариация $\delta J[y,\eta(x)]$ линейно зависит от $\eta(x)$ и $\eta'(x)$.

\section{Необходимые условия локального экстремума}
\begin{thm} \label{ch29thm1}
Если $\hat{y}(x)\in M$ является решением простейшей вариационной задачи, то необходимо $\delta J[\hat{y},\eta (x)]=0$ для любой допустимой $\eta (x)$.
\end{thm}

\begin{proof}
Пусть для определенности $\hat{y}(x)\in M$ дает слабый локальный минимум для функционала $J(y)$, т.е. $\exists\epsilon>0$ такое, что $J(\hat{y}+h)\geq J(\hat{y})$ для $\forall h(x)\in \coci[1]{[a,b]}$, для которой $||h(x)||<\epsilon$. Положим $h(x)=\alpha\eta(x)$, где $\alpha\in \bbR,\;\eta(x)\in \coci[1]{[a,b]}$. Тогда $\hat{y}(x)+h(x)\in M$ и для достаточно малых $|\alpha|$ при фиксированной $\eta(x)$
\begin{gather}
||h(x)||_{C^1[a,b]}=|\alpha|\left\{\max_{[a,b]}\limits |\eta(x)|+\max_{[a,b]}\limits |\eta'(x)|\right\},\\
\Phi(\alpha)=J(\hat{y}+\alpha\eta)\geq J(\hat{y})=\Phi(0)
\end{gather}
Это значит, что дифференцируемая функция $\Phi(\alpha)$ имеет минимум при $\alpha=0$. Значит, $\Phi ' (0)=0$, и тогда $\delta J[y,\eta(x)]=\Phi ' (0)=0,\; \forall \eta(x)\in \coci[1]{[a,b]}$
\end{proof}

\begin{lemm}[Основная лемма вариационного исчисления] \label{ch29.2lemm1}
Если $f(x)\in C[a,b]$ и $\int_a^b\limits f(x)\eta(x)\,dx=0$ для $\forall\eta(x)\in \coci[1]{[a,b]}$, то $f(x)\equiv 0$ на $[a,b]$.
\end{lemm}
\begin{proof}
Рассуждаем от противного. Пусть $f(x) \not\equiv  0$ на $[a,b]$. Тогда $\exists x_0\in(a,b)$ такая, что $f(x_0)\neq 0$. Пусть для определенности $f(x_0)>0$. Из непрерывности $f(x)$ на $[a,b]$ следует, что $\exists\epsilon>0$ такое, что $f(x)\geq\frac{1}{2}f(x_0)$, $\forall x\in [x_0-\epsilon,x_0+\epsilon]\subset (a,b).$ Возьмем
$$
\eta(x)=\begin{cases} 
[x-(x_0-\epsilon)]^2\cdot [x-(x_0+\epsilon)]^2, &x\in[x_0-\epsilon,x_0+\epsilon],\\
0, &x\notin[x_0-\epsilon,x_0+\epsilon].
\end{cases}
$$
Нетрудно проверить, что функция $\eta(x)\in \coci[1]{[a,b]}$. По интегральной теореме о среднем получаем, что
\begin{multline*}
\int_a^b\limits f(x)\eta(x)\,dx=\int\limits_{x_0-\epsilon}^{x_0+\epsilon} f(x)\eta(x)\,dx =\\= f(\zeta)\int\limits_{x_0-\epsilon}^{x_0+\epsilon}\eta(x)\,dx \geq \frac{1}{2} f(x_0)\int\limits_{x_0-\epsilon}^{x_0+\epsilon}  \eta (x)\,dx>0,
\end{multline*}
где $\zeta\in[x_0-\epsilon,x_0+\epsilon]$. А это неравенство противоречит условию теоремы. Значит, наше предположение о том, что $f(x)\not\equiv  0$ на $[a,b]$ неверно. Лемма доказана.
\end{proof}


\begin{thm}\label{ch29thm2}
Пусть функция $F(x,y,p)$ "--- дважды непрерывно дифференцируема при $\forall x\in[a,b]$, $\forall(y,p)\in \bbR^2_{(y,p)}$. Если дважды непрерывно дифференцируемая функция $\hat{y}(x)$ является решением простейшей вариационной задачи, то необходимо функция $\hat{y}(x)$ на $[a,b]$ удовлетворяет уравнению Эйлера
\begin{equation} \label{ch29eq5}
\frac{\partial F}{\partial y}-\frac{d}{dx}\cdot\frac{\partial F}{\partial y'}=0
\end{equation}
\textup{(}здесь $\frac{d}{dx}$ "--- полная производная по $x$\textup{)}.
\end{thm}

\begin{proof}
Если $\hat{y}(x)$ "--- решение задачи, то в силу теоремы \ref{ch29thm1} $\delta J[\hat{y},\eta(x)] = 0$ для любой допустимой вариации $\eta(x)$. Учитывая, что $\eta(a)=\eta(b)=0$, проинтегрируем по частям слагаемое, содержащее $\eta'(x)$ в формуле \eqref{ch29eq4}. Это законно, так как выражение
$$
\frac{d}{dx}\frac{\partial F[x,y(x),y'(x)]}{\partial y'}\biggr|_{y=\hat{y}(x)} = \left[\frac{\partial^2F}{\partial y'\partial x}+\frac{\partial^2F}{\partial y'\partial y}\cdot y'+\frac{\partial ^2F}{\partial {y'}^{2}}\cdot y'' \right]\biggr|_{y=\hat{y}(x)}
$$
в силу условий теоремы является непрерывной на $[a,b]$ функцией. Имеем
\begin{multline*}
0=\int\limits_a^b\left\{\frac{\partial F[x,\hat{y}(x),\hat{y}'(x)]}{\partial y}\cdot \eta(x)+\frac{\partial F [x,\hat{y}(x),\hat{y}'(x)]}{\partial y'}\cdot\eta'(x)\right\}\,dx=\\=\frac{\partial F[x,\hat{y}(x),\hat{y}'(x)]}{\partial y'}\cdot \eta(x)\biggr|_{x=a}^{b}+\int\limits_a^b\left\{\frac{\partial F}{\partial y} - \frac{d}{dx}\frac{\partial F}{\partial y'}\right\}\biggr|_{y=\hat{y}(x)}\cdot\eta(x)\,dx=\\=\int\limits_a^b\left\{\frac{\partial F}{\partial y}-\frac{d}{dx}\frac{\partial F}{\partial y'}\right\}\biggr|_{y=\hat{y}(x)}\cdot \eta(x)\,dx.
\end{multline*}

Поскольку функция  $\eta(x)\in\coci[1]{[a,b]}$, а функция
$$
\left\{\frac{\partial F(x,y,y')}{\partial y}-\frac{d}{dx}\cdot\frac{\partial F(x,y,y')}{\partial y'} \right\}\biggr|_{y=\hat{y}(x)}
$$
является непрерывной на $[a,b]$, то в силу леммы  \ref{ch29.2lemm1} (основной леммы вариационного исчисления)
$$
\left\{\frac{\partial F(x,y,y')}{\partial y} - \frac{d}{dx}\cdot\frac{\partial F(x,y,y')}{\partial y'}\right\}\biggr|_{y=\hat{y}(x)}\equiv 0,\quad \forall x\in [a,b].
$$
Это значит, что $\hat{y}(x)$ удовлетворяет уравнению Эйлера \eqref{ch29eq5}.
\end{proof}

\begin{defn}
Всякое решение уравнения Эйлера \eqref{ch29eq5} называют \textit{экстремалью} функционала \eqref{ch29eq1}. Всякая же экстремаль $y(x)$ функционала \eqref{ch29eq1}, являющаяся допустимой функцией, т.е. $y(x)\in M$, называется \textit{допустимой экстремалью} функционала \eqref{ch29eq1}.
\end{defn}

Из теоремы \ref{ch29thm2} вытекает, что только среди допустимых экстремалей \eqref{ch29eq1}, т.е. среди экстремалей, удовлетворяющих граничным условиям \eqref{ch29eq2}, нужно искать решение простейшей вариационной задачи. 

