\font\Large=cmr10 at 20pt
\def\fudge#1{\smash{\hbox{\Large#1}}}
\chapter[Линейное отображение конечномерных линейных прост\-ранств, его матрица. Свойства собственных векторов и собственных значений линейных преобразований.]{Линейное отображение конечномерных линейных пространств, его матрица. Свойства собственных векторов и собственных значений линейных преобразований.}
\section{Линейное отображение конечномерных линейных пространств, его матрица}
Пусть $L_1$ и $L_2$ "--- линейные пространства над одним  и тем же полем~$K$.
\begin{defn}
Отображение $\phi\colon L_1\rightarrow L_2$ называется \textit{линейным отображением}, если
\begin{enumerate}
\item $\forall x_1,x_2 \in L_1 \hookrightarrow \phi(x_1+x_2)=\phi(x_1)+\phi(x_2)$
\item $\forall x \in L_1, \forall \lambda \in K \hookrightarrow \phi(\lambda x)=\lambda \phi(x)$
\end{enumerate}
\end{defn}
В частности, из определения следует, что $\phi(0_{L_1})=0_{L_2}$.
\begin{defn}
\textit{Образ} отображения "--- это множество 
\begin{equation}
\Im\phi=\phi(L_1)=\{y\in L_2\,\big|\, \exists x\in L_1: \phi(x)=y\}
\end{equation}
\end{defn}
\begin{defn}
\textit{Ядро} отображения "--- это множество 
\begin{equation}
\Ker\phi=\{x\in L_1\,\big|\, \phi(x)=0_{L_2}\}
\end{equation}
\end{defn}
\begin{defn}
Отображение $\phi\colon L_1\rightarrow L_2$ называется \textit{инъективным}, если никакие два различных вектора из $L_1$ не имеют одинаковых образов:
\begin{equation}\label{22.1.inject}
\forall x_1,x_2 \in L_1\cquad \phi(x_1)=\phi(x_2) \hookrightarrow x_1=x_2
\end{equation}
\end{defn}  
\begin{defn}
Отображение $\phi\colon L_1\rightarrow L_2$ называется \textit{сюръективным}, если любой элемент $L_2$ имеет прообраз в $L_1$:
\begin{equation}\label{22.1.surject}
\forall y \in L_2 \quad \exists x \in L_1\cquad \phi(x)=y \Leftrightarrow \Im \phi = L_2
\end{equation}
\end{defn}  

\begin{stt} 
\begin{enumerate}
\item $\Ker\phi$ "--- линейное подпространство пространства~$L_1$; 
\item $\Im\phi$ "--- линейное подпространство пространства~$L_2$.
\end{enumerate}
\end{stt}
\begin{thm}
Линейное преобразование $\phi$ инъективно $\Leftrightarrow \Ker\phi=\{0\}$
\end{thm}
\begin{proof} $ $
\linebreak\vspace*{-\baselineskip}
\begin{itemize}
\item[\underline{$\Longrightarrow:$}]   Пусть $\phi$ инъективно, то есть выполняется \eqref{22.1.inject}. Тогда \\ $\forall x\in L_1:\phi(x)=0=\phi(0) \hookrightarrow x=0,$ то есть $\Ker\phi=\{0\}$.
\item[\underline{$\Longleftarrow:$}] 
Пусть $\Ker\phi=\{0\}$ и $\phi(x_1)=\phi(x_2)$. Тогда $\phi(x_1-x_2)=0 \Rightarrow x_1-x_2=0$. Таким образом, выполняется условие \eqref{22.1.inject}.
\end{itemize}
\vspace{-1.65\baselineskip}
\end{proof}

\begin{defn}
Линейное отображение $\phi: L\rightarrow L$, отображающее пространство $L$ в себя, называется \textit{линейным преобразованием}.
\end{defn}

Напомним, говорят, что \textit{размерность} линейного пространства $\dim L = n$, если в нём существует базис из $n$ векторов.

Рассмотрим запись преобразования $\phi$  в базисах. Пусть $\dim L_1 = n$, $e=\norm{e_1 \ldots e_n}$ "--- базис в $L_1$; $\dim L_2 = m$, $f=\norm{f_1 \ldots f_m}$ "--- базис в $L_2$.
\begin{equation*}
\forall x \in L_1 \hookrightarrow x=\sum_{j=1}^n x_je_j \Rightarrow \phi(x)\xlongequal{linear}\sum_{j=1}^n x_j\phi(e_j) 
\end{equation*}
\begin{equation*}
\phi(e_j) \in L_2 \Rightarrow \phi(e_j)=\sum_{i=1}^m a_{ij}f_i \Rightarrow 
\end{equation*}
\begin{equation*}
\Rightarrow \phi(x)=\sum_{j=1}^n\sum_{i=1}^m x_ja_{ij}f_i = \sum_{i=1}^m(\sum_{j=1}^n x_ja_{ij})f_i = \sum_{i=1}^m y_if_i
\end{equation*}
Отсюда и из единственности разложения по базису получаем закон преобразования координат:
\begin{equation}
y_i=\sum_{j=1}^n a_{ij}x_j, i=\overline{1,m} \Leftrightarrow Y_f^\uparrow=A_{\phi,e,f}X_e^\uparrow,
\end{equation}
где через $X_e^\uparrow$, или, сокращённо, $X$ обозначаются столбцы координат $\begin{Vmatrix}
x_1 \\ x_2 \\ \vdots \\ x_n
\end{Vmatrix}$ при разложении элемента $x$ по базису $e$: $x=e X_e^\uparrow$
\begin{defn}
Матрица $A_{\phi,e,f}=(a_{ij})_{\substack{ 1\le i\le m \\ 1\le j\le n }}=\norm{\phi(e_1)^\uparrow \ldots \phi(e_n)^\uparrow}$, состоящая из столбцов координат образов базисных элементов $e_j$ в разложении по базисным элементам $f_i$, называется \textit{матрицей линейного отображения} $\phi$ в паре базисов $e$ и $f$.
\end{defn}
\begin{stt}\label{22.1.KerIm} {О вычислении образа и ядра преобразования с помощью его матрицы} $ $\\
1) $x \in \Ker\phi \Leftrightarrow A_\phi X^\uparrow=0^\uparrow$ \\
2) $\Im\phi=\langle\phi(e_1) \ldots \phi(e_n)\rangle=\langle a_1^\uparrow \ldots a_n^\uparrow\rangle$, где $a_j^\uparrow$ "--- столбцы матрицы $A_\phi$, а $\langle a_1^\uparrow \ldots a_n^\uparrow\rangle$ "--- \textit{линейная оболочка}, то есть множество всех линейных комбинаций $\lambda_1 a_1^\uparrow + \ldots + \lambda_n a_n^\uparrow, \lambda_1\ldots\lambda_n \in \bbR$.
\end{stt}
\begin{thm}
Пусть дано линейное отображение $\phi:L_1\rightarrow L_2$. Тогда $\dim L_1=\dim\Ker\phi+\dim\Im\phi$.
\end{thm}
\begin{proof}
Из пункта 1) утверждения \ref{22.1.KerIm} и теоремы \ref{21.2.th.common} билета 21 следует, что размерность ядра преобразования равна количеству параметрических неизвестных, то есть $n-\rg A_\phi$, а его базис в координатной записи имеет вид фундаментальной системы решений уравнения $A_\phi X^\uparrow=0^\uparrow$.
Из пункта 2) того же утверждения следует, что $\dim\Im\phi=\rg A_\phi$.
\end{proof}    
\begin{notion}
Из доказанного, однако, не следует, что для любого преобразования $\phi$ пространства $L_2$ пространство разложимо в сумму ядра и образа преобразования.
\end{notion}
\begin{exmpl} $ $ \\
1) Для преобразования двумерного пространства с матрицей $A_\phi=\begin{pmatrix}
0 && 1 \\ 0 && 0
\end{pmatrix} \\
\Ker\phi=\Im\phi=\langle e_1\rangle, \Ker\phi+\Im\phi=\langle e_1\rangle \neq L = \langle e_1, e_2\rangle$  \\
2) В пространстве многочленов $P^{(n)}(x)$ степени не выше $n$ преобразование $\phi=\dfrac{d}{dx}$ будет линейным с $\Ker\phi=P^{(0)}(x)$ и $\Im\phi=P^{(n-1)}(x)$
\end{exmpl}

\section{Свойства собственных векторов и собственных значений линейных преобразований}
Пусть дано линеное преобразование $\phi\colon L \rightarrow L$, где $L$ - линейное пространство над полем $K$.
\begin{defn}
Вектор $x \in L$: $x \neq \vv 0$ называется \textit{собственным вектором} преобразования $\phi$, если 
\begin{equation}
\exists \lambda \in K\cquad \phi(x)=\lambda x,
\end{equation}
$\lambda$ называется \textit{собственным значением} преобразования $\phi$ 
\end{defn}
\begin{thm}
Пусть $x_1$,~\ldots, $x_n$ являются собственными векторами линейного преобразования $\phi$, отвечающими попарно разным собственным значениям $\lambda_1$,~\ldots,~$\lambda_n$. Тогда $x_1$,~\ldots, $x_n$ образуют линейно независимую систему векторов.
\end{thm}
\begin{proof}
Воспользуемся методом математической индукции:
\linebreak\vspace*{-\baselineskip}
\begin{itemize}
\item[\underline{$n=1:$}] Вектор $x_1$ является собственным, значит, по определению не равен нулевому $\Rightarrow$ образует линейно-независимую систему.
\item[\underline{$n-1:$}] Пусть теорема верна для $n-1\ge 1$ собственных векторов.

\item[\underline{$n:$}] Пусть для некоторых $\alpha_1,\dots,\alpha_n \hookrightarrow \alpha_1 x_1 + \ldots+\alpha_n x_n = 0$. Покажем, что $\alpha_1=\ldots=\alpha_n=0$.

$\phi(\sum_{i=1}^n \alpha_i x_i)=\sum_{i=1}^n \alpha_i \lambda_i x_i = \phi(0)=0$. \\
Получаем систему: \\
$\begin{cases}
\alpha_1 x_1 + \alpha_2 x_2 + \ldots + \alpha_n x_n = 0 & |\cdot \lambda_n \\
\alpha_1 \lambda_1 x_1 + \alpha_2 \lambda_2 x_2 + \ldots + \alpha_n \lambda_n x_n = 0 \\
\end{cases}$ \\
Вычтем второе равенство из первого и получим:
\begin{equation*}
\alpha_1 (\lambda_n-\lambda_1) x_1 + \alpha_2 (\lambda_n-\lambda_2) x_2 + \ldots + \alpha_{n-1} (\lambda_n-\lambda_{n-1}) x_{n-1} = 0
\end{equation*}
По предположению индукции $\forall i=\overline{1,n-1} \hookrightarrow \alpha_i(\lambda_n-\lambda_i)=0$, откуда в силу условия теоремы $\alpha_1=\ldots=\alpha_{n-1}=0 \Rightarrow \alpha_n x_n=0 \Rightarrow \alpha_n=0$
\end{itemize}
\vspace{-1.65\baselineskip}
\end{proof}
\begin{cons}
Если $\dim L = n$ и $\phi$ имеет $n$ различных собственных значений, то из собственных векторов преобразования $\phi$ можно составить базис $L$.
\end{cons}  
\begin{stt}
Линейное преобразование $\phi$ приводимо к диагональному виду тогда и только тогда, когда в $L$ существует базис $ h = \norm{h_1,...,h_n}$ из собственных векторов $\phi$. При этом $A_{\phi,h}=
\begin{pmatrix}
\lambda_1 &           &        & \\
        & \lambda_2 &        & \fudge {0 }  \\
        &			& \ddots & \\
\fudge{ 0}&			&		 & \lambda_n \\
\end{pmatrix}$, где $\lambda_i$ удовлетворяют $\phi(h_i)=\lambda_i h_i$.
\end{stt} 

Теперь обсудим способ вычисления собственных значений и собственных векторов. Пусть в пространстве $L$ задан базис $ e=\norm{e_1,...,e_n}$. Распишем в координатном представлении закон преобразования собственного вектора $x=e X^\uparrow$:
\begin{equation}\label{22.2.eigenvectors}
A_\phi X^\uparrow=\lambda X^\uparrow \Leftrightarrow (A_\phi-\lambda E)X^\uparrow=0
\end{equation}
Уравнение \ref{22.2.eigenvectors} имеет нетривиальные решения, если матрица вырожденна:
\begin{equation}\label{22.2.eigenvalues}
\bigchi_{A_\phi}(\lambda)\equiv\det(A_\phi-\lambda E)=0
\end{equation}
\begin{defn}
Уравнение \ref{22.2.eigenvalues} называется \textit{характеристическим}, его корни --- \textit{характеристическими корнями}, а многочлен $\bigchi_{A_\phi}(\lambda)$ --- \textit{характеристическим многочленом матрицы} $A_\phi$.
\end{defn}
\begin{stt}
Если $\lambda$ --- собственное значение, то оно является характеристическим корнем. Характеристический корень $\lambda_0$ является собственным значением, если $\lambda_0 \in K$.
\end{stt}
\begin{exmpl}
Пусть $\phi$ --- преобразование поворота на угол $\alpha$ в двумерном пространстве над полем вещественных чисел $K=\bbR$: $A_\phi=\begin{pmatrix}
\cos\alpha && -\sin\alpha \\ \sin\alpha && \cos\alpha
\end{pmatrix}$ и $\alpha \neq k\pi$. Тогда собственными значениями являются $\lambda_{1,2}=e^{\pm i\alpha} \notin K$. Данное преобразование не имеет собственных значений, а значит, и собственных векторов.
\end{exmpl}

\begin{defn}
Пусть $\phi:L\rightarrow L$ --- линейное преобразование пространства $L$ над полем $K$, $\lambda \in K$. Множество $L_\lambda$ собственных векторов, отвечающих собственному значению $\lambda$, называется \textit{собственным подпространством} для собственного значения $\lambda$.
\end{defn}
\begin{defn}
\textit{Геометрической кратностью} характеристического корня $\lambda_0$ называется размерность пространства $L_{\lambda_0}$.
\textit{Алгебраической кратностью} называется число $k \in \bbN$, если $\bigchi(\lambda)=(\lambda-\lambda_0)^k p(\lambda)$, где $p(\lambda)$ - такой многочлен, что $p(\lambda_0) \neq 0$.
\end{defn}

\begin{thm}
Геометрическая кратность характеристического корня не превосходит его алгебраической кратности.
\end{thm}
Действительно, если $\lambda_0$ --- корень геометрической кратности $m$, то в $L_{\lambda_0}$ можно выбрать базис~$e_1$, \ldots,~$e_m$, а затем дополнить его до линейно независимого набора элементами $e_{m+1}$, \ldots,~$e_n \in L$. Далее нужно перейти к полученному базису и воспользоваться определением матрицы преобразования:
\begin{equation*}
A_{\phi,e}=
\begin{Vmatrix}
\begin{tabular}{ccc|c}
  $\lambda_0$ & 			&$\fudge{0 }$& \\
     			& $\ddots$ 	& 			& $\fudge C$ \\
  $\fudge{ 0}$&			& $\lambda_0$ & \\ \hline
     			&  			& 			& \\
     			& $\fudge 0$ 	& 			& $\fudge D$ \\
\end{tabular} 
\end{Vmatrix} 
\end{equation*}
\begin{equation*}
\bigchi_\phi=
\begin{vmatrix}
\begin{tabular}{ccc|c}
  $\lambda_0-\lambda$ & 			&$\fudge{0 }$ 			& \\
     			& $\ddots$ 	& 			& $\fudge C$ \\
  $\fudge{ 0}$&			& $\lambda_0-\lambda$ & \\ \hline
     			&  			& 			& \\
     			& $\fudge 0$ 	& 			& $ D-\lambda E$ \\
\end{tabular} 
\end{vmatrix} = \begin{vmatrix}
    \lambda_0-\lambda & 			& \\
     			& \ddots 	&  \\
     			&  & \lambda_0-\lambda\\ 
\end{vmatrix} \begin{vmatrix}
D-\lambda E
\end{vmatrix}=
\end{equation*}
\begin{equation}
=(\lambda_0-\lambda)^m \det(D-\lambda E) \Rightarrow k \ge m
\end{equation}
