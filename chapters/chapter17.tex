\part[Гармонический анализ]{Гармонический анализ}

\chapter{Достаточные условия сходимости тригонометрического ряда Фурье в точке.}\label{chapter17}
\section{Ортогональные системы и ряды Фурье}
\subsection{Ортогональные и ортонормированные системы функций}
Пусть задана система функций 
\begin{equation} \label{ch17.1eq00}
\phi_1(x),\, \phi_2(x),\, \ldots,\, \phi_n(x),\, \ldots,
\end{equation}
области определения которых имеют непустое пересечение. Тогда любой функциональный ряд вида
\begin{equation} \label{ch17.1eq01}
\sum_{n = 1}^{\infty} a_n \phi_n(x),
\end{equation}
где $a_n$ "--- некоторые числа, называется \textit{рядом по системе функций~\eqref{ch17.1eq00}}, а числа $a_n$ "--- \textit{коэффициентами ряда \eqref{ch17.1eq00}.}

Говорят, что \textit{функция $f(x)$ разложена в ряд по системе функций~\eqref{ch17.1eq00}}, если указаны числа $a_1$,~$a_2$, \dots,~$a_n$, \dots, такие, что ряд~\eqref{ch17.1eq01} в любой точке области определения функции~$f$ сходится и 
$$
\sum_{n = 1}^{\infty} a_n \phi_n(x)=f(x) \quad \forall x \in D_f
$$ 

\begin{defn}
Любые две функции $\phi(x)$	и $\psi(x)$, определенные на промежутке $\Delta$, называются \textit{ортогональными на $\Delta$}, если их произведение интегрируемо на~$\Delta$ (в собственном или несобственном смысле) и
$$
\int_\Delta\phi(x) \psi(x)\,dx=0
$$
Очевидно, функция, тождественно равная нулю на промежутке~$\Delta$, ортогональна на~$\Delta$ любой функции, определенной на $\Delta$.
\end{defn}

\begin{defn}
Система функций, определенных на промежутке $\Delta$, называется \textit{ортогональной на $\Delta$}, если любые две функции этой системы ортогональны на $\Delta$.
\end{defn}

\begin{defn}
Система функций 
\begin{equation} \label{ch17.1eq0}
\{1,\,\cos kx,\,\sin kx\}, \quad k\in \bbN
\end{equation}
называется тригонометрической системой функций.			
\end{defn}
Легко проверить, что тригонометрическая система~\eqref{ch17.1eq0} ортогональна на интервале $(-\pi;\pi)$. А так как все функции это системы периодические с периодом $2\pi$, то справедливо следующее утверждение:

\textit{Тригонометрическая система \eqref{ch17.1eq0} ортогональна на любом промежутке длины $2\pi$}.	

Обобщение системы~\eqref{ch17.1eq0} является система функций
$$
1,\,\cos \frac{\pi}{l}x,\,\sin \frac{\pi}{l}x,\,\ldots,\,\cos n\frac{\pi}{l}x,\,\sin n\frac{\pi}{l}x,\,\ldots,
$$
где $l>0$, которую тоже будем называть \textit{тригонометрической системой}. Очевидно, она ортогональная на любом промежутке длины $2l$.
\begin{defn}
Система функций \eqref{ch17.1eq0}, определенных на промежутке $\Delta$, называется \textit{ортонормированной на $\Delta$}, если она ортогональна на $\Delta$ и 
\begin{equation} \label{ch17.1eq000}
\int _\Delta{|\phi_n(x)|^2\,dx=1} \quad \forall n \in \bbN.
\end{equation}
Условие \eqref{ch17.1eq000} называется \textit{условием нормировки.}
\end{defn}

\subsection{Ряды Фурье по ортогональным системам функций}

Пусть $f(x)$, $x \in \Delta$, разложена в ряд по системе функций
\begin{equation} \label{ch17.1eq02}
\phi_1(x),\,\phi_2(x),\,\ldots,\,\phi_n(x),\,\ldots,
\end{equation}
ортогональной на промежутке $\Delta$:
\begin{equation} \label{ch17.1eq03}
f(x)= \sum_{n = 1}^{\infty} a_n \phi_n(x),
\end{equation}
Найдем формулы для определения коэффициентов этого ряда. Для этого обе части равенства~\eqref{ch17.1eq03} умножим на $\phi_k(x)$ и проинтегрируем по промежутку $\Delta$, предполагая, что полученный ряд можно интегрировать почленно. В результате получим равенство
$$
\int_\Delta{f(x) \phi_k(x)}\,dx=a_k\int_\Delta|\phi_k(x)|^2\,dx,
$$
так как интегралы от произведений $\phi_n(x)\psi_n(x)$ при $n\ne k$ равны нулю (в силу ортогональности системы \eqref{ch17.1eq02})

Предположим еще, что ортогональную систему~\eqref{ch17.1eq02} можно нормировать, т.е. что
$$
\int_\Delta|\phi_k(x)|^2\,dx\ne 0 \quad \forall n
$$
Тогда для коэффициентов $a_n$ ряда~\eqref{ch17.1eq03} получим формулу
\begin{equation} \label{ch17.1eq04}
a_n= \frac{1}{\int_\Delta|\phi_n(x)|^2\,dx} \int_\Delta f(x)\phi_n(x)\,dx, \quad n\in \bbN.
\end{equation}

\begin{defn}
Числа $a_n$, определяемые по формулам~\eqref{ch17.1eq04} называются \textit{коэффициентами Фурье функции $f$ по ортогональной системе~\eqref{ch17.1eq0}}, а ряд $\sum_{n = 1}^{\infty} a_n \phi_n(x)$ называется \textit{рядом Фурье функции~$f$\rindex{ряд!Фурье} по этой системе.} 	
\end{defn}

\begin{defn} 
Функция $f$, определенная на конечном промежутке $\Delta=(a;b)\subset \bbR$, за исключением, быть может, конечного числа точек из $\Delta$, называется \textit{абсолютно интегрируемой}\rindex{функция!абсолютно интегрируемая} на промежутке $\Delta$, если существует конечный набор элементов~$c_0$,~$c_1$, \ldots,~$c_m \in \bboR$, такой, что
\begin{enumerate}
\item
$a=c_0<c_1<\ldots<c_m=b$;
\item
$\forall [\alpha;\beta]\subset(c_{k-1};c_k)$, $k\in \overline{1,m}$ функция~$f$ интегрируема по Риману на $[\alpha;\beta]$;	
\item
$\int_\Delta |f(x)|\,dx$ сходится;
\item
$ \forall\epsilon>0$ функция $f$ интегрируема по Риману на множестве 
$$G_{\epsilon}=\Delta \setminus (\bigcup_{k=0}^{m}(O_{\epsilon}(c_k))$$
\end{enumerate}	
\end{defn}
\section{Тригонометрические ряды Фурье}	
	
Для любой функции $f$, определенной и абсолютно интегрируемой на конечном интервале $(a;b)$, определены коэффициенты Фурье по тригонометрической системе, ортогональной на интервале $(a;b)$:
\begin{equation} \label{ch17eq1}
1,\, \cos \frac{\pi}{l}x,\, \sin \frac{\pi}{l}x,\,\ldots,\,\cos n\frac{\pi}{l}x,\, \sin n\frac{\pi}{l}x,\,\ldots,
\end{equation}
где $l = (b - a)/2$. Соответствующий ряд Фурье обычно записывают в следующем виде:
\begin{equation} \label{ch17eq2}
\frac{a_0}{2} + \sum_{n = 1}^{\infty} a_{n} \cos n\frac{\pi}{l}x + b_{n} \sin n\frac{\pi}{l}x
\end{equation}
Из общей формулы для коэффициентов Фурье следует, что коэффициенты $a_0$, $a_n$, $b_n$ ряда~\eqref{ch17eq2} вычисляются по формулам:
\begin{align} \label{ch17eq3}
a_0 &= \frac{1}{l} \int_{a}^{b} f(x)\,dx,\\
a_n &= \frac{1}{l} \int_{a}^{b} f(x)\cos n\frac{\pi}{l}x\, dx,\\
b_n &= \frac{1}{l} \int_{a}^{b} f(x)\sin n\frac{\pi}{l}x\, dx,\\
\end{align}

\begin{defn}
Числа $a_0$, $a_n$, $b_n$, определяемые по формулам выше, называются \textit{коэффициентами Фурье функции $f$ по тригонометрической системе~\eqref{ch17eq1}}.
\end{defn}

Так как общий случай заменой $\xi = \frac{\pi}{l} x$ сводится к тригонометрической системе, ортогональной на $(-\pi;\pi)$, то в дальнейшем будем изучать в основном только ряды по тригонометрической системе
$$
1,\, \cos x,\, \sin x,\,\ldots,\,\cos nx,\, \sin nx,\,\ldots,
$$
которую будем называть \textit{стандартной тригонометрической системой.}

Вместо тригонометрической системы~\eqref{ch17eq1} ортогональной на интервале~$(-l;l)$, часто рассматривают систему комплекснозначных функций
\begin{equation} \label{ch17eq6}
\phi_{\nu}(x) = e^{i\nu\frac{\pi}{l}x},\quad \nu \in \bbZ,
\end{equation}
Ее тоже называют тригонометрической, так как
$$
e^{i\nu\frac{\pi}{l}x} = \cos \nu\frac{\pi}{l}x + i \sin \nu\frac{\pi}{l}x
$$
\begin{defn}
Для любой функции $f(x) \in \accentset{*}{L}^{R}_1(-l;l)$ числа
\begin{equation}
c_{\nu} = \frac{1}{2l} \int_{-l}^{l} f(x)\, e^{-i\nu\frac{\pi}{l}x}\,dx,\quad \nu = 0,\, \pm 1,\, \pm 2,\,\ldots
\end{equation}
называются \textit{коэффициентами Фурье функции $f$ по ортогональной системе}~\eqref{ch17eq6}, причем
\begin{equation}\label{2018:ch17:eq15}
c_0 = \frac{a_0}{2}, \quad c_{k} = \frac{a_k - ib_k}{2}, \quad c_{-k} = \frac{a_k + ib_k}{2} \quad \forall k \in \bbN,
\end{equation}
где $a_0, a_n, b_n$ "--- коэффициенты Фурье функции~$f$ по тригонометрической системе~\eqref{ch17eq1}.
\end{defn}

\begin{defn}
Выражение
\begin{equation}
\sum_{\nu = -\infty}^{\infty} c_{\nu}e^{i\nu\frac{\pi}{l}x},
\end{equation}
где $c_{\nu} = c_{\nu}(f)$, называют \textit{рядом Фурье функции f по системе}~\eqref{ch17eq6}, а сумма
\begin{equation}
T_{n}(f;x) = \sum_{\nu = -n}^{n} c_{\nu}e^{i\nu\frac{\pi}{l}x}
\end{equation}
называется \textit{n-й частичной суммой} этого ряда.
\end{defn}

Из формул~\eqref{2018:ch17:eq15} следует, что
\begin{multline*}
T_{n}(f;x) = \frac{a_0}{2} + \sum_{k = 1}^{n} \frac{a_k-ib_k}{2} e^{k\frac{\pi}{l}x} + \sum_{k = 1}^{n} \frac{a_k + ib_k}{2} e^{-k\frac{\pi}{l}x} =\\ = \frac{a_0}{2} + \sum_{k = 1}^{n} a_{k} \cos k\frac{\pi}{l}x + b_{k} \sin k\frac{\pi}{l}x
\end{multline*}

Таким образом, $T_{n}(f;x)$ "--- это $n$-я частичная сумма ряда Фурье функции $f$ по тригонометрической системе~\eqref{ch17eq1}.

Как уже говорилось, в основном будем рассматривать случай, когда $l=\pi$ и функция $f(x)$ периодическая с периодом $2\pi$. В этом случае
\begin{equation}\label{2018:ch17:eq18}
T_{n}(f;x) = \sum_{\nu = -n}^{n} c_{\nu}e^{i\nu x},
\end{equation}
где
\begin{equation}\label{2018:ch17:eq19}
c_v = \frac{1}{2\pi} \int_{-\pi}^{\pi} f(\xi) e^{i\nu \xi}\,d\xi,\ \nu\in\bbZ.
\end{equation}

\section{Интегральное представление частичных сумм рядов Фурье по тригонометрической системе. Ядро Дирихле}

Рассмотрим $n$-ю частичную сумму ряда Фурье функции $f(x)\in \accentset{*}{L}^{R}_1(-\pi;\pi)$ по тригонометрической системе, ортогональной на интервале $(-\pi;\pi)$. Как показано в прошлом пункте, имеют место формулы~\eqref{2018:ch17:eq18} и  \eqref{2018:ch17:eq19}. Подставляя одну в другую, получим
$$
T_{n}(f;x) = \frac{1}{2\pi} \int_{-\pi}^{\pi} f(\xi) \sum_{\nu = -n}^{n} e^{i\nu(x-\xi)}\,d\xi
$$
\begin{defn}
Функция $D_n(x) = \sum_{\nu = -n}^{n} e^{i\nu x}$ называется \textit{ядром Дирихле порядка $n$}.
\end{defn}

Очевидно, 
$$
D_n(x) = 1 + 2 \sum_{\nu = -n}^{n} \cos{\nu x}
$$
Отсюда следует, что $D_n(x)$ "--- четная $2\pi$-периодическая функция, $D_n(0) = 1+2n$ и
$$
\frac{1}{\pi}\int_{0}^{\pi} D_n(x) \,dx =1
$$
По формуле для суммы геометрической прогрессии получаем, что
$$
D_n(x) = \frac{e^{-inx} - e^{-inx +ix}}{1-e^{ix}} = 
\frac{\sin(n+\frac{1}{2})x}{\sin{\frac{x}{2}}}
$$
для любого $x\ne 2k\pi$, $k \in \bbZ$ (и равно $2n+1$ в противоположном случае, как уже было замечено).

С помощью ядра Дирихле $n$-я частичная сумма ряда Фурье функции $f(x)$ выражается следующим образом:
$$
T_{n}(f;x) = \frac{1}{2\pi} \int_{-\pi}^{\pi} f(\xi) D_n(\xi-x)\,d\xi = \frac{1}{2\pi} \int_{-\pi-x}^{\pi-x} f(\xi + x) D_n(\xi)\,d\xi.
$$
А так как подынтегральная функция является $2\pi$-периодической, то 
$$
T_{n}(f;x)= \frac{1}{2\pi} \int_{-\pi}^{\pi} f(\xi + x) D_n(\xi)\,d\xi.
$$

\section{Теорема Римана об осцилляции}
\begin{defn}
Для любой функции $f$ замыкание множества точек $x \in D_{f}$, в которых $f(x) \neq 0$, называется \textit{носителем функции~$f$} и обозначается $\supp f$.
\end{defn}
\begin{defn}
Функция, определенная на $\bbR$, называется \textit{финитной\rindex{функция!финитная}}, если ее носитель ограничен, т.е. если она равна нулю вне некоторого отрезка.
\end{defn}
\begin{defn}
Функция, определенная на некотором промежутке~$\Delta$, называется \textit{ступенчатой\rindex{функция!ступенчатая}}, если существует разбиение промежутка~$\Delta$ на конечное число промежутков, на каждом из которых она постоянна.
\end{defn}
\begin{thm} \label{ch17thm1}
Если функция~$f(x)$ абсолютно интегрируема на промежутке~$\Delta$, то для любого $\epsilon > 0$ существует финитная ступенчатая функция $\phi(x)$ такая, что $\supp\phi \subset \overline{\Delta}$ и
\begin{equation} \label{ch17eq7}
\int_{\Delta} |f(x) - \phi(x)|\,dx < \epsilon.
\end{equation}
\end{thm}
\begin{proof}
Выберем некоторое $\epsilon > 0$ и построим финитную ступенчатую функцию $\phi(x)$, удовлетворяющую неравенству.

Из определения абсолютно интегрируемой функции $f$ на промежутке $\Delta$ следует, что существует ограниченное измеримое множество $g_{\epsilon} \subset \Delta$, на котором функция $f$ интегрируема по Риману, причем
$$
\int_{\Delta} |f(x)|\,dx - \int_{g_{\epsilon}} |f(x)|\,dx = \int_{\Delta\setminus g_{\epsilon}} |f(x)|\,dx
$$

Через $f_{\epsilon}(x)$ обозначим функцию, равную $f(x)$ на $g_{\epsilon}$ и нулю вне $g_{\epsilon}$. Очевидно,
\begin{equation} \label{ch17eq8}
\int_{\Delta} |f(x) - f_{\epsilon}(x)|\,dx = \int_{\Delta\setminus g_{\epsilon}} |f(x)|\,dx < \frac{\epsilon}{2}
\end{equation}

Функция $f_{\epsilon}(x)$ равна нулю вне ограниченного множества $g_{\epsilon} \subset \Delta$, поэтому ее носитель содержится в некотором отрезке $[a;b] \subset \overline{\Delta}$. На этом отрезке она интегрируема по Риману, так как она интегрируема по Риману на $g_{\epsilon}$ и на $[a;b]\setminus g_{\epsilon}$. Поэтому существует разбиение $\tau$ отрезка $[a;b]$ на промежутки $\Delta_1$, \ldots,~$\Delta_N$ такое, что
$$
\int_{a}^{b} f_{\epsilon}(x)\,dx - s(f_{\epsilon};\tau) < \frac{\epsilon}{2},
$$
где $s(f_{\epsilon};\tau)$ "--- нижняя сумма Дарбу функции $f_{\epsilon}(x)$, т.е.
$$
s(f_{\epsilon};\tau) = \sum_{j = 1}^{N} m_{j}|\Delta_j|,
$$
где $m_j = \inf\limits_{x \in \Delta_j}\,f_{\epsilon}(x)$. Через $\phi(x)$ обозначим ступенчатую функцию, которая равна $m_j$ на $\Delta_j$, $j = 1,\ldots,N$, и нулю вне $[a;b]$. Очевидно, $\phi(x) \le f_{\epsilon}(x)$ для любого $x \in \bbR$,
$$
\int_{\Delta} \phi(x)\,dx = \int_{a}^{b} \phi(x)\,dx = s(f_{\epsilon};\tau),
$$
и, следовательно,
\begin{equation} \label{ch17eq9}
\int_{\Delta} |f(x) - \phi(x)|\,dx = \int_{a}^{b} f_{\epsilon}(x)\,dx - s(f_{\epsilon};\tau) < \frac{\epsilon}{2}
\end{equation}
Из неравенств \eqref{ch17eq8} и \eqref{ch17eq9} следует, что финитная ступенчатая функция~$\phi(x)$ удовлетворяет неравенству~\eqref{ch17eq7} теоремы. Действительно:
\begin{multline*}
\int_{\Delta} |f(x) - \phi(x)|\,dx \le \int_{\Delta} |f(x) - f_{\epsilon}(x)|\,dx+\\
+\int_{\Delta} |f(x) - \phi(x)|\,dx < \frac{\epsilon}{2} + \frac{\epsilon}{2} = \epsilon \tag*{\qedhere}
\end{multline*}
\end{proof}

\begin{thm}[Римана об осцилляции\rindex{теорема!Римана о осцилляции}] \label{ch17thm2}
Если функция $f(x)$ абсолютно интегрируема на промежутке $\Delta$, то
\begin{equation} \label{ch17eq10}
\lim_{\lambda \to \infty}\int_{\Delta} f(x)e^{i\lambda x}dx = 0
\end{equation}
\end{thm}
\begin{proof}
Для характеричстических функций любого конечного промежутка это утверждение очевидно. Действительно, если $\xi$ и $\eta$ "--- концы промежутка, то
$$
\left|\int_{\xi}^{\eta} e^{i\lambda x}\,dx\right| = \left|\cfrac{e^{i\lambda \eta} - e^{i\lambda \xi}}{i\lambda}\right| \le \frac{2}{|\lambda|}.
$$

А так как любая финитная ступенчатая функция есть линейная комбинация характеристических функций конечного числа конечных промежутков, то утверждение теоремы верно для любой такой функции.

Согласно теореме~\ref{ch17thm1}, для любого $\epsilon > 0$ существует финитная ступенчатая функция $\phi(x)$ такая, что
$$
\int_{\Delta} |f(x) - \phi(x)|\,dx < \frac{\epsilon}{2}.
$$

Тогда
$$
\left|\int_{\Delta} f(x)e^{i\lambda x}\,dx\right| \le \int_{\Delta} |f(x) - \phi(x)|\,dx\ +\ \left|\int_{\Delta} \phi(x)e^{i\lambda x}\,dx\right|.
$$

Последний интеграл, согласно уже доказанному, стремится к нулю при $\lambda \to \infty$, поэтому
$$
\exists \lambda_{\epsilon} \cquad \forall \lambda, |\lambda| > \lambda_{\epsilon} \quad \left|\int_{\Delta} \phi(x)e^{i\lambda x}\,dx\right| < \frac{\epsilon}{2}.
$$
Следовательно,
$$
\left|\int_{\Delta} f(x)e^{i\lambda x}\,dx\right| < \epsilon \quad \forall \lambda, |\lambda| > \lambda_{\epsilon},
$$
что и доказывает равенство~\eqref{ch17eq10}.
\end{proof}

Отметим, что здесь параметр $\lambda$ может принимать любое значение из $\bbR$ и, в частности, может стремиться как к $+\infty$, так и к $-\infty$. Следовательно, наряду с~\eqref{ch17eq10} выполняется и равенство
$$
\lim_{\lambda \to \infty}\int_{a}^{b} f(x)e^{-i\lambda x}\,dx = 0
$$

\begin{cons}
Если функция $f(x)$ абсолютно интегрируема на промежутке $\Delta$, то
\begin{align*}
&\lim_{\lambda \to \infty}\int_{\Delta} f(x)\cos{\lambda x}\,dx = 0\\
&\lim_{\lambda \to \infty}\int_{\Delta} f(x)\sin{\lambda x}\,dx = 0
\end{align*}
\end{cons}

\section{Признаки сходимости тригонометрических рядов Фурье в точке}
\subsection{Признак Липшица}

Пусть функция $f(x)$ определена в некоторой окрестности точки $x_0$. Говорят, что она в этой точке удовлетворяет \textit{условию Липшица порядка} $\alpha > 0$, если существуют постоянные $C$ и $\delta > 0$ такие, что 

\begin{equation} \label{ch17eq4}
|f(x_0 + \xi) - f(x_0)| \le C |\xi|^\alpha \quad \forall \xi \in (-\delta; \delta).
\end{equation}

\begin{thm} [Признак Липшица\rindex{признак!Липшица} сходимости ряда Фурье в точке] \label{ch17thm3}
Если функция $f(x) \in \accentset{*}{L}^{R}_1(-\pi;\pi)$ в точке $x_0 \in \bbR$ удовлетворяет условию Липшица порядка $\alpha > 0$, то ее ряд Фурье в точке $x_0$ сходится к $f(x_0)$.
\end{thm}

\begin{proof}
Напомним, что для $n$-й частичной суммы ряда Фурье функции $f$ справедлива формула
$$
T_n(f;x_0) = \frac{1}{2\pi} \int_{-\pi}^{\pi} f(x_0 + \xi) D_n(\xi) \,d\xi,
$$
где
$$
D_n(\xi) = \frac{\sin{\lambda_n \xi}}{\sin{\frac{\xi}{2}}}, \, \frac{1}{2\pi}\int_{-\pi}^{\pi} D_n(\xi) \,d\xi = 1, \, \lambda_n = n + \frac12.
$$

Поэтому 
$$
T_n(f; x_0) - f(x_0) = \frac{1}{2\pi}\int_{-\pi}^{\pi}(f(x_0 + \xi) - f(x_0))D_n(\xi)\,d\xi.
$$

По условию функция $f$ в точке $x_0$ удовлетворяет условию Липшица~\eqref{ch17eq4}. Не ограничивая общности, можно считать, что $\delta < \pi$. Тогда функция 
$$
F(\xi) = \frac{f(x_0 + \xi) - f(x_0)}{\sin{\frac{\xi}{2}}}
$$
удовлетворяет неравенству
$$
|F(\xi)| \le \frac{C|\xi|^\alpha}{\left| \sin{\frac{\xi}{2}} \right|} \quad \forall \xi \in (-\delta; \delta).
$$
А так как функция $\sin{\frac{\xi}{2}}$ выпукла вверх на отрезке $[0; \pi]$, то
$$
\sin{\frac{\xi}{2}} \ge \frac{\xi}{\pi} \quad \forall \xi \in [0; \pi],
$$
и поэтому 
$$
|F(\xi)| \le \pi C |\xi|^{\alpha - 1} \quad \forall \xi \in [0; \pi].
$$

Следовательно,
\begin{multline} \label{ch17eq5}
|T_n(f; x_0) - f(x_0)| \le \frac{1}{2} C \int_{-h}^{h} |\xi|^{\alpha - 1}\,d\xi + \\
+ \frac{1}{2\pi} \left| \int_{-\pi}^{-h} F(\xi) \sin{\lambda_n \xi} \,d\xi \right| + \frac{1}{2\pi} \left| \int_{h}^{\pi} F(\xi) \sin{\lambda_n \xi} \,d\xi \right|
\end{multline}
для любого $h \in (0; \delta)$ и любого $n \in \bbN$. Для каждого $h \in (0;\delta)$ функция~$F(\xi)$ абсолютно интегрируема на интервалах $(-\pi; -h)$ и $(h; \pi)$. Согласно теореме Римана об осцилляции, интегралы по этим интервалам стремятся к нулю при $n \to \infty$. Поэтому из \eqref{ch17eq5} следует неравенство
$$
\overline{\lim\limits_{n \to \infty}}|T_n(f; x_0) - f(x_0)| \le C \cdot \frac{1}{\alpha}h^{\alpha}.
$$
А так как оно справедливо для любого $h \in (0; \delta)$, то 
$$
\overline{\lim\limits_{n \to \infty}}|T_n(f; x_0) - f(x_0)| = 0.
$$

Очевидно, если верхний предел неотрицательной последовательности равен нулю, то это последовательность сходится и ее предел равен нулю.
\end{proof}

\begin{cons}
Если функция $f(x) \in \accentset{*}{L}^{R}_1(-\pi;\pi)$ в точке $x_0 \in \bbR$ дифференцируема или непрерывна и имеет конечные односторонние призводные, то в этой точке ее ряд Фурье сходится к $f(x_0)$.
\end{cons}

\begin{cons}
Если функция $f(x), \: x \in [a;b], \: a < b,$ на интервале $(a;b)$ непрерывна, абсолютно интегрируема и в каждой точке $x \in (a;b)$ дифференцируема или имеет конечные односторонние производные, то в любой точке $x \in (a;b)$ ее ряд Фурье сходится и 
$$
f(x) = \frac{a_0}{2} + \sum_{n = 1}^{\infty} a_n \cos{n \frac{\pi}{l} x} + b_n \sin{n \frac{\pi}{l} x},
$$
где $l = \frac{b - a}{2}$, $a_0 = a_0(f)$, $a_n = a_n(f)$, $b_n = b_n(f)$. Если, кроме того, функция $f$ непрерывна на $[a;b]$, $f(a) = f(b)$ и существуют конечные односторонние производные $f'_+(a)$ и $f'_-(b)$, то и в точках $x = a$, $x = b$ ряд Фурье сходится к $f(x)$.
\end{cons}

%TODO Нужны ли односторонние условия Липшица

\subsection{Признак Дини}
Пусть функция $f(x)$ определена в окрестности точки $x_0$ и в этой точке непрерывна. Если, кроме того, существует $\delta > 0$ такое, что разностное отношение
$$
\frac{f(x_0 + \xi) - f(x_0)}{\xi}
$$
абсолютно интегрируемо на интервале $(-\delta;\delta)$, то говорят, что функция~$f(x)$ удовлетворяет \textit{условию Дини}.

Очевидно, если функция $f(x)$ в точке $x_0$ удовлетворяет условию Липшица порядка $\alpha > 0$, то в этой точке она удовлетворяет и условию Дини. Обратное утверждение является неверным.

\begin{thm} [Признак Дини сходимости ряда Фурье в точке] \label{ch17thm4}
Если функция $f(x) \in \accentset{*}{L}^{R}_1(-\pi;\pi)$ в точке $x_0 \in \bbR$ удовлетворяет условию Дини, то ее ряд Фурье в этой точке сходится к $f(x_0)$.
\end{thm}
\begin{proof}
Как и при доказательстве теоремы \ref{ch17thm3} для $h \in (0;\delta)$ получаем неравенство
$$
\overline{\lim\limits_{n \to \infty}}|T_n(f; x_0) - f(x_0)| \le \frac{1}{2\pi}\int_{-h}^{h}|F(\xi)|d\xi,
$$
где
$$
|F(\xi)| \le \pi\left|\frac{f(x_0 + \xi) - f(x_0)}{\xi}\right|.
$$
Отсюда и из условия Дини следует, что
\begin{equation*}
\overline{\lim\limits_{n \to \infty}}|T_n(f; x_0) - f(x_0)| = 0. \qedhere
\end{equation*}
\end{proof}

%TODO Нужны ли односторонние условия Дини

\subsection{Признак Дирихле}
%TODO Тут дофига используется, в том числе односторонние условия и ядро Дирихле. Требует ревизии
Будем говорить, что функция $f(x)$, определенная в некоторой окрестности точки $x_0$, в этой точке удовлетворяет \textit{условию Дирихле}, если существует $\delta > 0$ такое, что на интервалах $(x_0 - \delta;x_0)$ и $(x_0;x_0 + \delta)$ она монотонна и ограничена.

Очевидно, функция $f(x)$, удовлетворяющая условию Дирихле в точке~$x_0$, в этой точке имеет конечные односторонние пределы $f(x_0\pm 0)$.
\begin{thm} \label{ch17thm5}
Если функция $f(x) \in \accentset{*}{L}^{R}_1(-\pi;\pi)$ в точке $x_0 \in \bbR$ удовлетворяет условию Дирихле, то ее ряд Фурье в этой точке сходится к $M_f(x_0)$.
\end{thm}
\begin{proof}
Как известно,
\begin{multline*}
|T_n(f; x_0) - M_f(x_0)| \le \frac{1}{2\pi} \sum_{\pm} \left|\int_{h}^{\pi} F_{\pm}(\xi) \sin{\lambda_n \xi} \,d\xi \right|\ +\\
+\ \frac{1}{2\pi} \sum_{\pm} \left|\int_{0}^{h} (f(x_0\pm \xi) - f(x_0\pm 0))D_n(\xi) \,d\xi \right|
\end{multline*}
для любого $h \in (0;\pi)$.

Согласно условию Дирихле, существует $\delta > 0$ такое, что функции~$f(x_0\pm \xi)$ на интервале $(0;\delta)$ монотонны и ограничены. По второй теореме о среднем для любого $h \in (0;\delta)$ существуют $\theta_{\pm} \in [0;h]$ такие, что
$$
\int_{0}^{h} (f(x_0\pm \xi) - f(x_0\pm 0))D_n(\xi) \,d\xi\ = (f(x_0\pm h) - f(x_0\pm 0))\, \int_{\theta_{\pm}}^{h} D_n(\xi)\,d\xi
$$

Как известно, ядро Дирихле обладает следующим свойством: существует постоянная $C > 0$ такая, что
$$
\left|\int_{\alpha}^{\beta} D_n(\xi)\,d\xi\right| \le 2C\pi
$$
для любых $\alpha, \beta \in (0;\pi)$ и любого $n \in \bbN$. Следовательно для любого $h \in (0;\delta)$ и любого $n \in \bbN$ справедливо неравенство
\begin{multline*}
|T_n(f; x_0) - M_f(x_0)| \le\\
\le C\sum_{\pm}|f(x_0\pm h) - f(x_0\pm 0)| + \frac{1}{2\pi} \sum_{\pm} \left|\int_{h}^{\pi} F_{\pm}(\xi) \sin{\lambda_n \xi} \,d\xi \right|,
\end{multline*}
из которого следует, что
$$
\overline{\lim\limits_{n \to \infty}}|T_n(f; x_0) - M_f(x_0)| \le C\sum_{\pm}|f(x_0\pm h) - f(x_0\pm 0)|
$$
для любого $h \in (0;\delta)$. А так как в последнем неравенстве правая часть стремится к нулю при $h \to +0$, то
\begin{equation*}
\lim\limits_{n \to \infty}|T_n(f; x_0) - M_f(x_0)| = 0. \qedhere
\end{equation*}
\end{proof}

\begin{cons}
Если функция $f(x)$ ограничена и кусочно-монотонна на интервале $(-\pi;\pi)$, то в любой точке $x \in (-\pi;\pi)$ ее ряд Фурье сходится к
$$
M_f(x) = \frac{f(x+0) + f(x-0)}{2},
$$
а в точках $-\pi$ и $\pi$ он сходится к $\frac{f(-\pi+0) - f(\pi-0)}{2}$. В частности, в точках $x \in (-\pi;\pi)$, где $f$ непрерывна, ряд Фурье сходится к $f(x)$.
\end{cons}
\begin{cons}
Если непрерывная $2\pi$-периодическая функция $f$ кусочно-монотонна на отрезке $[-\pi;\pi]$, то в любой точке $x \in \bbR$ ее ряд Фурье сходится к $f(x)$.
\end{cons}
