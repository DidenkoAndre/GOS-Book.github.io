\chapter{Формула Стокса.}
\section{Формула Стокса}
\subsection{Формула Стокса для гладкой параметрически заданной поверхности}
Рассмотрим гладкую параметрически заданную поверхность с краем
\begin{equation} \label{yaa14e1}
S=\{\vv{r}(u,v),\quad(u,v)\in\overline{D} \}
\end{equation}

Будем предполагать, что здесь $D$ "--- это ограниченная область, граница которой состоит из конечного числа кусочно-гладких контуров. Тогда граница (край) поверхности \eqref{yaa14e1} тоже состоит из конечного числа кусочно-гладких контуров. Любую такую поверхность $S$ будем называть \textit{гладкой поверхностью с кусочно-гладкой границей}.

Если носитель поверхности $S$ содержится в некотором множестве $G\subset \bbR^3$, то будем говорить, что поверхность $S$ лежит в $G$, и писать $S\subset G$.

\begin{thm}
Если функции $P(x,y,z),\, Q(x,y,z),\, R(x,y,z)$ определены и непрерывно дифференцируемы в области $G\subset \bbR^3$, то для любой гладкой поверхности $S\subset G$ c кусочно-гладкой границей $\partial S$ справедливы формулы
\begin{equation} \label{yaa14e2}
\int\limits_{\partial S} P\,dx = \iint\limits_{S} \pd{P}{z}\,dz\,dx-\pd{P}{y}\,dx\,dy,
\end{equation}
\begin{equation} \label{yaa14e3}
\int\limits_{\partial S} Q\,dy = \iint\limits_{S} \pd{Q}{x}\,dx\,dy- \pd{Q}{z}\,dy\,dz,
\end{equation}
\begin{equation} \label{yaa14e4}
\int\limits_{\partial S} R\,dz =\iint\limits_{S} \pd{R}{y}\,dy\,dz- \pd{R}{z}\,dz\,dx,
\end{equation}
где ориентации поверхности $S$ и ее границы $\partial S$ согласованы.
\end{thm}
\begin{proof}
Пусть гладкая поверхность $S$ задана уравнением \eqref{yaa14e1} и ориентировна своими параметрами, а граница $\partial D$ области $D$ состоит из одного или нескольких кусочно-гладких кусков вида $\gamma = \{u(t),v(t),t\in[a;b]\}$, каждый из которых параметром $t$ ориентирован положительно относительно области $D$. Тогда, как известно (см.выше???), граница $\partial S$ поверхности $S$ состоит из конечного числа кусочно-гладких кусков вида
$$
\Gamma = \{\vv{r}(u(t),v(t)),\quad t\in[a;b]\},
$$
ориентация каждого из которых параметром $t$ согласована с ориентацией поверхности $S$.

Для каждого куска $\Gamma\subset\partial S$ по формуле замены переменных
$$
\int\limits_{\Gamma} P\,dx = \int\limits_\gamma (Px'_u)\,du+(Px'_v)\,dv.
$$
Просуммировав по всем $\Gamma\subset\partial S$, получим
\begin{equation} \label{yaa14e5}
\int\limits_{\partial S} P\,dx = \int\limits_{\partial D} (Px'_u)\,du+(Px'_v)\,dv.
\end{equation}

При дополнительном условии, что функция $x(u,v)$ имеет непрерывную смешанную производную, интеграл по $\partial D$, согласно формуле Грина, равен следующему двойному интегралу:
$$
\iint\limits_D\left(\pd{(Px_v)}{u} -\pd{(Px'_u)}{v}\right)\,du\,dv.
$$
Преобразуем подынтегральное выражение этого интеграла:
\begin{multline*}
\pd{(Px'_v)}{u} - \pd{(Px'_u)}{v} = \left(\pd{P}{x}x'_u+\pd{P}{y}y'_u+\pd{P}{z}z'_u\right)x'_v-\left(\pd{P}{x}x'_v+\pd{P}{y}y'_v+\pd{P}{z}z'_v\right)x'_u =\\= \pd{P}{y}(y'_ux'_v-y'_vx'_u)+\pd{P}{z}(z'_ux'_v-z'_vx'_u)=-\pd{P}{y}\cdot \pd{(x,y)}{(u,v)}+\pd{P}{z}\cdot\pd{(z,x)}{(u,v)}.
\end{multline*}
В результате получим равенство
\begin{equation} \label{yaa14e6}
\iint\limits_{\partial D} (Px'_u)\,du+(Px'_v)\,dv = \int\limits_D\left(\pd{P}{z}\cdot\pd{(z,x)}{(u,v)}-\pd{P}{y}\cdot \pd{(x,y)}{(u,v)}\right)\,du\,dv.
\end{equation}
Отметим, что оно справедливо и без предположения, что функция $x(u,v)$ имеет непрерывную смешанную производную. (Это утверждение можно доказать предельным переходом.)

Из равенств \eqref{yaa14e5}, \eqref{yaa14e6} и формулы для вычисления поверхностных интегралов следует, что
$$
\int\limits_{\partial S} P\,dx = \iint\limits_D \pd{P}{z}\cdot\pd{(z,x)}{(u,v)}du\,dv -\iint\limits_D \pd{P}{y}\cdot\pd{(x,y)}{(u,v)}du\,dv = \iint\limits_S \pd{P}{z}dz\,dx-\pd{P}{y}dx\,dy.
$$
Формула \eqref{yaa14e2} доказана. Формулы \eqref{yaa14e3},\eqref{yaa14e4} доказываются аналогично.

Теорема доказана.
\end{proof}

Сложив равенства \eqref{yaa14e2}"--- \eqref{yaa14e4}, получим формулу
\begin{multline*}
\int\limits_{\partial S} P\,dx+Q\,dy+R\,dz = \iint\limits_S \left(\pd{R}{y}-\pd{Q}{z}\right)dy\,dz + \left(\pd{P}{z}-\pd{R}{x}\right)dz\,dx+\\+\left(\pd{Q}{x}-\pd{P}{y}\right)dx\,dy.
\end{multline*}
Эта формула (как, в частности, любая из формул \eqref{yaa14e2}-\eqref{yaa14e4}) называется \textit{формулой Стокса}\rindex{формула!Стокса}.

Если поверхность $S$ лежит на плоскости $z=0$, то формула Стокса превращается формулу Грина: 
$$
\int\limits_S P\,dx+Q\,dy = \iint\limits_S\left(\pd{Q}{x}-\pd{P}{y}\right)dx\,dy.
$$


