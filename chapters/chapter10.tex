\chapter[Экстремумы функций нескольких переменных. Необходимые условия, достаточные условия.]{Экстремумы функций нескольких переменных. Необходимые условия, достаточные условия\footnotemark.}
\footnotetext{Снова на всякий случай предупрежу, что материал этого билета читался в 3 семестре, на лекциях по \glqq Кратным интегралам и теории поля\grqq, однако по разумным соображениям я отношу этот материал к многомерному анализу.}

\section{Экстремумы функций нескольких переменных}

Пусть $f: D_f \to \bbR, \quad D_f \subset \bbR^n$

\begin{defn}
Точка $x_0 \in D_f$ будет называться точкой \textit{локального минимума (максимума)}, если 
\begin{equation} \label{ch10eq1}
\exists \delta > 0 \cquad \forall x \in D_f \cap O_\delta(x_0) \cquad f(x) \ge f(x_0)\quad \bigl(f(x) \le f(x_0)\bigr)
\end{equation} 
Точка $x_0$ будет называться точкой \textit{строгого минимума (максимума)} :
\begin{equation}\label{ch10eq1'}
\exists \delta > 0\cquad \forall x \in D_f \cap O_\delta(x_0) \cquad f(x) > f(x_0)\quad \bigl(f(x) < f(x_0)\bigr)
\tag{\ref{ch10eq1}$'$} 
\end{equation} 
\end{defn}

\begin{defn}
Точка $x_0 \in D_f$ называется точкой локального экстремума, если она является либо точкой локального минимума, либо точкой локального максимума функции $f$.

Точка $x_0$ не является точкой локального экстремума $\Leftrightarrow$ $$\forall \delta > 0 \quad \exists x_1, x_2 \in D_f \cap \overset{\circ}{O}_\delta(x_0) \cquad f(x_1) < f(x_0) < f(x_2)$$
\end{defn}
\section{Необходимые условия, достаточные условия}

\begin{thm}[Необходимое условие экстремума] \label{ch10thm1}
Пусть фунция $f$ дифференцируема в точке $x_0 \in \bbR^n$. Тогда если $x_0$ является точкой локального экстремума функции $f$, то $\frac{\partial f}{\partial x_j}(x_0) = 0 \quad \forall j \in \overline{1,n}$, т.е. $\grad f(x_0) = \overset{\rightarrow}{\theta}.$
\end{thm}

\begin{defn}
Точка $x_0 \in D_f$ называется стационарной точкой функции $f$, если функция $f$ дифференцируема в точке $x_0$ и $\grad f(x_0) = \overset{\rightarrow}{\theta}$.
\end{defn}

\begin{thm}
Пусть функция $f$ определена и имеет частные производные 2-го порядка в некоторой окрестности точки $x_0 \in \bbR^n$, причем частные производные 2-го порядка непрерывны в точке $x_0$.

Тогда, если $x_0$ "--- точка локального максимума функции $f$, то $\nabla f(x_0) = \overset{\rightarrow}{\theta}; \quad \Phi_2(x_0, h) \le 0 \quad \forall h \in \bbR^n.$
\end{thm}

\begin{proof}
$\nabla f(x_0) = \overset{\rightarrow}{\theta}$ согласно Теореме $\ref{ch10thm1}$. Согласно формуле Тейлора с остаточным членом в форме Пеано (т.к. $df(x_0) = 0$)
\begin{equation} \label{ch10eq3}
f(x) = f(x_0) + \frac{1}{2} \Phi_2(x_0, h) + \alpha(x)|h|^2, \quad \text{где} \quad h = x - x_0 \in \bbR \quad \text{при} \quad \alpha(x) \overset{x \to x_0}{\longrightarrow} 0.
\end{equation}

$\forall x \in D_f \setminus x_0$ положим $\xi = \frac{x - x_0}{|x - x_0|}; \quad \rho = |x - x_0| \quad \Rightarrow \quad \rho > 0, \quad |\xi| = 1$.

$h = x - x_0 = \rho \xi \quad \Rightarrow \quad \Phi_{2}(x_0, h) = \Phi_2(x_0, \rho \xi) = \rho^2 \Phi_(x_0, \xi)$, т.к. $\Phi_2$ "--- квадратичная форма $\Rightarrow \frac{f(x) - f(x_0)}{\rho^2} = \frac{1}{2}\Phi_2(x_0, \xi) + \alpha(x)$

Т.к. $x_0$ "--- точка локального максимума, то $\exists \delta_0: \quad \forall x \in O_\delta(x_0) \cap D_f$. $f(x) - f(x_0) \le 0 \quad \Rightarrow \quad \forall \xi \in S_1(\bbR^n), (|\xi| = 1): \Phi_2(x_0, \xi) = 2 \frac{1}{\rho^2}(f(x) - f(x_0)) - \alpha(x)$

$\Phi_2(x_0, \xi) = \lim\limits_{\rho \to 0} \left(2 \frac{1}{\rho^2}(f(x) - f(x_0)) - \alpha(x) \right)$

$\exists \lim\limits_{\rho \to 0} \alpha(x) = 0$, $\Phi_2(x_0, \xi)$ от $\rho$ не зависит, правая часть имеет предел $\Longrightarrow \quad \exists \lim\limits_{\rho \to 0} \left( \frac{1}{\rho^2}(f(x) - f(x_0) \right) \le 0$

$\Rightarrow \quad \forall \xi \in S_1(\bbR^n): \Phi_2(x_0, \xi) \le 0$

$\Rightarrow \quad \forall h \in \bbR^n: \quad \Phi_2(x_0, h) = |h^2| \cdot \Phi_2\left( x_0, \frac{h}{|h|} \right) \le 0$ 
\end{proof}

\begin{thm} [Достаточное условие экстремума] \label{ch10thm3}
Пусть функция $f$ определена и 2-ды непрерывно-дифференцируема в некоторой окрестности точки $x_0 \in \bbR^n$. Тогда если $\nabla f(x_0) = \vv{\theta}$ и $\Phi_2(x_0, h) > 0 \quad \forall h \not= \vv{\theta}$ , то $x_0$ "--- точка строгого локального минимума функции $f$.
\end{thm}

\begin{proof}
Т.к. $\Phi_2$ непрерывна на компакте $S_1(\bbR^n)$, то она достигает минимума в точке $\xi_0 \in S_1(\bbR^n):  m = \Phi_2(x_0, \xi_0) = \inf\limits_{|\xi| = 1} \Phi_2(x_0, \xi) > 0$

Справедлива формула $\eqref{ch10eq3} \quad \Rightarrow$
$$
f(x) - f(x_0) = \frac{1}{2}|h^2| \cdot \left( \Phi_2 \left( x_0, \frac{\overline{h}}{|h|} \right) + \alpha(x) \right) \ge \frac{1}{2} |h|^2(m + \alpha(x))
$$

\begin{multline*}
\alpha(x) \overset{x \to x_0}{\Longrightarrow} 0 \\ \Rightarrow\\ \Rightarrow  \; \exists \delta > 0: \forall x \in \overset{\circ}{O}_\delta(x_0) \cap D_f: f(x) - f(x_0) \ge \frac{1}{2}|h|^2 \left( m - \frac{m}{2} \right) = \frac{m}{4}|h|^2 > 0 \\ \quad \Rightarrow \quad \text{Получили теорему } \ref{ch10thm3}
\end{multline*}
\end{proof}