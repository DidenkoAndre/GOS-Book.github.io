%% my macros

%% Math fonts
\newcommand{\bbA}{\mathbb{A}}
\newcommand{\bbB}{\mathbb{B}}
\newcommand{\bbC}{\mathbb{C}} % комплексные числа
\newcommand{\bbD}{\mathbb{D}} % дисперсия
\newcommand{\bbE}{\mathbb{E}} % математическое ожидание
\newcommand{\bbF}{\mathbb{F}}
\newcommand{\bbG}{\mathbb{G}}
\newcommand{\bbH}{\mathbb{H}}
\newcommand{\bbI}{\mathbb{I}}
\newcommand{\bbJ}{\mathbb{J}}
\newcommand{\bbK}{\mathbb{K}}
\newcommand{\bbL}{\mathbb{L}}
\newcommand{\bbM}{\mathbb{M}}
\newcommand{\bbN}{\mathbb{N}} %натуральные числа
\newcommand{\bbO}{\mathbb{O}}
\newcommand{\bbP}{\mathbb{P}} % вероятность
\newcommand{\bbQ}{\mathbb{Q}} % рациональные числа
\newcommand{\bbR}{\mathbb{R}} % действительные числа
\newcommand{\bbS}{\mathbb{S}}
\newcommand{\bbT}{\mathbb{T}}
\newcommand{\bbU}{\mathbb{U}}
\newcommand{\bbV}{\mathbb{V}}
\newcommand{\bbW}{\mathbb{W}}
\newcommand{\bbX}{\mathbb{X}}
\newcommand{\bbY}{\mathbb{Y}}
\newcommand{\bbZ}{\mathbb{Z}} %целые числа

%Матожидание и Дисперсия. (старая версия -- сейчас не используем)
\newcommand{\ccM}{\mathcal{M}}
\newcommand{\ccD}{\mathcal{D}}

% Привычное написание букв каппа, эпсилон и фи и знаков "больше, или равно", "меньше, или равно", пустого множества
\renewcommand{\kappa}{\varkappa }
\renewcommand{\epsilon}{\varepsilon}
\renewcommand{\phi}{\varphi}
\renewcommand{\le}{\leqslant}
\renewcommand{\ge}{\geqslant}
\renewcommand{\emptyset}{\varnothing}

%Возможно правильная расстановка пробелов в кванторах - НЕ ИСПОЛЬЗУЙТЕ НИ ЗА ЧТО НА СВЕТЕ, ЛУЧШЕ РУКАМИ. 

%они еще есть в документе, только потому что я дурак. надо удалять их по максимуму.
\renewcommand{\fa}{\quad\forall}%renewcommand из-за package fontawesome, который добавляет кучу красивых символов (смайлики, рожицы, значки)
\newcommand{\ex}{\quad\exists}

%Множества с чертой:
\newcommand{\bboR}{\overline{\mathbb{R}}}
\newcommand{\bboC}{\overline{\mathbb{C}}}

%Проколотая окрестность
\newcommand{\dneio}[2]{\overset{\raisebox{0pt}[0pt][0pt]{$\scriptscriptstyle\circ$}}{O}_{#1}({#2})}
\newcommand{\dnei}[1]{\overset{\raisebox{0pt}[0pt][0pt]{$\scriptscriptstyle\circ$}}{O}({#1})}
\newcommand{\coci}[2][]{\overset{\raisebox{0pt}[0pt][0pt]{$\scriptscriptstyle\circ$}}{C^{#1}}{#2}}

% Полная производная
\newcommand{\D}[2]{\frac{d{#1}}{d{#2}}}

% Частная производная
\newcommand{\pd}[2]{\frac{\partial{#1}}{\partial{#2}}}

%%%%%%%%%%%%%%%%%%%%%%%%%%%%%%%%%%%%%%%%%%

% Действительная и мнимая части
\def\Re{\mathop{\mathrm{Re}}\nolimits}
\def\Im{\mathop{\mathrm{Im}}\nolimits}

%Ядро отображения
\def\Ker{\mathop{\mathrm{Ker}}\nolimits}

%Внутренность множества.
\DeclareMathOperator*{\interior}{int} % * предполагает использование \limits

% Носитель
\def\supp{\mathop{\mathrm{supp}}\limits}

% Дивергенция
\def\Div{\mathop{\mathrm{div}}\nolimits}

% Ротор
\def\rot{\mathop{\mathrm{rot}}\nolimits}

% Градиент
\def\grad{\mathop{\mathrm{grad}}\limits}

% Константа
\def\const{\mathop{\mathrm{const}}\nolimits}

% Ранг
\def\rg{\mathop{\mathrm{rg}}\nolimits}

%% Вычет
\def\res{\mathop{\rm res}\limits}

%% Расстояние
\def\dist{\mathop{\rm dist}\limits}

%% Ковариация
\def\cov{\mathop{\rm cov}\limits}

%% Интеграл в смысле главного значения
\def\v.p.{\mathop{\rm v.p.}\nolimits}

%Гиперболические функции
\def\sh{\mathop{\rm sh}\nolimits}
\def\ch{\mathop{\rm ch}\nolimits}
\def\th{\mathop{\rm th}\nolimits}
\def\cth{\mathop{\rm cth}\nolimits}

%Арктангенс
\def\arctg{\mathop{\rm arctg}\nolimits}
\def\arcctg{\mathop{\rm arcctg}\nolimits}

%Обратные гиперболические функции 
\def\arsh{\mathop{\rm arsh}\nolimits}
\def\arch{\mathop{\rm arch}\nolimits}
\def\arth{\mathop{\rm arth}\nolimits}
\def\arcth{\mathop{\rm arcth}\nolimits}

%%%%%%%%%%%%%%%%%%%%%%%%%%%%%%%%%%%%%%%%%%

% Крупная хи
\newcommand{\bigchi}{\text{\scalebox{1.5}{$\chi$}}}

% Римские цифры
\makeatletter
\newcommand*{\rom}[1]{\expandafter\@slowromancap\romannumeral #1@}
\makeatother

%% Text fomats
\newcommand{\tbi}[1]{\textbf{\textit{#1}}}

%% Сокращение для одной штуки

\newcommand{\Gp}{(\partial G)^{+}}

\newcommand{\Gm}{(\partial G)^{-}}

%
%
%
%
%
%
%
%
%
%
%%Это перешло сюда по наследству от теорфиза, пусть будет, мало ли. 

% Скобки (высокие)
\newcommand{\brc}[1]{\left ( {#1} \right )}

% Скобки фигурные (высокие)
\newcommand{\brcr}[1]{\left\{ {#1} \right\}}

% Скобки квадратные (высокие)
\newcommand{\brs}[1]{\left [ {#1} \right ]}

% Усреднение
\newcommand{\avg}[1]{\langle{#1}\rangle}

% Усреднение (высокое)
\newcommand{\avgh}[1]{\left\langle{#1}\right\rangle}

% Бра-вектор
\newcommand{\bra}[1]{\left\langle{#1}\right|}

% Кет-вектор
\newcommand{\ket}[1]{\left|{#1}\right\rangle}

% Скалярное произведение
\newcommand{\bk}[2]{\langle{#1}|{#2}\rangle}

% Скалярное произведение (высокое)
\newcommand{\bkh}[2]{\left\langle{#1}|{#2}\right\rangle}

% Проектор
\newcommand{\proj}[2]{\ket{#1}\bra{#2}}

% Матричный элемент
\newcommand{\bfk}[3]{\langle{#1}|{#2}|{#3}\rangle}

% Матричный элемент (высокий)
\newcommand{\bfkh}[3]{\left\langle{#1}\left|{#2}\right|{#3}\right\rangle}

% Модуль
\providecommand{\abs}[1]{\left\lvert{#1}\right\rvert}

% Норма
\providecommand{\norm}[1]{\lVert#1\rVert}

% След матрицы

\def\sp{\mathop{\mathrm{sp}}\nolimits}
