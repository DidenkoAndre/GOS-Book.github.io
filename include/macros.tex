% Привычное написание букв каппа, эпсилон и фи и знаков "больше, или равно", "меньше, или равно", пустого множества
\renewcommand{\kappa}{\varkappa}
\renewcommand{\epsilon}{\varepsilon}
\renewcommand{\phi}{\varphi}
\renewcommand{\le}{\leqslant}
\renewcommand{\ge}{\geqslant}
\renewcommand{\emptyset}{\varnothing}

% Полная производная
\newcommand{\D}[2]{\frac{d{#1}}{d{#2}}}

% Частная производная
\newcommand{\pd}[2]{\frac{\partial{#1}}{\partial{#2}}}

%Множества с чертой:
\newcommand{\bboR}{\overline{\mathbb{R}}}
\newcommand{\bboC}{\overline{\mathbb{C}}}

% Проколотая окрестность (можно и удалить в будущем)
\newcommand{\dneio}[2]{\accentset{\circ}{O}_{#1}({#2})}
\newcommand{\dnei}[1]{\accentset{\circ}{O}({#1})}
\newcommand{\coci}[2][]{\accentset{\circ}{C}^{#1}{#2}}

% Крупная хи
\newcommand{\bigchi}{\text{\scalebox{1.5}{$\chi$}}}

% Сокращение для одной штуки в ТФКП
\newcommand{\Gp}{(\partial G)^{+}}
\newcommand{\Gm}{(\partial G)^{-}}

% desired spaces in noninline math after colon
\newcommand{\cquad}{{:}\quad}

%%%%%%%%%%%%%%%%%%%%%%%%%%%%%%%%%%%%%%%%%%%%%%%%%%%%%%%%%%%
% QUESTIONABLE MACROSES - убрать их?
% Модуль
\providecommand{\abs}[1]{\left\lvert{#1}\right\rvert}
% Норма
\providecommand{\norm}[1]{\lVert#1\rVert}
%%%%%%%%%%%%%%%%%%%%%%%%%%%%%%%%%%%%%%%%%%%%%%%%%%%%%%%%%%%

% Math fonts
\newcommand{\bbC}{\mathbb{C}} % комплексные числа
\newcommand{\bbD}{\mathbb{D}} % дисперсия
\newcommand{\bbE}{\mathbb{E}} % математическое ожидание
\newcommand{\bbN}{\mathbb{N}} % натуральные числа
\newcommand{\bbP}{\mathbb{P}} % вероятность
\newcommand{\bbQ}{\mathbb{Q}} % рациональные числа
\newcommand{\bbR}{\mathbb{R}} % действительные числа
\newcommand{\bbZ}{\mathbb{Z}} % целые числа

\newcommand{\ccN}{\mathcal{N}}%нормальное распределение

% Действительная и мнимая части
\let\Re\relax
\DeclareMathOperator{\Re}{Re}
\let\Im\relax
\DeclareMathOperator{\Im}{Im}

% Дивергенция
\let\div\relax
\DeclareMathOperator{\div}{div}

% Ядро отображения
\DeclareMathOperator{\Ker}{Ker}

% Внутренность множества.
\DeclareMathOperator*{\interior}{int}

% Носитель
\DeclareMathOperator*{\supp}{supp}

% Ротор
\DeclareMathOperator{\rot}{rot}

% Градиент
\DeclareMathOperator{\grad}{grad}

% Константа
\DeclareMathOperator{\const}{const}

% Ранг
\DeclareMathOperator{\rg}{rg}

% Вычет
\DeclareMathOperator*{\res}{res}

% Расстояние
\DeclareMathOperator*{\dist}{dist}

%% Ковариация
\DeclareMathOperator*{\cov}{cov}

% Интеграл в смысле главного значения
\DeclareMathOperator{\v.p.}{v.p.}

% Обратные гиперболические функции
\DeclareMathOperator{\arsh}{arsh}
\DeclareMathOperator{\arch}{arch}
\DeclareMathOperator{\arth}{arth}
\DeclareMathOperator{\arcth}{arcth}

% Диагональная матрица
\DeclareMathOperator{\diag}{diag}

% След матрицы
\DeclareMathOperator{\tr}{tr}

% След матрицы
\DeclareMathOperator{\sign}{sign}

% значок дуги
\makeatletter
\DeclareFontFamily{U}{tipa}{}
\DeclareFontShape{U}{tipa}{m}{n}{<->tipa10}{}
\newcommand{\arc@char}{{\usefont{U}{tipa}{m}{n}\symbol{62}}}%

\newcommand{\arc}[1]{\mathpalette\arc@arc{#1}}

\newcommand{\arc@arc}[2]{%
  \sbox0{$\m@th#1#2$}%
  \vbox{
    \hbox{\resizebox{\wd0}{\height}{\arc@char}}
    \nointerlineskip
    \box0
  }%
}
\makeatother
